
\chapter{Data fitting with finite elements}
\label{cha:datafitting}


\section{Introduction}

We show here how finite elements can be used for fitting one, two or
three-dimensional fields. The data could, for examples, be a series of
temperature measurememts (as in figure ***) and there is a need to find the
finite element nodal parameteres which, together with the chosen basis
functions, 'fit' the data in a least-sqquares sense. This linear least-squares
probelm is considerd in \secref{cha:linearfieldfitting} Another common
requirement is to fit nodal geometric parameters describing a finite element
surface in 3D space to a set of non-uniformly spced coordinates defining the
original surface. Geometric fitting problems are nonlinear when the $\xi_i$
-coordinates found at the orthogonal projection of the data onto the finite
element mesh are changed by the fitting process. A jucicious choice of
coordinate system can sometimes avoid the problem and keep the geometric
fitting linear, (this situation considers in **\secref{}. Full nonlinear
geometric fitting is considered in \secref)




\section{Linear field fitting}
\label{sec:linearfieldfitting}

\subsection{The linear field fitting problem}

Consider the two-dimensional bilinear element shown in
\figref{fig:linfieldfit}. The element surrounds a set of data points (shown by
the x's in \figref{fig:linfieldfit}) which consist of measured values $u_{d}$,
$d=1..D$, of some field (such as temperature) at specified locations $(x_{d},
y_{d})$. The field fitting problem is to find the values of the finite element
nodal parameters $u_{n}$, $n=1..4$, which minimise the sum of squared
differences $(u(\xi_{d})-u_{d})^{2}$ between $u_{d}$ and the finite element
field $u$ evaluated at the $(\xi_{1},\xi_{2})$ coordinates of data point $d$ 
$(u(\xi_{d}))$.

\begin{figure}[htpb] \centering
  \input{figs/datafitting/linfieldfit.pstex}
  \caption{Least-squares fitting of finite element nodal parameters $u_{n}$,
    $n=1..4$, to measured data values $u_{d}$, $d=1..D$, at point locations
    $(x_{d}, y_{d})$.}
  \label{fig:linfieldfit}
\end{figure}

\subsection{Calculation of data point projections}

The first requirement, therefore, is to find the $\xi_{i}$ coordinates of each
data point $(x_{d}, y_{d}$). For a bilinear element (with the basis functions
given by \eqref{eqn:2,3DE}) this is a straightforward inversion of the
following relations for each data point:

\begin{equation}
  \begin{array}{rcl}
    x_{d}&=&(1-\xi_{1}^{d})(1-\xi_{2}^{d})x_{1}+\xi_{1}^{d}(1-\xi_{2}^{d})x_{2}
    +(1-\xi_{1}^{d})\xi_{2}^{d}x_{3}+\xi_{1}^{d}\xi_{2}^{d}x_{4} \\
    y_{d}&=&(1-\xi_{1}^{d})(1-\xi_{2}^{d})y_{1}+\xi_{1}^{d}(1-\xi_{2}^{d})y_{2}
    +(1-\xi_{1}^{d})\xi_{2}^{d}y_{3}+\xi_{1}^{d}\xi_{2}^{d}y_{4}
  \end{array}
  \label{eqn:datapointposfield}
\end{equation}

where $(x_{n}, y_{n}), n=1..4$, are the specified node positions and $(x_{d},
y_{d}), d=1..D$, are the specified data point positions. To solve
\bref{eqn:datapointposfield} for $\xi_{1}^{d}$ and $\xi_{2}^{d}$ we rearrange
\bref{eqn:datapointposfield} as

\begin{equation}
  \begin{array}{rcl}
    a\xi_{1}^{d}+b\xi_{2}^{d}+c\xi_{1}^{d}\xi_{2}^{d}&=&d \\
    A\xi_{1}^{d}+B\xi_{2}^{d}+C\xi_{1}^{d}\xi_{2}^{d}&=&D 
  \end{array}
  \label{eqn:dataxipos1}
\end{equation}
where
\begin{displaymath}
  a=x_{2}-x_{1}, \quad b=x_{3}-x_{1}, \quad c=x_{1}-x_{2}-x_{3}+x_{4}, \quad
  \mbox{and} \quad d=x_{d}-x_{1}
\end{displaymath}
and
\begin{displaymath}
  A=y_{2}-y_{1}, \quad B=y_{3}-y_{1}, \quad C=y_{1}-y_{2}-y_{3}+y_{4}, \quad
  \mbox{and} \quad D=y_{d}-y_{1}
\end{displaymath}
then
\begin{equation}
  \xi_{2}^{d}=\dfrac{d-a\xi_{1}^{d}}{b+c\xi_{1}^{d}}=\dfrac{D-A\xi_{1}^{d}}
  {B+C\xi_{1}^{d}}
  \label{eqn:dataxipos2}
\end{equation}
or
\begin{equation}
  \alpha(\xi_{1}^{d})^{2}+\beta(\xi_{1}^{d})+\gamma=0
  \label{eqn:dataxipos3}
\end{equation}
where
\begin{displaymath}
  \alpha=Ac-aC, \quad \beta=dC-Dc+Ab-aB, \quad \mbox{and} \quad \gamma=Bd-bD
\end{displaymath}
Solving \bref{eqn:dataxipos3} gives
\begin{equation}
  \xi_{1}^{d}= \left\{ \begin{array}{ll} 
      \dfrac{-\beta\pm\sqrt{\beta^{2}-4\alpha\gamma}}{2\alpha} & 
      \mbox{when } \alpha\neq 0 \\
      -\dfrac{\gamma}{\beta} & \mbox{when } \alpha = 0
    \end{array} \right.
  \label{eqn:dataxipos4}
\end{equation}
and then $\xi_{2}^{d}$ is recovered from \bref{eqn:dataxipos2}.

\subsection{Least squares field fitting}

Once the $\xi_{i}$ coordinate positions of each data point
$\xi_{d}=(\xi_{1}^{d},\xi_{2}^{d})$ is known, and interpolation of the
(unknown) nodal values of $u_{n}$ ($n=1..4$) gives
\begin{displaymath}
  u(\xi_{d})=\Psi_{n}u_{n}
\end{displaymath}
where there is an implied sum over $n=1..4$. Now the weighted sum of squared
differences between this value and the measured value $u_{d}$ for $d=1..D$, is

\begin{equation}
%  {\cal F(\vect{u})}=\dsum_{d=1}^{D}w_{d}\left(\Psi_n(\xi_{d})u_{n}-
%    u_{d}\right)^{2}
  eqn here
  \label{eqn:datasumsqs}
\end{equation}
where $w_{d}$ is the weight for data point $d$. For measured data a good
choice for $w_{d}$ is one over the variance of the error for data point $d$.

Minimising \bref{eqn:datasumsqs} with respect to the nodal parameters $u_{m}$
gives
\begin{equation}
  \delby{{\cal F}}{u_{m}}=2\dsum_{d=1}^{D}w_{d}\left(\Psi_n(\xi_{d})u_{n}-
    u_{d}\right)\Psi_m(\xi_{d}) = 0
\end{equation}
or
\begin{equation}
  \left[\dsum_{d=1}^{D}w_{d}\Psi_m(\xi_{d})\Psi_n(\xi_{d})\right]u_{n}=
  \dsum_{d=1}^{D}w_{d}\Psi_m(\xi_{d})u_{d} \quad \quad m=1..4
  \label{eqn:linfieldfiteqn}
\end{equation}
Note that there is an implied sum over $n=1..4$ on the left hand side of
\eqref{eqn:linfieldfiteqn}. For the single 4-node element shown in
\figref{fig:linfieldfit}. 
\todo{check capital?}
\eqref{eqn:linfieldfiteqn} gives four unknowns
$u_{1}, u_{2}, u_{3}$ and $u_{4}$. If more than one element is used in the
fitting, the matrix and vector on the left and right hand sides, respectively,
of \bref{eqn:linfieldfiteqn} are
\begin{equation}
  E_{mn}=\dsum_{d=1}^{D}w_{d}\Psi_m(\xi_{d})\Psi_n(\xi_{d}) \quad \mbox{and} 
  \quad f_{m}=\dsum_{d=1}^{D}w_{d}\Psi_m(\xi_{d})u_{d}
\end{equation}
which can then be assembled into a global system of equations in exactly the
same fashion as occurs in the finite element solution of a boundary value
problem (see \secref{sec:OdSSHC-2.1}).

\subsection{Gauss point fitting}

An extension to linear field fitting is Gauss point fitting. Consider the
problem of finding the nodal stress field from a finite element analysis. The
finite element formulation for a stress analysis typically solves for
displacements, not stresses (see \chapref{cha:linearelasticity}). Stresses are
a derived quantity and are calculated from the constitutive law from the
strains which are calculated (in terms of nodal displacements) at the Gauss
points. The result of the analysis is that it is only possible to obtain a
Gauss point based description of the stresses. To obtain a nodal field based
description of the stress field we can use a fitting approach very similar to
the one described above for field fitting. The only difference in this case is
that our 'data points' are located at the Gauss points and hence the $\xi_d$
location is just the Gauss point location $\xi_{g}$. The values of the field
$u_{d}$ are just the values of stress calculated at the Gauss points and the
weights for each data point $w_{d}$ are 1.

\section{Linear geometric fitting}
\label{sec:lineargeometricfitting}

\subsection{Introduction}

Another common requirement is to fit nodal geometric parameters describing a
finite element geometry to a set of non-uniformly spaced coordinates defining
the original geometry. Geometric fitting differs from field fitting in that
when fitting with more than one geometric variable we have a non-linear
problem since the mesh data projection is no longer orthogonal.

Geometric fitting problems can be turned into a linear problem in two
ways. The first is to only fit geometries were there is one geometric
variable. The second is to keep the data projections constant throughout the
fit. Both options will now be considered.

\subsection{Fitting with one geometric variable}

A Geometric fitting problem with one variable is linear hence it is desirable
to try and arrange a geometric fitting problem so that only one variable is
being fitted. This can often be achieved by adopting a non-rectangular
cartesian coordinate system. For example consider a fitting problem in polar
coordinates as shown in \figref{fig:polargeomfit}

\todo{missing figure}

\begin{figure}[htpb] \centering
%  \input{figs/datafitting/polargeomfit.pstex}
%  \caption{Fitting the radial coordinate in a polar coordinate mesh. There are
%    four elements and four nodes. The dashed lines show the projections of the
%    data points (x) onto the starting mesh.}
  \label{fig:polargeomfit}
\end{figure}

In this case the orthogonal data point projections $\xi_{d}$ are independent
of the radius and can be easily found from the theta coordinate of the data
point $\theta_{d}$. This independence from radius means we can formulate the
geometric fitting problem with only one geometric variable, $r$ being
fitted. As with the linear field fitting case we can formulate an error
function as the weighted sum of squares of the individual errors, that is

\begin{equation}
  {\cal F(\vect{r})}=\dsum_{d=1}^{D}w_{d}\left(\Psi_{n}(\xi_{d})r_{n}-r_{d}
  \right)^{2}
\end{equation}

Minimising this error function with respect to the nodal radii $r_{n}$ we
obtain
\begin{equation}
  \delby{{\cal F}}{r_{m}}=2\dsum_{d=1}^{D}w_{d}\left(\Psi_n(\xi_{d})r_{n}-
    r_{d}\right)\Psi_m(\xi_{d}) = 0
\end{equation}
or in terms of element stiffness matrices
\begin{equation}
  E_{mn}r_{n}=f_{m}
\end{equation}
where
\begin{equation}
  E_{mn}=\dsum_{d=1}^{D}w_{d}\Psi_m(\xi_{d})\Psi_n(\xi_{d}) \quad \mbox{and} 
  \quad f_{m}=\dsum_{d=1}^{D}w_{d}\Psi_m(\xi_{d})u_{d}
\end{equation}

Another example of a change of coordinate systems is a prolate spheroidal
coordinates for fitting a heart geometry.

\subsection{Linear geometric fitting with fixed $\xi_{d}$ locations}

In some geometric fitting problems there is no appropriate change of 
coordinates and a rectangular cartesian system must be used. This means that,
in general, we require both x and y (and z) in the fit at the same time.
This results in a non-linear problem as the data point projection $\xi_{d}$
will now, in general, change as we change any geometric variable during the
fit. Hence in order to obtain a linear problem the data point projections
must remain fixed during the fit. To see how this works consider the geometric
fitting problem as shown in \figref{fig:fixedxifitbefore}.

\begin{figure}[htpb] \centering
  %\epsfig{file=epsfiles/beforefit.eps,width=15cm}
  \caption{Geometric fitting problem. The three nodal y-coordinates 
    $y_{1}, y_{2}$ and $y_{3}$ of a two linear element basis function mesh are
    fitted to five data points. The initial locations of the nodes are
    $(x_{1},y_{1})=(0,0), (x_{2},y_{2})=(1,0)$ and $(x_{3},y_{3})=(2,0)$. The
    location of the data points are
    $(x^{1},y^{1})=(0.2,1.2),(x^{2},y^{2})=(0.8,1.5),(x^{3},y^{3})=(1.3,1.5),
    (x^{4},y^{4})=(1.5,1.1)$ and $(x^{5},y^{5})=(1.7,0.8)$. The initial $\xi$
    projections are $\xi^{1}=0.2$ and $\xi^{2}=0.8$ in element 1 and
    $\xi^{3}=0.3,\xi^{4}=0.5$ and $\xi^{5}=0.7$ in element 2.}
  \label{fig:fixedxifitbefore}
\end{figure}

Now computing the error functions as a weighted sum of squares we obtain
\begin{equation}
  {\cal F(\vect{r})}=\dsum_{d=1}^{D}w_{d}\left(\Psi_{n}(\xi_{d})y_{n}-y_{d}
  \right)^{2}
\end{equation}

Minimising this error function with respect to the nodal parameters $y_{n}$ we
obtain the element matrix equation
\begin{equation}
  E_{mn}y_{n}=f_{m}
\end{equation}
where, as before,
\begin{equation}
  E_{mn}=\dsum_{d=1}^{D}w_{d}\Psi_m(\xi_{d})\Psi_n(\xi_{d}) \quad \mbox{and} 
  \quad f_{m}=\dsum_{d=1}^{D}w_{d}\Psi_m(\xi_{d})u_{d}
\end{equation}

Now for element 1
\begin{equation}
  \begin{array}{rcl}
    E_{11}&=&(1-\xi^{1})(1-\xi^{1})+(1-\xi^{2})(1-\xi^{2}) \\
    &=&(1-0.2)(1-0.2)+(1-0.8)(1-0.8)=0.68 \\
    E_{12}&=&E_{21}=(1-\xi^{1})\xi^{1}+(1-\xi^{2})\xi^{2} \\
    &=&(1-0.2)(0.2)+(1-0.8)(0.8)=0.32 \\
    E_{22}&=&\xi^{1}\xi^{1}+\xi^{2}\xi^{2} \\
    &=& (0.2)(0.2)+(0.8)(0.8)=0.68 \\
    f_{1}&=&(1-\xi^{1})y^{1}+(1-\xi^{2})y^{2} \\
    &=&(1-0.2)(1.2)+(1-0.8)(1.5)=1.26 \\
    f_{2}&=&\xi^{1}y^{1}+\xi^{2}y^{2} \\
    &=&(0.2)(1.2)+(0.8)(1.5)=1.44
  \end{array}
\end{equation}
and for element 2
\begin{equation}
  \begin{array}{rcl}
    E_{11}&=&(1-\xi^{3})(1-\xi^{3})+(1-\xi^{4})(1-\xi^{4})+(1-\xi^{5})
    (1-\xi^{5})\\
    &=&(1-0.3)(1-0.3)+(1-0.5)(1-0.5)+(1-0.7)(1-0.7)=0.83 \\
    E_{12}&=&E_{21}=(1-\xi^{3})\xi^{3}+(1-\xi^{4})\xi^{4}+(1-\xi^{5})\xi^{5}\\
    &=&(1-0.3)(0.3)+(1-0.5)(0.5)+(1-0.7)(0.7)=0.67 \\
    E_{22}&=&\xi^{3}\xi^{3}+\xi^{4}\xi^{4}+\xi^{5}\xi^{5}\\
    &=&(0.3)(0.3)+(0.5)(0.5)+(0.7)(0.7)=0.83 \\
    f_{1}&=&(1-\xi^{3})y^{3}+(1-\xi^{4})y^{4}+(1-\xi^{5})y^{5}\\
    &=&(1-0.3)(1.5)+(1-0.5)(1.1)+(1-0.7)(0.8)=1.84\\
    f_{2}&=&\xi^{3}y^{3}+\xi^{4}y^{4}+\xi^{5}y^{5}\\
    &=&(0.3)(1.5)+(0.5)(1.1)+(0.7)(0.8)=1.56
  \end{array}
\end{equation}

Assembling these element matrices and vectors into a global system of
equations we get
\begin{equation}
  \begin{bmatrix}
    0.68 & 0.32 & 0.00 \\
    0.32 & 1.51 & 0.67 \\
    0.00 & 0.67 & 0.83 
  \end{bmatrix}
  \begin{bmatrix}
    y_{1} \\
    y_{2} \\
    y_{3}
  \end{bmatrix} =
  \begin{bmatrix}
    1.26 \\
    3.28 \\
    1.56
  \end{bmatrix}
\end{equation}

Solving these gives
\begin{equation}
  \begin{bmatrix}
    y_{1} \\
    y_{2} \\
    y_{3}
  \end{bmatrix} =
  \begin{bmatrix}
    1.03210 \\
    1.74420 \\
    0.47152
  \end{bmatrix}
\end{equation}

The fitted solution is shown in \figref{fig:fixedxifitafter}.
\begin{figure}[htpb] \centering
  %\epsfig{file=epsfiles/afterfit.eps,width=15cm}
  \caption{Fitted mesh. The $\xi_{d}$ locations on the fitted mesh are
    $\xi^{1}=0.212$ and $\xi^{2}=0.752$ in element 1 and $\xi^{3}=0.233,
    \xi^{4}=0.504$ and $\xi^{5}=0.726$ in element 2.}
  \label{fig:fixedxifitafter}
\end{figure}
It should be noted that after the fit the new data point projections have
changed after the fit. Hence it general there maybe some benefit from
reapplying the fitting procedure to the new data point projection i.e. 
iterating on the fit.

\section{Geometric fitting with Hermite elements}

\subsection{Review of cubic Hermite interpolation}

\subsubsection{Cubic Hermite basis functions}

One of the most commonly used basis functions in finite elements are Lagrange
basis functions which preserve continuity of the geometric coordinates across
element boundaries by interpolating nodal coordinates which are shared by
adjacent elements i.e.  $C^{0}$ continuity.  The interpolation formula for
linear Lagrange interpolation is given in \eqref{eqn:linLagrangeinterp}.
\begin{equation}
  \fnof{\vect{x}}{\xi}=\varphi_{1}(\xi)\vect{x}_{1}+\varphi_{2}(\xi)\vect{x}_{2}
  \label{eqn:linLagrangeinterp}
\end{equation}
where $\vect{x}_{n}$ is the geometric position of local node $n$ and the two
one-dimensional linear Lagrange basis functions are given in
\eqref{eqn:linLagrangeBfuns}.
\begin{eqnarray}
  \varphi_{1}(\xi)=1-\xi & \varphi_{2}(\xi)=\xi
  \label{eqn:linLagrangeBfuns}
\end{eqnarray}

Cubic Hermite functions, on the other hand, also preserve continuity of the
derivative of these coordinates with respect to $\xi$ across element
boundaries by defining additional nodal parameters \todo{here}
%\dbyat{\vect{x}}{\xi}{n}
i.e. $C^{1}$ continuity. The interpolation formula within an element is given
by
\begin{equation}
%  \fnof{\vect{x}}{\xi}=\Psi_{1}^{0}(\xi)\vect{x}_{1}+\Psi_{1}^{1}(\xi)\dbyat{\vect{x}}
%  {\xi}{1}+\Psi_{2}^{0}(\xi)\vect{x}_{2}+\Psi_{2}^{1}(\xi)\dbyat{\vect{x}}
%  {\xi}{2}
  \todo{eqn}
  \label{eqn:cubHermxiinterp}
\end{equation}
where the four one-dimensional cubic Hermite basis functions are given in 
\eqref{eqn:cubHermBfuns}.
\begin{eqnarray}
  \Psi_{1}^{0}(\xi)=1-3\xi^{2}+2\xi^{3} & \Psi_{1}^{1}(\xi)=\xi(\xi-1)^{2} 
  \nonumber \\
  \Psi_{2}^{0}(\xi)=\xi^{2}(3-2\xi) & \Psi_{2}^{1}(\xi)=\xi^{2}(\xi-1)
  \label{eqn:cubHermBfuns}
\end{eqnarray}

One further step is required to make cubic Hermite basis functions useful in
practice.  Consider now the two cubic Hermite elements as shown in 
\figref{fig:cubHermelem}.

\begin{figure}[htbp] \centering
  \input{figs/datafitting/cubichermiteelem.pstex}
  \caption{Two cubic Hermite elements (denoted by {\bf 1} and {\bf 2}) formed
    from three nodes (shown as a $\bullet$ and denoted by 1, 2 and 3) and 
    having arc-lengths $s_{1}$ and $s_{2}$.}
  \label{fig:cubHermelem}
\end{figure}

The derivative \todo{here}
%\dbyat{\vect{x}}{\xi}{n} defined at node $n$ is dependent upon
the local element $\xi$-coordinate and is therefore, in general, different in
the two adjacent elements. Therefore we carry a physical derivative
%$\dbyat{\vect{x}}{s}{n}$ at nodes and use
\begin{equation}
%  \dbyat{\vect{x}}{\xi}{n}=\brdby{\vect{x}}{s}_{\Delta(n,e)}.\brdby{s}{\xi}_{e}
  \label{eqn:xitoarclength}
\end{equation}
%to determine \dbyat{\vect{x}}{\xi}{n}. Here \dby{\vect{x}}{s} is a physical arc-
length derivative, $\Delta(n,e)$ is the global node number of local node $n$
%in element $e$, $\brdby{s}{\xi}_{e}$ is an element 'scale factor', denoted by
$S_{e}$, which scales the arc-length derivative to the $\xi$-coordinate
derivative.  Thus $\dby{\vect{x}}{s}$ is constrained to be continuous across
element boundaries rather than $\dby{\vect{x}}{\xi}$.

There is one condition that must be placed on the $\xi$ to arc-length
transformation to ensure that we have arc-length derivatives. This condition
is that the arc-length derivative vector at a node must have unit magnitude,
that is
\begin{equation}
%  \norm{\brdby{\vect{x}}{s}_{n}}=1
  \label{eqn:cubHermnormconst}
\end{equation}
This ensures that we have continuity with respect to a physical parameter
rather than with respect to a mathematical parameter $\xi$. The set of mesh
parameters, \vect{u}, for cubic Hermite interpolation hence contains the set
of nodal values (or positions), the set of nodal arc-length derivatives and
the set of scale factors.

\subsubsection{Bicubic Hermite basis functions}

Bicubic Hermite basis functions are the two-dimensional extension of the
one-dimensional cubic Hermite basis functions. They are formed from the tensor
(or outer) product of two one-dimensional basis functions as defined in
\eqref{eqn:cubHermBfuns}. The interpolation formula for a point $(\xi_{1},
\xi_{2})$ within an element is obtained from the bicubic Hermite interpolation
formula,
\begin{eqnarray}
%\vect{x}(\xi_{1},\xi_{2}) &=& \Psi_{1}^{0}(\xi_{1})\Psi_{1}^{0}(\xi_{2})
%    \vect{x}_{1}+\Psi_{2}^{0}(\xi_{1})\Psi_{1}^{0}(\xi_{2})\vect{x}_{2} + 
%    \nonumber \\
%    & & \Psi_{1}^{0}(\xi_{1})\Psi_{2}^{0}(\xi_{2})\vect{x}_{3} +
%    \Psi_{2}^{0}(\xi_{1})\Psi_{2}^{0}(\xi_{2})\vect{x}_{4} + \nonumber \\
%    & & \Psi_{1}^{1}(\xi_{1})\Psi_{1}^{0}(\xi_{2})\delbyat{\vect{x}}
%    {\xi_{1}}{1}+\Psi_{2}^{1}(\xi_{1})\Psi_{1}^{0}(\xi_{2})\delbyat{\vect{x}}
%    {\xi_{1}}{2} + \nonumber \\
%    & & \Psi_{1}^{1}(\xi_{1})\Psi_{2}^{0}(\xi_{2})\delbyat{\vect{x}}
%    {\xi_{1}}{3}+\Psi_{2}^{1}(\xi_{1})\Psi_{2}^{0}(\xi_{2})\delbyat{\vect{x}}
%    {\xi_{1}}{4} + \nonumber \\
%    & & \Psi_{1}^{0}(\xi_{1})\Psi_{1}^{1}(\xi_{2})\delbyat{\vect{x}}
%    {\xi_{2}}{1}+\Psi_{2}^{0}(\xi_{1})\Psi_{1}^{1}(\xi_{2})\delbyat{\vect{x}}
%    {\xi_{2}}{2} + \nonumber \\
%    & & \Psi_{1}^{0}(\xi_{1})\Psi_{2}^{1}(\xi_{2})\delbyat{\vect{x}}
%    {\xi_{2}}{3}+\Psi_{2}^{0}(\xi_{1})\Psi_{2}^{1}(\xi_{2})\delbyat{\vect{x}}
%    {\xi_{2}}{4} + \nonumber \\
%    & & \Psi_{1}^{1}(\xi_{1})\Psi_{1}^{1}(\xi_{2})\deltwobyat{\vect{x}}
%    {\xi_{1}}{\xi_{2}}{1} +
%    \Psi_{2}^{1}(\xi_{1})\Psi_{1}^{1}(\xi_{2})\deltwobyat{\vect{x}}
%    {\xi_{1}}{\xi_{2}}{2} + \nonumber \\
%    & & \Psi_{1}^{1}(\xi_{1})\Psi_{2}^{1}(\xi_{2})\deltwobyat{\vect{x}}
%    {\xi_{1}}{\xi_{2}}{3} +
%    \Psi_{2}^{1}(\xi_{1})\Psi_{2}^{1}(\xi_{2})\deltwobyat{\vect{x}}
%    {\xi_{1}}{\xi_{2}}{4}
    \label{eqn:bicubHerminterp}
\end{eqnarray}

As with the one-dimensional cubic Hermite elements, the derivatives with
respect to $\xi$ in the two-dimensional interpolation formula above are
expressed as the product of a nodal arc-length derivative and a nodal scale
factor. This is, however, complicated by the fact that there are now scale
factors for each $\xi$ direction at each node. For node $n$ we
have
\begin{equation}
%  \delbyat{\vect{x}}{\xi_{i}}{n}=\brdelby{\vect{x}}{s_{i}}_{\Delta(n,e)}.
%  \left(S_{i}\right)_{e}
  \label{eqn:xitoarclength2}
\end{equation}
and for the cross-derivative
\begin{equation}
%  \deltwobyat{\vect{x}}{\xi_{1}}{\xi_{2}}{n}=\brdeltwoby{\vect{x}}{s_{1}}{s_{2}}_
%  {\Delta(n,e)}.\left(S_{1}\right)_{e}.\left(S_{2}\right)_{e}
  \label{eqn:xitoarclength3}
\end{equation}

As with the one-dimensional cubic Hermite case conditions must be placed on
this transformation in order to maintain $C^{1}$ continuity. A sufficient
condition is that the scale factor at a node in one element must be the same
as the scale factor at the same node in an adjacent element. That is, the
elemental scale factor should be nodally based so that the same scale factor
is used at a given node regardless of the current element. With this condition
satisfied any choice of scale factor will give $C^{1}$ continuity across
element boundaries. The choice of the scale factor will, however, affect the
spacing of $\xi$ with arc-length. It is often computationally desirable to
have a uniform spacing of $\xi$ with respect to arc-length (for example, not
biasing the Gaussian quadrature scheme to one end of the element). To achieve
this uniform spacing a good choice of the nodal scale factor (denoted from now
on as ${\cal S}_{n}$ for node $n$) is the average of the two arc-lengths on
either side of the node. This condition must also be applied together with
arc-length derivatives for each $\xi$ direction as defined by
\eqref{eqn:cubHermnormconst}.

If $n_{\ominus}$ is the node or line immediately before node $n$ (in the sense
of increasing $\xi$) and $n_{\oplus}$ is the node or line immediately after
node $n$, the nodal scale factor is given by
\begin{equation}
  {\cal S}_{n}=\frac{s_{n_{\ominus}}+s_{n_{\oplus}}}{2}
  \label{eqn:avearclenscale}
\end{equation}
This is termed average arc-length scaling.  For example consider node 2 in
\figref{fig:cubHermelem} then
\[{\cal S}_{2}=\frac{s_{1}+s_{2}}{2}\]
Thus, for an element $e$, the one-dimensional cubic Hermite interpolation
formula in \bref{eqn:cubHermxiinterp} becomes
\begin{equation}
%  \fnof{\vect{x}}{\xi}=\sum_{n=1,2}\left[\Psi_{n}^{0}(\xi)\vect{x}_{n}+\Psi_{n}^{1}(\xi)
%  \brdby{\vect{x}}{s}_{n}.{\cal S}_{n}\right]
  \label{eqn:cubHerminterp}
\end{equation}
To calculate the arc-length for a particular element an iterative process is
needed. The arc-length for an one-dimensional element is defined as
\begin{equation}
%  \mbox{arc-length}=\int_{0}^{1}\norm{\dby{\vect{x}(\xi)}{\xi}}d\xi=
%  \int_{0}^{1}\sqrt{\brdby{x(\xi)}{\xi}^{2}+\brdby{y(\xi)}{\xi}^{2}}d\xi
  \label{eqn:arclendef}
\end{equation}
However, since the interpolation of \fnof{\vect{x}}{\xi}, as defined in
\bref{eqn:cubHerminterp}, uses the arc-length in the calculation of the
scaling factor, an iterative root finding technique is needed to obtain the
arc-length.

With nodal based scale factors and arc-length derivatives we achieve a good
arc-length to $\xi$ spacing, continuity is maintained with respect to a
physically meaningful parameter and the numerical results when using the mesh
are more accurate.

\subsection{Least squares fitting with cubic Hermite interpolation}

\subsubsection{Problem formulation}

Consider a set of rectangular cartesian data with geometric positions
$\vect{z}_{d},d=1..D$. For each data point we can find the position on the mesh
which has the smallest distance to that data point. This point is the
orthogonal projection of the data point onto the mesh and has geometric
position $\vect{z}$. The point $\vect{z}$ is also given by the local element
co-ordinate $\vect{\xi}_{d}$ as is shown in \figref{fig:dataproj}.

To calculate $\vect{\xi}_{d}$ a non-linear iterative procedure is required.
Given a $\vect{\xi}$ position for the data point projection within an element
the geometric position of this projection is given by the standard
interpolation formula (\eqref{eqn:cubHermxiinterp} or
\eqref{eqn:bicubHerminterp}). An error function can then be set up as the
Euclidean distance between this position and the actual position of the data
point. The $\vect{\xi}$ position which minimises this function can the be found
by using the Newton-Rhapson root finding method on the derivative of this
function. This $\vect{\xi}$ position is the orthogonal projection of the data
point.

\begin{figure} \centering
  \input{figs/datafitting/datapos.pstex}
  \caption{Definition of a data point projection into an element. The data
    point at geometric location $\vect{z}_{d}$ is projected into an element at a
    geometric position $\vect{z}$ and element co-ordinate $\xi_{d}$.}
  \label{fig:dataproj}
\end{figure}

For simplicity only two-dimensional fitting (i.e. cubic Hermite elements) will
be covered in this section. Hence, given $\xi_{d}$, $\vect{z}$ can be found from
interpolation i.e.
\begin{equation}
%  \vf{z}{\xi_{d}}=\sum_{n=1,2}\left[\onevf{\Psi_{n}^{0}}{\xi_{d}}\vect{x}_{n}+
%  \onevf{\Psi_{n}^{1}}{\xi_{d}}\brdby{\vect{x}}{s}_{n}.{\cal S}_{n}\right]
  \label{eqn:datapointpos}
\end{equation}

The measure of error for each data point is defined as the Euclidean distance
between the data point and its closest projection onto the current mesh:
\begin{equation}
%  \onevf{f_{d}}{\xi_{d}}=\norm{\vf{z}{\xi_{d}}-\vect{z}_{d}}
\end{equation} 

For a given projection of the data points onto the mesh (i.e. $\xi_{d}$ is
held constant) the objective function to be minimised in the fit is then formed
as the sum-of-squares of the individual errors.
\begin{equation}
%  \vvf{{\cal F}}{u}=\sum_{d=1,D}\gamma_{d}\onevf{f_{d}^{2}}{\xi_{d}}=
%  \sum_{d=1,D}\gamma_{d}\norm{\onevf{\vect{z}}{\xi_{d}}-\vect{z}_{d}}^{2}
  \label{eqn:linfitobjfun}
\end{equation}
where $\gamma_{d}$ is a weight for each data point and \vect{u} is a vector of
mesh parameters.

The fitting problem is to find the set of mesh parameters that minimises 
this objective function. Substituting \bref{eqn:datapointpos} into
\bref{eqn:linfitobjfun} and differentiating we obtain
\begin{equation}
%  \delby{\vvf{{\cal F}}{u}}{\left(x_{j}\right)_{m}}=2\sum_{d=1,D}\gamma_{d}
%  \left(\sum_{n=1,2}\left[\onevf{\Psi_{n}^{0}}{\xi_{d}}\left(x_{j}\right)_{n}+
%  \onevf{\Psi_{n}^{1}}{\xi_{d}}\brdelby{x_{j}}{s}_{n}.{\cal S}_{n}\right]-
%  {z_{d}}_{j}\right)\onevf{\Psi_{m}^{0}}{\xi_{d}} 
  \label{eqn:lindatajacnode}
\end{equation}
\begin{equation}
%  \delby{\vvf{{\cal F}}{u}}{\brdelby{x_{j}}{s}_{m}}=2\sum_{d=1,D}\gamma_{d}
%  \left(\sum_{n=1,2}\left[\onevf{\Psi_{n}^{0}}{\xi_{d}}\left(x_{j}\right)_{n}+
%  \onevf{\Psi_{n}^{1}}{\xi_{d}}\brdelby{x_{j}}{s}_{n}.{\cal S}_{n}\right]-
%  {z_{d}}_{j}\right)\onevf{\Psi_{m}^{1}}{\xi_{d}}{\cal S}_{m} 
  \label{eqn:lindatajacderiv}
\end{equation}

A minimum can thus be found to the objective function by setting both 
\eqnrefs{eqn:lindatajacnode}{eqn:lindatajacderiv} to zero. This will result in
a linear system only if the scale factors are kept constant during the fit. 
That is the vector $\vect{u}$ will contain the nodal positions and the nodal
arc-length derivatives. With this restriction we can obtain
\begin{eqnarray*}
%  \sum_{d=1,D}\gamma_{d}\left(\sum_{n=1,2}\left[\onevf{\Psi_{n}^{0}}{\xi_{d}}
%  \left(x_{j}\right)_{n}+\onevf{\Psi_{n}^{1}}{\xi_{d}}{\cal S}_{n}
%  \brdelby{x_{j}}{s}_{n}\right]\onevf{\Psi_{m}^{0}}{\xi_{d}}\right)
%  &=&\sum_{d=1,D}\gamma_{d}\onevf{\Psi_{m}^{0}}{\xi_{d}}{z_{d}}_{j} \\ 
%  \sum_{d=1,D}\gamma_{d}\left(\sum_{n=1,2}\left[\onevf{\Psi_{n}^{0}}{\xi_{d}}
%  \left(x_{j}\right)_{n}+\onevf{\Psi_{n}^{1}}{\xi_{d}}{\cal S}_{n}
%  \brdelby{x_{j}}{s}_{n}\right]\onevf{\Psi_{m}^{1}}{\xi_{d}}{\cal S}_{m}\right)
%  &=&\sum_{d=1,D}\gamma_{d}\onevf{\Psi_{m}^{1}}{\xi_{d}}{\cal S}_{m}{z_{d}}_{j}
\end{eqnarray*}
This is a linear system of equations for the element of the form
\begin{equation}
  K_{mn}u_{n}=f_{m}
  \label{eqn:linsystem}
\end{equation}

A linear system of equations governing the entire mesh can then be found by
assembling a global stiffness matrix from all the individual element matrices.
This can then be solved to yield the nodal positions and derivatives which
minimises the error in the mesh. An example of fitting is shown in
\figref{fig:cubHermfit}.

\begin{figure} \centering
  \input{figs/datafitting/fitting.pstex}
  \caption{Geometric data fitting with cubic Hermite elements. (a) The data
     points (+) are shown projected onto the two element mesh at some 
     intermediate stage in the fitting procedure. (b) The final fitted mesh.}
  \label{fig:cubHermfit}
\end{figure}

\subsubsection{Sobelov Smoothing}

With an insufficient number of data points, fitting 'noisy' data or fitting
data that has an uneven spread, a smoothness constraint \cite{young:1989}
can be introduced by adding a second term to the objective function:
%\[\vvf{F}{u}=\sum_{d=1,D}\gamma_{d}\norm{\onevf{\vect{z}}{\vect{\xi}_{d}}-
%\vect{z}_{d}}^{2}+\int_{\Omega} \onevf{g}{\vvvf{u}{\xi}}d\vect{\xi}\]

The first term measures the error in the surface from the data and the second
term measures deformation of the surface. To measure the deformation of the
surface a $p^{th}$ order weighted Sobelov norm 
\cite{terzopoulos:1986,tikhonov:1977} is used, defined by
\begin{equation}
%  \onevf{g_{p,w}}{\vvvf{u}{\xi}}=\sum_{q=0}^{p}\sum_{i+j=q}w_{ij}
%  \norm{\frac{\del^{q}\vect{u}}{\del^{i}\xi_{1}\del^{j}\xi_{2}}}^{2}
  \label{eqn:Sobnorm}
\end{equation}
where $w_{ij}$ are the weights for the norm. The addition to the objective
function, called the Sobelov value, is defined as
\begin{equation}
%  \onevf{G}{\vvvf{u}{\xi}}=\int_{\Omega}{\onevf{g}{\vvvf{u}{\xi}}d
%    \vect{\xi}}
  \label{eqn:Sobvalue}
\end{equation}
where $\Omega$ is the mesh domain.

For the case of two-dimensional fitting (i.e. cubic Hermite elements) $j=0$
(as there is no $\xi_{2}$ direction) and for the case of three-dimensional
fitting (i.e. bicubic Hermite elements) $j=0..2$. Consider the case for $p=2$,
and for 2D: $w_{0}=0, w_{1}=\alpha, w_{2}=\beta$, and for 3D: $w_{00}=0,
w_{01}=w_{10}=\alpha, w_{20}=w_{02}= \beta, w_{11}=2\beta$. The Sobelov value
now becomes
\begin{equation}
  \begin{array}{lllll}
%    &\onevf{G}{\vvvf{u}{\xi}}&=&\displaystyle\int_{\Omega}{\left\{
%    \alpha\norm{\dby{\vect{u}}{\xi}}^{2}+\beta\norm{\dtwosqby{\vect{u}}{\xi}}^{2}
%    \right\}d\xi} & \mbox{for 2D} \nonumber \\ 
%    \mbox{or}&\onevf{G}{\vvvf{u}{\xi}}&=&\displaystyle\int_{\Omega}
%    \left\{\alpha\left(\norm{\delby{\vect{u}}{\xi_{1}}}^{2}+\norm{\delby{\vect{u}}
%    {\xi_{2}}}^{2}\right)\right.+ & \nonumber \\ 
%    &&&\left.\beta\left(\norm{\deltwosqby{\vect{u}}{\xi_{1}}}^{2}+
%    2\norm{\deltwoby{\vect{u}}{\xi_{1}}{\xi_{2}}}^{2}+
%    \norm{\deltwosqby{\vect{u}}{\xi_{2}}}^{2}\right)\right\}d\vect{\xi}
%    & \mbox{for 3D} \nonumber
  \end{array}
  \label{eqn:ptwoSobnorm}
\end{equation}

The parameter $\alpha$ controls the tension of the surface and the parameter
$\beta$ controls the degree of surface curvature
\cite{terzopoulos:1986}.

\subsection{Non-linear geometric fitting with Hermite elements}

\subsubsection{Problem formulation}

One problem that arises when using linear fitting with cubic Hermite elements
is that arc-length derivatives and average arc-length scaling are not
maintained during the fit. In linear fitting there is no information supplied
in the linear fitting model that enforces arc-length derivatives and the scale
factors are held constant during the fit. Here we consider how to fit the data
whilst maintaining arc-length derivatives and an even spacing of arc-length
with $\xi$. Because both the value of the arc-length for the element and the
relationship between the derivatives in the various spatial directions depend
upon the mesh parameters in a non-linear fashion, the only way to ensure
arc-length derivatives are maintained during fitting is to use a non-linear
fitting procedure.

Consider the following: Let \vect{u} be the vector of mesh parameters, 
$\vect{z}_{d}$ the vector of the location of the data points in space and 
%\twovf{\vect{z}}{\vect{u}}{\vect{\xi}_{d}} 
the vector of the location of the closest projection (at the points given by
%the vector $\vect{\xi}_{d}$) of data point $d$ onto the mesh. Now the error in
the $d^{th}$ data point can be expressed by an error vector
\begin{equation}
%\twovf{\vect{e}_{d}}{\vect{u}}{\vect{\xi}_{d}}=\twovf{\vect{z}}{\vect{u}}{\vect{\xi}_{d}}
%-\vect{z}_{d}
\label{eqn:errorvec}
\end{equation}
and by a residual,
\begin{equation}
%\twovf{f_{d}}{\vect{u}}{\vect{\xi}_{d}}=\norm{\twovf{\vect{e}_{d}}{\vect{u}}
%{\vect{\xi}_{d}}}^{2}
\label{eqn:dataresidvec}
\end{equation}

%An objective function, \twovf{{\cal F}}{\vect{u}}{\vect{\xi}_{d}}, is formed 
as the sum of squares of the individual residuals for the mesh \vect{u} and
projections $\vect{\xi}_{d}$
\begin{equation}
%\twovf{{\cal F}}{\vect{u}}{\vect{\xi}_{d}}= \twovft{\vect{f}}{\vect{u}}{\vect{\xi}_{d}}
%\twovf{\vect{f}}{\vect{u}}{\vect{\xi}_{d}}
\label{eqn:dataresfun}
\end{equation}
%The fitting problem then becomes, for constant $\vect{\xi}_{d}$,
\begin{eqnarray*}
%\min_{\vect{u} \in \Re} & \vvf{{\cal F}}{u} = \twovft{\vect{f}}{\vect{u}}
%{\vect{\xi}_{d}}\twovf{\vect{f}}{\vect{u}}{\vect{\xi}_{d}}
\end{eqnarray*}

In order to maintain arc-length derivatives a non-linear constraint is needed.
This constraint comes from the geometric properties of arc-length derivatives.
For a node $n$ and $\xi$-direction $i$ the magnitude of the vector of
arc-length derivatives in the various spatial directions must be 1 as detailed
in \eqref{eqn:cubHermnormconst}.  Hence the constraint is
\begin{equation}
%  c=\norm{\brdelby{\vect{x}}{s_{i}}_{n}}=1
  \label{eqn:normconst}
\end{equation}
This also implies simple bounds on the derivative variables:
%\[-1 \leq \brdelby{x_{j}}{s_{i}}_{n} \leq 1\]

Thus the fitting problem can be written in terms of a non-linearly constrained,
non-linear optimisation problem
\begin{eqnarray}
%  \min_{\vect{u} \in \Re,~\vect{\xi}_{d} \mbox{const}} & \vvf{{\cal F}}{u} = 
%  \twovft{\vect{f}}{\vect{u}}{\vect{\xi}_{d}}\twovf{\vect{f}}{\vect{u}}{\vect{\xi}_{d}} 
  \label{eqn:nloproblem} \\
%  \mbox{subject to} & \vect{a} \leq \left(\begin{array}{c}
%  \vect{u} \nonumber \\ \vvvf{c}{u} \end{array} \right) \leq \vect{b} \nonumber
\end{eqnarray}
where \vect{a} is a vector of lower bounds, \vect{b} a vector of upper bounds 
%and \vvvf{c}{u} a vector of non-linear constraints. This type of optimisation 
problem can be solved with readily available non-linear optimisation packages 
such as NPSOL \cite{gill:1986}. 

In order to ensure that there is an approximately uniform spacing of $\xi$
with arc-length two approaches can be used. The first approach is to use
an iterative technique for the scale factors and is detailed here. The
second approach is to include the scale factors in the optimisation problem
and is detailed in the next section.

The constraint on the nodal scale factors (being the average of the line
lengths either side of the node) is placed upon the problem to ensure that the
arc-length to $\xi$ spacing is approximately uniform. As we are only getting
approximately uniform arc-length to $\xi$ spacing we can relax the constraint
on nodal scale factors. If the nodal scale factors are held fixed during the
fit we will not have average arc-length scaling throughout the fitting process
but we will have a reasonable approximation. With the nodal scale factors
fixed the variables ${\cal S}_{n}$ can be removed from the vector of mesh
%parameters \vect{u}.

The approach is to hold the scale factors constant, fit the mesh with
these scale factors, and then update the scale factors (based on the new mesh)
to be average arc-length. This process can be repeated iteratively until the
desired fit has be achieved. It should also be noted that
\eqref{eqn:nloproblem} is defined only for a constant data point projection.
With this iterative approach the data point projections are also updated
at the same time as the scale factors.

The convergence of the fitting problem can be measured in two ways. The first
is the convergence of the RMS error in the data and second is the convergence
in the magnitude of the nodal scale factors.

The algorithm is therefore:
\begin{enumerate}
  \item Define initial mesh (and calculate initial nodal scale factors)
  \item Calculate the initial data point projections
  \item Repeat until converged or the maximum number of iterations is exceeded
  \begin{enumerate}
    \item Fit the mesh to the data by solving \bref{eqn:nloproblem}
    \item Update the scale factors to be average arc-lengths based on the new
      mesh
    \item Recalculate the data point projections on the new mesh
  \end{enumerate}
\end{enumerate}

\subsubsection{Alternative formulation}
\label{sec:alternativeformulation}

An alternative approach to ensuring average arc-length scale factors can be
formulated by including the scale factors in the optimisation problem as
optimisation variables.  The vector of mesh parameters, \vect{u}, is hence
extended to include the nodal scale factors. With this a new constraint
can be introduced to \eqref{eqn:nloproblem} to ensure that there is an
approximately uniform spacing of $\xi$ with arc-length. This can be achieved
if the nodal scale factor is the average on the arc-lengths on either side of
that node. If $\left({\cal S}_{i}\right)_{n}$ is the average arc-length for
node $n$ in the $\xi_{i}$ direction as given by \bref{eqn:avearclenscale} then
the constraint that the nodal scale factor equals the average arc-length is
given by
\begin{equation}
%  c=\frac{\left(s_{i}\right)_{n\ominus}+\left(s_{i}\right)_{n\oplus}}{2}-
%  \left({\cal S}_{i}\right)_{n}=0
\end{equation}
or
\begin{equation} 
%  c=\frac{1}{2}\left(\int_{0}^{1}\norm{\delby{\vvvf{x}{\xi}}{\xi_{i}}}_
%  {n_{\ominus}}d\xi_{i}+\int_{0}^{1}\norm{\delby{\vvvf{x}{\xi}}{\xi_{i}}}_
%  {n_{\oplus}}d\xi_{i}
%  \right)-\left({\cal S}_{i}\right)_{n}=0
  \label{eqn:arclenconst}
\end{equation}
Note that no summation over $\xi_{i}$ is implied.

This also generates a simple bound on the scale-factors:
%\[\left({\cal S}_{i}\right)_{n} > 0\]

This formulation of the non-linear fitting problem does have one practical
limitation. With average arc-length scaling the interpolation within an
element depends on the arc-length on the neighbouring elements, and the
arc-length of the neighbouring elements depends on their neighbouring elements
and so on.  This results in a 'global mesh' in that every part of the mesh
is dependent on every other part of the mesh. The implication of this is that
the entire mesh must be included in the fit otherwise average arc-length
scaling cannot be achieved. This is not a desirable feature for very large
problems or for problems where only a small part of the mesh is in error and
needs to be fitted. The formulation still requires iteration for the data
point projections but does have the advantage that the number of iterations 
required is reduced as the scale factors are found during the fit. This is 
at the expense of having to solve a much larger non-linear optimisation
problem with more variables and, more importantly, more non-linear constraints.

The fit can be considered converged when either the data point projections or
the RMS error in the fit does not change significantly between iterations.
The algorithm for the non-linear data fitting procedure is as follows:

\begin{enumerate}
  \item Define the initial mesh (and calculate the initial nodal scale factors)
  \item Calculate the initial data point projections
  \item Repeat until converged or the maximum number of iterations is exceeded
  \begin{enumerate}
    \item Fit the mesh to the data by solving \bref{eqn:nloproblem} (with
      the new constraints)
    \item Recalculate the data point projections on the new mesh
  \end{enumerate}
\end{enumerate}

\subsubsection{Residual and constraint Jacobians}
\label{sec:resandcontjacs}

Solution of the non-linear problem given by \bref{eqn:nloproblem} will
generally require the evaluation of objective gradient (or residual vector
Jacobian) and the constraint Jacobian with respect to the optimisation (mesh)
parameters. For simplicity in this section we will be concerned with
two-dimensions only (i.e.  one-dimensional cubic Hermite elements). The
residual and constraint Jacobians will be given for the alternative
formulation as this covers all cases. If the scale factors are found by
iteration (i.e. the problem as described in \secref{sec:probform}) the
Jacobians with respect to the nodal scale factors can be ignored, as can the
second constraint (\eqref{eqn:arclenconst}) which ensures average arc-length
scale factors.  In this case there are three basic types of variable within an
element: the nodal variables $\left(x_{j} \right)_{n}$, the derivative
%variables $\brdelby{x_{j}}{s}_{n}$ and the nodal scale factors ${\cal S}_{n}$.
The residual and constraint Jacobians are given below for each of these three
variable types.

%Consider the error vector for the data point $d$, $\vect{e}_{d}$, defined in
\bref{eqn:errorvec}, and its corresponding residual $f_{d}$, defined in
\bref{eqn:dataresidvec}. The position of the projection of data point $d$
within the element is given by \bref{eqn:datapointpos}.  Substituting
\bref{eqn:datapointpos} into \eqnrefs{eqn:errorvec}{eqn:dataresidvec} and
differentiating with respect to the various optimisation variables we can
obtain the Jacobian of the residual vector:
\begin{eqnarray}
%  \delby{f_{d}}{\left(x_{j}\right)_{n}}&=&2\Psi^{0}_{n}(
%  \xi_{d})~{e_{d}}_{j} \label{eqn:dataresjacnode} \\
%  \delby{f_{d}}{\brdelby{x_{j}}{s}_{n}}&=&2\Psi^{1}_{n}(\xi_{d})~{\cal S}_{n}~
%  {e_{d}}_{j} \label{dataresjacderiv} \\
%  \delby{f_{d}}{{\cal S}_{n}}&=&2\Psi^{1}_{n}(\xi_{d})~\brdelby{\vect{x}}
%  {s}_{n}\cdot\vect{e}_{d} \label{eqn:dataresjacline}
\end{eqnarray}

Note that if the data residual was
%\[\twovf{f_{d}}{\vect{u}}{\vect{\xi}_{d}}=\norm{\twovf{\vect{e}_{d}}{\vect{u}}
%{\vect{\xi}_{d}}}\]
\eqref{eqn:dataresjacnode} would become
%\[\delby{f_{d}}{\left(x_{j}\right)_{n}}=\frac{\Psi^{0}_{n}(\xi_{d})~{e_{d}}_{j}}
%{f_{d}}\] which is singular at the optimal solution $f_{d}=0$. To avoid
numerical problems the residual is therefore defined by
\bref{eqn:dataresidvec}.

Now consider the constraint Jacobians. Differentiating constraint
\bref{eqn:normconst} with respect to the mesh parameters gives constraint
gradients:
\begin{eqnarray}
%  \delby{c}{\left(x_{j}\right)_{n}}&=&0 \label{eqn:normjacnode} \\
%  \delby{c}{\brdelby{x_{j}}{s}_{n}}&=&\frac{\brdelby{x_{j}}{s}_{n}}{\norm{
%  \brdelby{\vect{x}}{s}_{n}}}\label{eqn:normjacderiv} \\
%  \delby{c}{{\cal S}_{n}}&=&0 \label{eqn:normjacline}
\end{eqnarray}

To calculate the gradients for constraint \bref{eqn:arclenconst} we first
compute the rate of change of \vect{x} with respect to $\xi$ within an element
by differentiating \bref{eqn:cubHerminterp}:
\begin{equation}
%  \delby{\fnof{\vect{x}}{\xi}}{\xi}=\sum_{n=1,2}\left[\delby{\onevf{\Psi_{n}^{0}}{\xi}}
%  {\xi}\vect{x}_{n}+\delby{\onevf{\Psi_{n}^{1}}{\xi}}{\xi}\brdelby{\vect{x}}
%  {s}_{n}.{\cal S}_{n}\right]
  \label{eqn:delxdelxiinterp}
\end{equation}

Substituting \bref{eqn:delxdelxiinterp} into
\bref{eqn:arclenconst} and differentiating with respect to the optimisation
variables gives the constraint gradients:
\begin{eqnarray}
%  \delby{c}{\left(x_{j}\right)_{n}}&=&\half\left({\int_{0}^{1}\frac{\left(
%  \delby{\onevf{\Psi_{2}^{0}}{\xi}}{\xi}\delby{\onevf{x_{j}}{\xi}}{\xi}
%  \right)_{n_{\ominus}}}{\norm{\delby{\fnof{\vect{x}}{\xi}}{\xi}}_{n_{\ominus}}}d\xi}+
%  \int_{0}^{1}{\frac{\left(\delby{\onevf{\Psi_{1}^{0}}{\xi}}{\xi}\delby{
%  \onevf{x_{j}}{\xi}}{\xi}\right)_{n_{\oplus}}}{\norm{\delby{\fnof{\vect{x}}{\xi}}
%  {\xi}}_{n_{\oplus}}}d\xi}\right) \label{eqn:arclenjacnode} \\
%  \delby{c}{\brdelby{x_{j}}{s}_{n}}&=&\half\left({\int_{0}^{1}\frac{\left(
%  \delby{\onevf{\Psi_{2}^{1}}{\xi}}{\xi}{\cal S}_{n}\delby{\onevf{x_{j}}
%  {\xi}}{\xi}\right)_{n_{\ominus}}}{\norm{\delby{\fnof{\vect{x}}{\xi}}{\xi}}_{n_{
%  \ominus}}}d\xi}+\int_{0}^{1}{\frac{\left(\delby{\onevf{\Psi_{1}^{1}}{\xi}}
%  {\xi}{\cal S}_{n}\delby{\onevf{x_{j}}{\xi}}{\xi}\right)_{n_{\oplus}}}
%  {\norm{\delby{\fnof{\vect{x}}{\xi}}{\xi}}_{n_{\oplus}}}d\xi}\right) 
%  \label{eqn:arclenjacderiv} \\
%  \delby{c}{{\cal S}_{n}}&=&\half\left(\int_{0}^{1}{\frac{\left(\delby{
%  \onevf{\Psi_{2}^{1}}{\xi}}{\xi}\brdelby{\vect{x}}{s}_{n}\cdot\delby{
%  \fnof{\vect{x}}{\xi}}{\xi}\right)_{n_{\ominus}}}{\norm{\delby{\fnof{\vect{x}}{\xi}}
%  {\xi}}_{n_{\ominus}}}d\xi}\right. + \nonumber \\
%  &&\hspace{3cm}\left.\int_{0}^{1}{\frac{\left(\delby{\onevf{\Psi_{1}^{1}}
%  {\xi}}{\xi}\brdelby{\vect{x}}{s}_{n}\cdot\delby{\fnof{\vect{x}}{\xi}}{\xi}\right)_{
%  n_{\oplus}}}{\norm{\delby{\fnof{\vect{x}}{\xi}}{\xi}}_{n_{\oplus}}}d\xi}\right) 
%  - 1 
   \label{eqn:arclenjacline}
\end{eqnarray}

Note that when the integrand in these formula contains $n_{\ominus}$
($n_{\oplus}$) the $\xi$ variable is assumed to be taken over the previous
(next) element.

Similar results can be found for three-dimensions (i.e. bicubic Hermite
elements).

\subsubsection{Sobelov smoothing}

In order to implement Sobelov smoothing in the form of the non-linear
optimisation problem given in \bref{eqn:nloproblem} we need to modify the
residual vector by defining an additional residual as the Sobelov value for
the mesh being fitted, that is
\begin{equation}
%  \twovf{f_{D+1}}{\vect{u}}{\vect{\xi}}=\onevf{G}{\vvvf{u}{\xi}}
  \label{eqn:addSobresidual}
\end{equation}
%where \onevf{G}{\vvvf{u}{\xi}} is defined in \bref{eqn:ptwoSobnorm}.
The objective function to minimise now becomes
%\[\vvf{F}{u}=\vvf{{\cal F}}{u}+\vvf{G^{2}}{u}=\vvf{{\cal F}}{u}+
%\vvf{{\cal G}}{u}\]
%where \vvf{{\cal F}}{u} is the data objective defined in \bref{eqn:dataresfun}
%and $\vvf{{\cal G}}{u}=\vvf{G^{2}}{u}$ is the Sobelov objective.

The gradients of this additional residual can then be found by differentiating
\bref{eqn:ptwoSobnorm} with respect to the mesh parameters. Considering the
case of two-dimensional fitting and breaking the domain up into elements we
obtain
%\[\vvf{G}{u}=\sum_{e}\int_{0}^{1}\left\{\alpha\normdelby{\fnof{\vect{x}}{\xi}}
%{\xi}^{2}+\beta\norm{\frac{\del^{2}\fnof{\vect{x}}{\xi}}{\del \xi^{2}}}^{2}
%\right\}d\xi\]
Interpolating within the element using cubic Hermite elements and 
differentiating with respect to the three different mesh parameters within
an element we obtain the additional residual gradients given in equations 
(\ref{eqn:sobjacnode}--\ref{eqn:sobjacline}).
\begin{eqnarray}
%  \delby{f_{D+1}}{\left(x_{j}\right)_{n}}&=&2\int_{0}^{1}\left\{\alpha
%  \delby{\onevf{\Psi_{n}^{0}}{\xi}}{\xi}\delby{\onevf{x_{j}}{\xi}}{\xi}+
%  \beta\frac{\del^{2}\onevf{\Psi_{n}^{0}}{\xi}}{\del\xi^{2}}\frac{\del^{2}
%  \onevf{x_{j}}{\xi}}{\del\xi^{2}}\right\}d\xi \label{eqn:sobjacnode} \\
%  \delby{f_{D+1}}{\brdelby{x_{j}}{s}_{n}}&=&2\int_{0}^{1}\left\{\alpha
%  \delby{\onevf{\Psi_{n}^{1}}{\xi}}{\xi}{\cal S}_{n}\delby{\onevf{x_{j}}
%  {\xi}}{\xi}+\beta\frac{\del^{2}\onevf{\Psi_{n}^{1}}{\xi}}{\del\xi^{2}}
%  {\cal S}_{n}\frac{\del^{2}\onevf{x_{j}}{\xi}}{\del\xi^{2}}\right\}d\xi 
%  \label{eqn:sobjacderiv} \\
%  \delby{f_{D+1}}{{\cal S}_{n}}&=&2\int_{0}^{1}\left\{\alpha\delby{
%  \onevf{\Psi_{n}^{1}}{\xi}}{\xi}\brdelby{\vect{x}}{s}_{n}\cdot\delby{\fnof{\vect{x}}
%  {\xi}}{\xi}+\beta
%  \frac{\del^{2}\onevf{\Psi_{n}^{1}}{\xi}}{\del\xi^{2}}\brdelby{\vect{x}}{s}_{n}
%  \cdot\frac{\del^{2}\fnof{\vect{x}}{\xi}}{\del\xi^{2}}\right\}d\xi \label{eqn:sobjacline}
\end{eqnarray}

Note that all the above integrals are with respect to one element. All the
contributions from each element need to be included to obtain the complete
residual gradients for the optimisation parameters.

%%% Local Variables: 
%%% mode: latex
%%% TeX-master: "/product/cmiss/documents/notes/fembemnotes/fembemnotes"
%%% compile-command: "cd ..; latex '\\nonstopmode\% List of examples package
%
% Chris Bradley
%
% It is only implemented for books at the moment
%

\NeedsTeXFormat{LaTeX2e}
\ProvidesPackage{lofexamples}[1996/10/16]

\newcommand{\listofexamplesname}{List of Examples}

\newcommand{\listofexamples}{ %
  \chapter*{\listofsymbolsname}%
  \addcontentsline{toc}{chapter}{\hbox to\@tempdima{\hfil}\listofexamplesname}%
  \markboth{\listofexamplesname}{\listofexamplesname} %
  \@starttoc{xmp} %
  }
\newcommand{\l@example}[2]{\@dottedtocline{1}{1.5em}{2.3em}{#1}{#2}}

\newcounter{examplecntr}[subsection]

\renewcommand{\theexamplecntr}{\thesubsection.\arabic{examplecntr}}

\newenvironment{example}[2]
  {\refstepcounter{examplecntr}
  \addcontentsline{xmp}{example}{Example \theexamplecntr: #1}
  \vspace{5pt}\par\noindent\rule{\linewidth}{1pt}
  {\bf Example \theexamplecntr}: #2
  \vspace{10pt}\newline}
  {\par\noindent\rule{\linewidth}{1pt}\vspace{5pt}}
'"
%%% End: 
