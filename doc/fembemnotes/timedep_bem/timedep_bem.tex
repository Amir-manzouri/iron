\chapter{The BEM for Parabolic PDES}

\section{Time-Stepping Methods}

Several approaches have been proposed for applying the BEM to parabolic
problems.  These methods can be broadly classified into two main approaches.
Either some form of time-stepping procedure is used to advance the solution
in time, or a semi-analytic technique is used which can directly calculate
a solution at a specified time. In this section time-stepping procedures
will be considered.

Time-stepping approaches discretise the time domain in some manner and use
some form of time marching scheme to advance the solution from one discrete
time to the next.  The two most commonly used time-stepping methods are the
coupled finite difference - BEM and the direct time integration method.
These two methods will be outlined in this section for the diffusion
equation 
\begin{equation}
  \laplacian{\fnof{u}{\vect{x},t}} = \dfrac{1}{\kappa}
  \delby{\fnof{u}{\vect{x},t}}{t}
\end{equation}
where the diffusivity $\kappa$ is a material parameter which can be a
constant or a function of position.

\subsection{Coupled Finite Difference - Boundary Element Method}
\label{sec:fdbem}

\index{Coupled finite difference - boundary element method}
This approach discretises the time-domain in a finite difference form.
Consider the variation between a time $t^{m}$ and a time $t^{m+1} =
t^{m} + \Delta t$.  The most common approach \cite{brebbia:1984} is to
assume that, for sufficiently small $\Delta t$, the time derivative can be
approximated using a first-order fully implicit finite difference scheme 
\begin{equation}
  \delby{\fnof{u}{\vect{x},t^{m+1}}}{t} =
  \dfrac{\fnof{u}{\vect{x},t^{m+1}} - \fnof{u}{\vect{x},t^{m}}}{\Delta t}
\label{eq:fdapprox}
\end{equation}
which allows the diffusion equation in this time-range to be approximated as
\begin{equation}
\laplacian{\fnof{u}{\vect{x},t^{m+1}}} - \dfrac{1}{\kappa \Delta t}
\fnof{u}{\vect{x},t^{m+1}} + \dfrac{1}{\kappa \Delta t} \fnof{u}{\vect{x},t^{m}} = 0
\label{eq:diffdapprox}
\end{equation}
Using this finite difference approximation the original parabolic equation
has been reduced to an elliptic equation.  Using the weighted residuals
method an integral equation can be generated from \eqnref{eq:diffdapprox}. 
\begin{equation}
\fnof{c}{\vect{\xi}}u^{m+1} + \goneint{u^{m+1} \delby{\omega}{n}}{\Gamma} =
\goneint{q^{m+1} \omega d\Gamma + \dfrac{1}{\kappa \Delta t}
\dintl{\Omega}{} u^{m} \omega}{\Omega}
\label{eq:fdbem}
\end{equation}
where $u^{m+1} = \fnof{u}{\vect{x},t^{m+1}}$ and $u^{m} = \fnof{u}{\vect{x},t^{m}}$.
The fundamental solution $\omega$ is a solution of the modified
Helmholtz equation
\begin{equation}
\laplacian{\fnof{\omega}{\vect{x},\vect{\xi}}} - \dfrac{1}{\kappa \Delta t}
\fnof{\omega}{\vect{x},\vect{\xi}} + \fnof{\delta}{\vect{\xi}} = 0 
\end{equation}
applied at some source point $\xi$.  The fundamental solution of the
modified Helmholtz equation is known in both two and three dimensions.  If
an internal solution is required at a specific time this can be determined
explicitly from \eqnref{eq:fdbem} where the fundamental solution is applied
at internal point $\xi$ and $\fnof{c}{\vect{\xi}} = 1$.

Unfortunately \eqnref{eq:fdbem} contains a domain integral.  This integral is
generally evaluated by using a domain mesh \cite{brebbia:1984}.  The domain
integral does not include any problem unknowns so a fairly coarse domain
mesh will generally suffice. Applying the BEM to \eqnref{eq:fdbem} gives
\begin{equation}
\matr{H} \vect{u^{m+1}} - \matr{G} \vect{q^{m+1}} = \matr{B} \vect{u^{m}}
\label{eq:fdbemsys}
\end{equation}
where $\matr{B}$ is a matrix containing the influence coefficients due to the
domain integral.  Using \eqnref{eq:fdbemsys} the solution
can be advanced in time.  $\matr{u^{0}}$ is known from the initial
conditions so a solution can be calculated at $t = t_{0} + \Delta t$.  A
solution at internal nodes can then be calculated.  The time-stepping
procedure can be repeated using the internal solution at  $t = t_{0} +
\Delta t$ as pseudo-initial conditions for the next time-step.  

If a constant time-step is used the matrices $\matr{H}$, $\matr{G}$ and
$\matr{B}$ can be calculated once and stored. The boundary conditions can be
applied to form a solution system of the form $\matr{A} \vect{x^{m+1}} =
\vect{\tau}$ where $\vect{x^{m+1}}$ is the vector of unknown nodal values
at time $t^{m+1}$ and $\vect{\tau}$ is a vector constructed from known
nodal values from the previous time-step.  For a problem with
time-independent boundary conditions at each time-step it is only necessary
to update $\vect{\tau}$ and solve the system for $\vect{x^{m+1}}$.  If a
problem has time-dependent boundary conditions the solution system needs to
be reformed at each time-step.

This coupled finite difference - boundary element method (FD-BEM) was first
proposed by \citeasnoun{walker:1980} for the diffusion equation. It was
implemented and investigated by \citeasnoun{curran:1980}.  They found that
this method will only produce accurate results if \eqnref{eq:fdapprox}
accurately approximates the time derivative.  This will generally require
small time-steps to be adopted.  \citename{curran:1980} investigated the use
of a higher-order approximation to the time-derivative.  They found that
this improved the accuracy of the method.  Unfortunately it lead to a
deterioration in convergence behaviour.

\citeasnoun{tanaka:1994} proposed a generalised version of this time-stepping
scheme.  They approximated the time variation within an interval as
\begin{equation}
\fnof{u}{\vect{x},t} = \phi \fnof{u}{\vect{x},t^{m+1}} + \pbrac{1-\phi}
\fnof{u}{\vect{x},t^{m}}
\end{equation}
where $\phi$, termed the time-scheme parameter, is a constant in the range
$0 < \phi \leq 1$.  Substituting this approximation and a first-order
finite difference approximation of the time derivative into the diffusion
equation gives
\begin{equation}
  \phi \laplacian{u^{m+1}} + \pbrac{1 - \phi} \laplacian{u^{m}} = \dfrac{u^{m+1} -
    u^{m}}{\kappa \Delta t}
\label{eq:tandiffapprox}
\end{equation}
If $\phi = 1$ this approximation of the diffusion equation is
equivalent to the standard FD-BEM discussed earlier.  An integral equation
can be derived from \eqnref{eq:tandiffapprox}.  \citename{tanaka:1994}
implemented this method and found it gave accurate results for a range of
diffusion problems.  They tested the accuracy for a Crank-Nicolson scheme
($\phi = \frac{1}{2}$), a Galerkin scheme ($\phi = \frac{2}{3}$) and a
fully implicit scheme ($\phi = 1$). They found that the best results were
achieved using a Crank-Nicolson scheme. 

\subsection{Direct Time-Integration Method}

\index{Direct time-integration method}
Instead of converting the original parabolic equation to an elliptic
equation the problem can be treated directly in the time domain by directly
integrating over both time and space.  The weighted residual statement
using this approach is
\begin{equation}
  \gint{t_{0}}{t_{F}}{\goneint{\sqbrac{\laplacian{\fnof{u}{\vect{x},t}} -
        \dfrac{1}{\kappa} \delby{\fnof{u}{\vect{x},t}}{t}}
      \fnof{\omega}{\xi,\vect{x},t_{F},t}}{\Omega}}{t} = 0
\end{equation}
Integrating in time once and in space twice gives
\begin{equation}
  \fnof{c}{\xi} \fnof{u}{\xi,t_{F}} + \kappa \gint{t_{0}}{t_{F}}
  {\goneint{u \delby{\omega}{n}}{\Gamma}}{t} = \kappa
  \gint{t_{0}}{t_{F}}{\goneint{q \omega}{\Gamma}}{t} +
  \goneint{\fnof{u}{\vect{x},t_{0}} \omega}{\Omega}
\label{eq:tdfs}
\end{equation}
where the fundamental solution $\omega$ satisfies
\begin{equation}
  \kappa \laplacian{\fnof{\omega}{\xi,\vect{x},t_{F},t}} 
  + \delby{\fnof{\omega}{\xi,\vect{x},t_{F},t}}{t} 
  + \fnof{\delta}{\xi} \fnof{\delta}{t_{F}} = 0
\end{equation}
This time dependent fundamental solution is known in two and three
dimensions.  Physically this fundamental solution represents the effect at
a field point $\vect{x}$ at time $t$ of a unit point source applied at a
point $\xi$ at time $t_{F}$.  If an internal solution is required at a
specific time this can be determined from \eqnref{eq:tdfs} with $\fnof{c}{\xi} = 1$.

The variation of $u$ and $q$ with time is unknown so it is still necessary
to step in time.  However, as the time dependence is included in the
fundamental solution, accurate results can be achieved using larger
time-steps than with the FD-BEM.  Two different time-stepping schemes can
be used.  Similarly to the FD-BEM, each time-step can be treated as a new
problem so that an internal solution is constructed at the end of each
time-step to be used as pseudo-initial conditions for the next time-step.
Alternatively the time integration process can be restarted at $t_{0}$ with
increasing numbers of intermediate steps used.  These two time-stepping
approaches are discussed in detail in \citeasnoun{brebbia:1984}.

The first method requires a new domain integral to be calculated after each
time-step due to the updated pseudo-initial conditions.  The second
time-stepping procedure involves only a domain integral at $t_{0}$ so,
ideally, a domain integral only needs to be calculated once. This, however,
will still require the user to create a domain mesh. As mentioned by
\citeasnoun{brebbia:1984}, in many practical cases the domain integral can
be avoided. If the initial conditions are $u_{0} = 0$ throughout the body
the domain integral equals zero. If the initial conditions satisfy
Laplace's equation $\laplacian{u_{0}} = 0$ then a Galerkin vector can be found
and the domain integral can be expressed as equivalent boundary integrals.
This includes many practical cases such as constant initial temperature or
an initial linear temperature profile.

Unfortunately, in practice it is not always feasible to restart the
integration process at $t_{0}$. At each time-step new $\matr{H}$ and
$\matr{G}$ matrices are required so if many time-steps are required the
storage capacity of the computer is likely to be exceeded.  This requires
the procedure to be restarted at some time where an internal solution is
constructed and used as pseudo-initial conditions to repeat the process.
Therefore, in practice, both time-stepping methods are likely to require
domain integration.

\section{Laplace Transform Method}

\index{Laplace transform method}
An alternative approach which avoids time-stepping is to solve the problem
in a transform domain which removes the time dependence of the problem.
The parabolic PDE is thus converted to an elliptic problem for which the
boundary element method has been shown to generally produce accurate
results.  Once the solution to the elliptic problem is determined in the
transform space a solution in the original space can be attained using an
inverse transform procedure.  The most appropriate transform approach for
parabolic problems is the Laplace transform.

Consider the diffusion equation
\begin{equation}
  \laplacian{\fnof{u}{\vect{x},t}} = \dfrac{1}{\kappa}
  \delby{\fnof{u}{\vect{x},t}}{t}
\label{eq:lap_diff}
\end{equation}
with appropriate boundary and initial conditions.  The Laplace transform
of $\fnof{u}{\vect{x},t}$ will be symbolised as $\fnof{U}{\vect{x},\lambda}$ and is
defined as
\begin{equation}
  \fnof{U}{\vect{x},\lambda} = \gint{0}{\infty}{e^{-\lambda t}
  \fnof{u}{\vect{x},t}}{t}
\end{equation}
Applying Laplace transforms to \eqnref{eq:lap_diff} gives
\begin{equation}
  \laplacian{\fnof{U}{\vect{x},\lambda}} = \dfrac{1}{\kappa} \pbrac{\lambda
    \fnof{U}{\vect{x},\lambda} - \fnof{u_{0}}{\vect{x}}}
\label{eq:laptrans}
\end{equation}
with transformed boundary conditions. $\fnof{u_{0}}{\vect{x}}$ is the initial
conditions of $u$. \eqnref{eq:laptrans} is an elliptic PDE which can be
readily solved using the boundary element method.  Once the solution is
determined in Laplace transform space this solution can be inverted to give
a solution in the time-domain. This inversion procedure requires solutions
to be generated for several values of the transform parameter $\lambda$.

This method was first proposed by \citeasnoun{rizzo:1970} and has since
been successfully used by other practitioners \cite{moridis:1991}
\cite{zhu:1994}.  \citeasnoun{liggett:1979} compared the Laplace transform
method with the time-dependent Green's function method. They noted that the
direct method is simpler to apply.  However, due to its greater efficiency,
they recommended the Laplace transform method for solving diffusion
problems.

One limitation of the Laplace transform method is that \eqnref{eq:laptrans}
is inhomogeneous so that applying the standard BEM will generate a domain
integral involving the initial conditions.  Traditionally this domain
integral has been calculated by using a domain discretisation
\cite{brebbia:1984}.  However, recently \citeasnoun{zhu:1994} proposed
using the DR-BEM to convert this domain integral term to equivalent
boundary integrals.  They chose to apply the DR-BEM based on the known
fundamental solution to the Laplace operator.  Considering
\eqnref{eq:laptrans} this means that the DR-BEM will be used to convert the
right-hand-side to equivalent domain integrals.  Therefore the required
DR-BEM approximation is
\begin{equation}
  \dfrac{1}{\kappa} \pbrac{\lambda \fnof{U}{\vect{x},\lambda} -
    \fnof{u_{0}}{\vect{x}}} = \dsuml{j=1}{N+L} f_{j}\alpha_{j}
\label{eq:drmapprox}
\end{equation}
The DR-BEM can now be applied to \eqnref{eq:laptrans}, giving a matrix
system of the form
\begin{equation}
  \pbrac{\matr{H} - \dfrac{\lambda}{\kappa}\matr{S}} \vect{U} - \matr{G}
  \vect{q} = -\dfrac{1}{\kappa} \matr{S} \vect{u_{0}}
\end{equation}
which can be reduced to a square system by applying boundary conditions.
Once the solution is determined for this elliptic equation in the transform
space a solution at a given time can be constructed using an inversion
process.

This Laplace transform dual reciprocity method (LT-DRM) can easily be
extended to equations of the form
\begin{equation}
  \laplacian{\fnof{u}{\vect{x},t}} = \dfrac{1}{\kappa}
  \delby{\fnof{u}{\vect{x},t}}{t} + \fnof{b}{\vect{x},u}
\end{equation}
in which case a matrix expression of the form 
\begin{equation}
  \pbrac{\matr{H} -\matr{R} - \dfrac{\lambda}{\kappa}\matr{S}} \vect{U}
  - \matr{G} \vect{q} = -\dfrac{1}{\kappa} \matr{S} \vect{u_{0}}
\end{equation}
is generated.  Zhu and his colleagues have successfully extended the LT-DRM
to solve diffusion problems with nonlinear source terms.

\section{The DR-BEM For Transient Problems}
\label{sec:drmtrans}

\index{Dual reciprocity boundary element method@Dual reciprocity
  BEM!transient problems}
The DR-BEM can also be applied to parabolic problems. Consider, 
for example, the diffusion equation
\begin{equation}
\laplacian{u}  = \dfrac{1}{\kappa} \delby{u}{t}
\end{equation}
where the thermal diffusivity, $\kappa$, is a constant.  In this case the
global approximation of $b$ implies a separation of variables such that
\begin{equation}
\delby{u}{t} = \dsuml{j=1}{M} \fnof{f_{j}}{\vect{x}} \fnof{\alpha_{j}}{t}
\label{eq:sovapprox}
\end{equation}
Using \eqnref{eq:sovapprox}, \eqnref{eq:drmsystem} becomes
\begin{equation}
  \matr{H} \vect{u} - \matr{G} \vect{q} = \dfrac{1}{\kappa} \pbrac{\matr{H}
    \matr{\hat{U}} - \matr{G} \matr{\hat{Q}}} \matr{F}^{-1}
  \vect{\delby{u}{t}}
\end{equation}
or
\begin{equation}
  \matr{C} \vect{\delby{u}{t}} + \matr{H} \vect{u} = \matr{G}
  \vect{q}\quad\text{where }\matr{C} = -\dfrac{1}{\kappa}
  \pbrac{\matr{H} \vect{\hat{U}} - \matr{G} \matr{\hat{Q}}} \matr{F}^{-1}
\label{eq:drmtransient}
\end{equation}
\Eqnref{eq:drmtransient} can be solved using a standard direct
time-integration method.

\citeasnoun{partridge:1990} recommended using a first-order finite difference
approximation to the time derivative 
\begin{equation}
\delby{u}{t} = \dfrac{u^{m+1} - u^{m}}{\Delta t}
\end{equation}
and linear approximations to $u$ and $q$ within a time-step.
\begin{align}
u & = \pbrac{1 - \phi_{u}} u^{m} + \phi_{u} u^{m+1} \\
q & = \pbrac{1 - \phi_{q}} q^{m} + \phi_{q} q^{m+1}
\end{align}
where $\phi_{u}$ and $\phi_{q}$ are weighting parameters with values in the
range $\brac{(}{0,1}{]}$ and the time-step is between times $t^{m}$ and $t^{m+1} =
t^{m} + \Delta t$.  Substituting these approximations into
\eqnref{eq:drmtransient} an expression at $t^{m+1}$ can be derived in terms
of values at $t^{m}$.
\begin{equation}
  \sqbrac{\dfrac{1}{\Delta t} \matr{C} + \phi_{u} \matr{H}} \vect{u^{m+1}} -
  \phi_{q} \matr{G} \vect{q^{m+1}} = \sqbrac{\dfrac{\vect{C}}{\Delta t} -
    \pbrac{1-\phi_{u}} \matr{H}} \vect{u^{m}} + \pbrac{1 - \phi_{q}} \matr{G}
  \vect{q^{m}}
\end{equation}
The values of $u^{0}$ and $q^{0}$ are known from the initial conditions so
a time-stepping procedure can be used.  If a constant time-step is used the
matrices $\matr{C}$, $\matr{H}$ and $\matr{G}$ only need to be constructed
once.  Using this two-level time-integration scheme the most common choice
of time-scheme parameters is $\phi_{u} = 0.5, \phi_{q} = 1.0$.

\section{The MRM for Transient Problems}

\index{Multiple reciprocity method!diffusion equation}The MRM can be applied
to the diffusion equation $\laplacian{u} = \dfrac{1}{\kappa} \delby{u}{t}$
using the fundamental solution of Laplace's equation.  In this case the
forcing function becomes $b_{0} = \dfrac{1}{\kappa} \delby{u}{t}$ and the
recurrence relationship defined by \eqnref{eq:recurr} becomes
\begin{equation}
u_{j+1} = \laplacian{u_{j}} = \dfrac{\del^{j} u}{\del t^{j}}
\end{equation}
The higher-order fundamental solutions are known for Laplace's equation.
In this case the MRM formulation becomes
\begin{equation}
  \fnof{c}{\xi} \fnof{u}{\xi} + \dsuml{j=0}{\infty}
  \goneint{\delby{\omega}{n} \dfrac{\del^{j} u}{\del t^{j}}}{\Gamma} =
  \dsuml{j=0}{\infty} \goneint{\omega \dfrac{\del^{j} q}{\del t^{j}}}{\Gamma}
\end{equation}

The standard BEM numerical procedure can be applied to this boundary
integral equation.  This gives the matrix system
\begin{equation}
  \matr{H_{0}} \vect{u} + \matr{H_{1}} \vect{\dot{u}} + \matr{H_{2}}
  \vect{\ddot{u}} + \ldots = \matr{G_{0}} \vect{q} + \matr{G_{1}} \vect{\dot{q}} +
  \matr{G_{2}} \vect{\ddot{q}} + \ldots
\label{eq:infsystem}
\end{equation}
where the matrices $\matr{H_{1}},\matr{G_{1}}$ etc are the influence
coefficient matrices relating to the higher-order fundamental solutions.
This equation can be solved using a time-integration procedure.

The most common approach is to solve this system numerically by
discretising the time domain and using a time-stepping procedure.  This
requires some interpolation between the two time-levels marked by $m$ and
$m+1$.  This most common approach is to use a linear approximation to
$u$ and $q$ in this time-range
\begin{align}
u & = \pbrac{1 - \phi} u^{m} + \phi u^{m+1} \\
q & = \pbrac{1 - \phi} q^{m} + \phi q^{m+1}
\end{align}
where $\phi$ has a value in the range 0 to 1.  Differentiating these linear
approximations gives
\begin{align}
\dot{u} & = \dfrac{u^{m+1} - u^{m}}{\Delta t^{m}} \\
\dot{q} & = \dfrac{q^{m+1} - q^{m}}{\Delta t^{m}}
\end{align}
and all the other derivatives vanish.

This allows \eqnref{eq:infsystem} to be simplified to
\begin{equation}
\matr{H_{l}} \vect{u^{m+1}} - \matr{G_{l}} \vect{q^{m+1}} = - \matr{H_{r}} \vect{u^{m}} + \matr{G_{r}} \vect{q^{m}}
\label{eq:firstorder}
\end{equation}
where $\matr{H_{l}} = \dfrac{1}{\Delta t^{m}} \matr{H_{1}} + \phi
\matr{H_{0}}$, $\matr{H_{r}} = - \dfrac{1}{\Delta t^{m}} \matr{H_{1}} +
\pbrac{1-\phi} \matr{H_{0}}$, $\matr{G_{l}} = \dfrac{1}{\Delta t^{m}}
\matr{G_{1}} + \phi \matr{G_{0}}$, $\matr{G_{r}} = - \dfrac{1}{\Delta t^{m}}
\matr{G_{1}}+ \pbrac{1-\phi} \matr{G_{0}}$.  This approach is termed a first-order
approach as it removes all but the first derivatives.  A second order approach
can be formulated by using quadratic interpolation of $u$ and $q$ within the
time-range.

Using \eqnref{eq:firstorder} the solution can be advanced in time.  If a
constant time-step is used the matrices $\matr{H_{l}}, \matr{H_{r}},
\matr{G_{l}}$ and $\matr{G_{r}}$ only need to be constructed once outside the
time-stepping loop.  If the boundary conditions are not time-dependent the
boundary conditions only need to be applied once.

%%% Local Variables: 
%%% mode: latex
%%% TeX-master: t
%%% End: 
