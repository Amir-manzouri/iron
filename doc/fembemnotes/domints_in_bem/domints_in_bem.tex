\chapter{Domain Integrals in the BEM}

\section{Achieving a Boundary Integral Formulation}

The principal advantage of the BEM over other numerical methods is the
ability to reduce the problem dimension by one.  This property is
advantagous as it reduces the size of the solution system leading to
improved computational efficiency.  This reduction of dimension also eases
the burden on the engineer as it is only necessary to construct a boundary
mesh to implement the BEM.

To achieve this reduction of dimension it is necessary to formulate the
governing equation as a boundary integral equation.  To achieve a boundary
integral formulation it is necessary to find an appropriate reciprocity
relationship for the problem and to determine an appropriate fundamental
solution\index{Fundamental solution}.  If either of these requirements cannot
be satisfied then a boundary integral formulation cannot be achieved.  The
most common difficulty in applying the BEM is in determining an appropriate
fundamental solution.

A linear differential equation can be expressed in operator form as $Lu =
\gamma$ where $L$ is a linear operator, $\gamma$ is an inhomogeneous source
term and $u$ is the dependent variable.  The fundamental solution for this
equation is a solution of
\begin{equation}
  L^{*} \fnof{\omega}{\vect{x},\vect{\xi}}  + \fnof{\delta}{\vect{\xi}} = 0
\end{equation}
where * indicates the adjoint of the operator $L$ and $\delta$ is the Dirac
delta function.  No specific boundary conditions are prescribed but in some
cases regularity conditions at infinity need to be satisfied. The
fundamental solution is a Green's function which is not required to satisfy
any boundary conditions and is therefore also commonly termed the
free-space Green's function.

The mathematical theory required to determine the fundamental solution of a
constant coefficient PDE is well-developed and has been used successfully
to determine the fundamental solutions for a wide range of constant
coefficient equations \cite{walker:1980} \cite{clements:1978}
\cite{ortner:1987}.  Fundamental solutions are known and have been
published for many of the most important equations in engineering such as
Laplace's equation, the diffusion equation and the wave equation
\cite{brebbia:1984b}.  However, by no means can it be guaranteed that the
fundamental solution to a specific differential equation is known.  In
particular, PDEs with variable coefficients do not, in general, have known
fundamental solutions.  If the fundamental solution to an operator cannot
be found then domain integrals cannot be completely removed from the
integral formulation.  Domain integrals will also arise for inhomogeneous
equations.

\citeasnoun{wu:1985} argued that the BEM has several advantages over
other numerical methods which justify its use for many practical problems -
even in cases where domain integration is required.  He argued that for
problems such as flow problems a wide range of phenomena are described by
the same governing equations. What distinguishes these phenomena is the
boundary conditions of the problem.  For this reason accurate description
of the boundary conditions is vital for solution accuracy.  The BEM
generates a formulation involving both the dependent variable $u$ and the
flux $q$. This allows flux boundary conditions to be applied directly which
cannot be achieved in either the finite element or finite difference
methods.

Another advantage of the BEM over other numerical methods is that it allows an
explicit expression for the solution at an internal point.  This allows a
problem to be subdivided into a number of zones for which the BEM can be
applied individually.  This zoning approach is suited to problems with
significantly different length scales or different properties in different
areas.

Domain integration can be simply and accurately performed in the BEM.
However, the presence of domain integrals in the BEM formulation negates
one of the principal advantages of the BEM in that the problem dimension is
no longer reduced by one.  Several methods have been developed which allow
domain integrals to be expressed as equivalent boundary integrals. In this
section these methods will be discussed.

\section{Removing Domain Integrals due to Inhomogeneous Terms}

Inhomogeneous PDEs occur for a large number of physical problems.  An
inhomogeneous term may arise due to a number of factors including a source
term, a body force term, or due to initial conditions in time-dependent
problems.  An inhomogeneous linear PDE can be expressed in operator form as
$Lu = \gamma$ where $\gamma$ is a known function of position or a non-zero
constant.  If the fundamental solution is known for the operator $L$, the
resulting BEM formulation will be
\begin{equation}
  \matr{H} \vect{u} - \matr{G} \vect{q} = -\goneint{\gamma \omega}{\Omega}
\label{eq:inhomog}
\end{equation}
The domain integral in this formulation does not involve any unknowns so
domain integration can be used directly to solve this equation.  This
requires discretising the domain into internal cells in much the same way
as for the finite element method.  As the domain integral does not involve
any unknown values accurate results can generally be achieved using a
fairly coarse mesh.  This method is simple and has been shown to produce
accurate results \cite{brebbia:1984b}.  This approach, however, requires a
domain discretisation and a numerical domain integration procedure which
reduces the attraction of the BEM over domain-based numerical methods.

\subsection{The Galerkin Vector technique}

For some particular forms of the inhomogeneous function $\gamma$ the domain
integral can be transformed directly into boundary integrals. 

Consider the Poisson equation $\laplacian{u} = \gamma$.  Applying the BEM gives
an equation of the form of \eqnref{eq:inhomog}.  Using Green's second
identity
\begin{equation}
  \goneint{\pbrac{\gamma \laplacian{v} - v
  \laplacian{\gamma}}}{\Omega} = 
  \goneint{\pbrac{\gamma \delby{v}{n} - v \delby{\gamma}{n}}}{\Gamma}
\end{equation}
domain integration can be avoided for certain forms of $\gamma$.  If a
\index{Galerkin vector} $v$ can be found which satisfies $\laplacian{v} = \omega$,
where $\omega$ is the fundamental solution of Laplace's equation, then for
the specific case of $\gamma$ being harmonic ($\laplacian{\gamma} = 0$) Green's
second identity can be reduced to
\begin{equation}
  \goneint{\gamma \omega}{\Omega} = \goneint{\pbrac{\gamma \delby{v}{n} - v
      \delby{\gamma}{n}}}{\Gamma} 
\end{equation}
Therefore if a Galerkin vector can be found and $\gamma$ is harmonic the
domain integral in \eqnref{eq:inhomog} can be expressed as equivalent
boundary integrals.

\citeasnoun{fairweather:1979} determined the Galerkin vector for the
two-dimensional Poisson equation and \citeasnoun{rangogni:1982} determined
the Galerkin vector for the three-dimensional Poisson equation.
\citeasnoun{danson:1981} showed how this method can be applied
successfully for a number of physical problems involving linear isotropic
problems with body forces.  He considered the practical cases where the
body force term arose due to either a constant gravitational load, rotation
about a fixed axis or steady-state thermal loading.  In each of these cases
the domain integral can be expressed as equivalent boundary integrals.

This Galerkin vector approach provides a simple method of expressing 
domain integrals as equivalent boundary integrals.  Unfortunately, it only
applies to specific forms of the inhomogeneous term $\gamma$ (\ie $\gamma$
is required to be harmonic).

\subsection{The Monte Carlo method}

Domain discretisation could be avoided by using a Monte Carlo technique.
This technique approximates a domain integral as a sum of the integrand
at a number of random points. Specifically, in two dimensions, a domain
integral $I$ is approximated as 
\begin{equation}
  I \approx \dfrac{A}{N} \dsuml{i=1}{N} \fnof{f}{x_{i},y_{i}}
\end{equation}
where $\fnof{f}{x_{i},y_{i}}$ is the value of the integrand at random point
$\pbrac{x_{i},y_{i}}$, $N$ is the number of random points used and $A$ is the
area of the region over which the integration is performed.  This
approximation allows a domain integral to be approximated by a summation
over a set of random points so domain integration can be performed without
requiring a domain mesh.  This method has the secondary advantage of
allowing the integration to be performed over a simple geometry enclosing
the problem domain - if a random point is not in the problem domain its
contribution is ignored.

The method was proposed by \citeasnoun{gipson:1987}.  Gipson has
successfully applied this method to a number of Poisson-type problems.
Unfortunately this method often proves to be computationally expensive as a
large number of integration points are needed for accurate domain
integration.  Gipson argues however that, as this method removes the burden
of preparing a domain mesh, the extra computational expense is justified.

\subsection{Complementary Function-Particular Integral method}
\label{sec:anapim}

A more general approach can be developed using particular solutions.  Consider
the linear problem $Lu = \gamma$. $u$ can be considered as the sum of the
complementary function $u_{c}$, which is a solution of the homogeneous
equation $Lu_{c} = 0$, and a particular solution $u_{p}$ which satisfies
$Lu_{p} = \gamma$ but is not required to satisfy the boundary conditions of
the problem.  Applying BEM to the governing equation using the expansion 
 $u =u_{c} + u_{p}$ gives
\begin{equation}
  \matr{H} \vect{u} - \matr{G} \vect{q} = \matr{H} \vect{u_{p}} - \matr{G}
  \vect{q_{p}}
\label{eq:partic}
\end{equation}
If a particular solution $u_{p}$ can be found, all values on the
right-hand-side of \eqnref{eq:partic} are known - reducing the problem to
\begin{equation}
  \matr{H} \vect{u} - \matr{G} \vect{q} = \vect{d}
\end{equation}
where $\vect{d}$ is a vector of known values.  This linear system can be
solved by applying boundary conditions.  

This approach can be applied in a situation where an analytic expression
for a particular solution can be found.  Unfortunately particular solutions
are generally only known for simple operators and for simple forms of
$\gamma$.  Alternatively an approximate particular solution could be
calculated numerically.  \citeasnoun{zheng:1991} proposed a method
where a particular solution is determined by approximating the
inhomogeneous source term using a global interpolation function.  This
approach is a special case of a more general method known as the dual
reciprocity boundary element method.

\section{Domain Integrals Involving the Dependent Variable}

Consider the linear homogeneous PDE $Lu = 0$.  For many operators the
fundamental solution to the operator $L$ may be unobtainable or may be in
an unusable form.  This is especially likely if $L$ involves variable
coefficients for which case it has been shown that it is particularly
difficult to find a fundamental solution.  Instead, a BEM formulation can
be derived based on a related operator $\hat{L}$ with known fundamental
solution.  A BEM formulation for $Lu = 0$ based on the operator $\hat{L}$
will be of the form
\begin{equation}
  \matr{H} \vect{u} - \matr{G} \vect{q} 
            = -\goneint{\pbrac{\hat{L} - L} u \omega}{\Omega}
\label{eq:relate}
\end{equation}
where $\omega$ is the fundamental solution corresponding to the operator 
$\hat{L}$. This integral equation is similar to \eqnref{eq:inhomog}.  
However in this case the domain integral term involves the dependent
variable $u$.  This problem could be solved using domain integration where
the internal nodes are treated as formal problem unknowns. 

\subsection{The Perturbation Boundary Element Method}

\index{Perturbation Boundary Element Method@Perturbation BEM}
\citeasnoun{rangogni:1986} proposed solving variable coefficient PDEs by
coupling the boundary element method with a perturbation method. He
considered the two-dimensional generalised Laplace equation
\begin{equation}
  \nabla \cdot \pbrac{ \fnof{\kappa}{x,y}\nabla\fnof{V}{x,y} } = 0
\label{eq:rangge2}
\end{equation}
Using the substitution $\fnof{V}{x,y} = \kappa^{-\dfrac{1}{2}} \fnof{u}{x,y}$
\eqnref{eq:rangge2} can be recast as a heterogeneous Helmholtz equation
\begin{equation}
  \laplacian{u} + \fnof{f}{x,y}u = 0
\label{eq:varhelm2}
\end{equation}
where $f$ is a known function of position.

\citename{rangogni:1986} treated 
this equation as a perturbation about Laplace's equation. He considered the class of equations
\begin{equation}
  \laplacian{u} + \varepsilon \fnof{f}{x,y} u = 0 \qquad 
    \text{where } 0 \leq \varepsilon \leq 1
\label{eq:eqnfamily}
\end{equation}
for which he sought a solution of the form
\begin{equation}
  u = u_{0} + \varepsilon u_{1} + \varepsilon^{2} u_{2} + \ldots =
  \dsuml{j=0}{\infty} u_{j} \varepsilon^{j}
\label{eq:uexpan}
\end{equation}
Substituting \eqnref{eq:uexpan} into \eqnref{eq:eqnfamily} and grouping powers
 of $\varepsilon$ gives
\begin{equation}
  \laplacian{u_{0}} + \varepsilon \pbrac{\laplacian{u_{1}} + fu_{0}} +
  \varepsilon^{2} \pbrac{\laplacian{u_{2}} + fu_{1}} + \ldots = 0
\label{eq:groups}
\end{equation}
A solution will only exist for all values of $\varepsilon$ if the terms at
each power of $\varepsilon $ equal zero. This allows \eqnref{eq:groups} to be
treated as an infinite series of distinct problems which can be solved
using the boundary element method. $u_{0}$ can be found by solving 
 $\laplacian{u_{0}} = 0$ which Rangogni assumes will satisfy the boundary 
conditions of the original problem.  Each successive $u_{j}$ can then be
found by solving a Poisson equation with homogeneous boundary conditions as 
 $u_{j-1}$ has been previously determined.  Rangogni used a domain
 discretisation to solve these Poisson problems.

\eqnref{eq:varhelm2} is a particular member of this family of equations for
which $\varepsilon =1$.  The solution to \eqnref{eq:varhelm2} is therefore
given by $\dsuml{j=0}{\infty} u_{j}$. Rangogni reported that in practice
this series converged rapidly and in his numerical examples he achieved
accurate results using only $u_{0}$ and $u_{1}$.

\citeasnoun{rangogni:1991} extended this coupled perturbation - boundary
element method to the general second-order variable coefficient PDE
\begin{equation}
  \laplacian{u} + \fnof{f}{x,y} \delby{u}{x} + \fnof{g}{x,y} \delby{u}{y} =
  \fnof{h}{x,y}
\end{equation}
He considered the family of equations
\begin{equation}
  \laplacian{u} + \varepsilon \sqbrac{\fnof{f}{x,y} \delby{u}{x} +
    \fnof{g}{x,y} \delby{u}{y}} = \fnof{h}{x,y} 
    \quad (0 \leq \varepsilon \leq 1)
\label{eq:family2}
\end{equation}
Applying the perturbation method to this family of equations allows
\eqnref{eq:family2} to be expressed as an infinite series of distinct Poisson
equations which can be solved using the boundary element method. Again
Rangogni used an domain mesh to solve these Poisson equations.  Rangogni
found that in practice convergence was rapid and accurate results were
produced.

\citeasnoun{gipson:1987b} considered a class of hyperbolic and elliptic
problems which can be transformed into an inhomogeneous Helmholtz equation.
They used the perturbation method to recast this as an infinite sequence of
Poisson equations. They avoided domain discretisation by using a Monte
Carlo integration technique \cite{gipson:1987} to evaluate the required
domain integrals.

\citeasnoun{lafe:1987} used the perturbation method to solve
steady-state groundwater flow problems in heterogeneous aquifers.  They
showed the method produced accurate results for simply varying hydraulic
conductivities with convergence after two or three terms.
\citename{lafe:1987} investigated the convergence of the perturbation
method.  They found that for rapidly varying hydraulic conductivity
convergence is not guaranteed.  From this investigation they concluded that
accurate results can be obtained so long as the hydraulic conductivity does
not vary by more than one order of magnitude within the solution domain.
If the hydraulic conductivity variation is more significant they recommend
using the perturbation method in conjunction with a subregion technique so
that the variation of conductivity within each subregion satisfies their
requirements.  This process could become computationally expensive,
particularly if convergence is not rapid, as the solution of multiple
subproblems will be required within each subregion.

%%%
%goes brown below here
%%%
\subsection{The Multiple Reciprocity Method}

The multiple reciprocity method 
 \index{Multiple reciprocity method|(} (MRM) was
initially proposed by \citeasnoun{nowak:1987} for the solution of transient
heat conduction problems.  Since then the method has been successfully applied
to a wide range of problems.  The MRM can be viewed as a generalisation of the
Galerkin vector approach.  Instead of using one higher-order fundamental
solution, the Galerkin vector, to convert the remaining domain integrals to
equivalent boundary integrals a series of higher-order fundamental solutions
is used.

 \index{Multiple reciprocity method!Poisson equation} Consider the Poisson equation
\begin{equation}
  \laplacian{u} = b_{0}
\end{equation}
where $b_{0} = \fnof{b_{0}}{\vect{x}}$ is a known function of position.
Applying BEM to this equation, using the fundamental solution to the Laplace
operator, gives
\begin{equation}
  \fnof{c}{\vect{\xi}} \fnof{u}{\vect{\xi}} + 
  \goneint{u \delby{\omega_{0}}{n}}{\Gamma} + 
  \goneint{b_{0} \omega_{0}}{\Omega} = \goneint{\omega_{0} \delby{u}{n}}{\Gamma}
\label{eq:mrmstart}
\end{equation}
where $\omega_{0}$ is the known fundamental solution to Laplace's equation
applied at point $\vect{\xi}$.  To avoid domain discretisation the domain integral
in \eqnref{eq:mrmstart} needs to be expressed as equivalent boundary
integrals.  Using MRM this is achieved by defining a higher-order
fundamental solution $\omega_{1}$ such that
\begin{equation}
  \laplacian{\omega_{1}} = \omega_{0}
\end{equation}
Using this higher-order fundamental solution the domain integral in
\eqnref{eq:mrmstart} can be written as
\begin{equation}
  \goneint{b_{0} \omega_{0}}{\Omega} =
  \goneint{b_{0}\laplacian{\omega_{1}}}{\Omega} 
\end{equation}
or 
\begin{equation}
  \goneint{b_{0} \omega_{0}}{\Omega} = 
  \goneint{\pbrac{u \delby{\omega_{1}}{n} - \omega_{1} \delby{u}{n}}}{\Gamma} +
  \goneint{\omega_{1} \laplacian{b_{0}}}{\Omega}
\label{eq:delb}
\end{equation}
This formulation has generated a new domain integral.  $b_{0}$ is a known
function so we can introduce a new function $b_{1}$ which can be determined
analytically from the relationship
\begin{equation}
  b_{1} = \laplacian{b_{0}}
\end{equation}
giving 
\begin{equation}
  \goneint{\omega_{1} \laplacian{b_{0}}}{\Omega} = 
  \goneint{\omega_{1} b_{1}}{\Omega}
\end{equation}
This process can be repeated by introducing a new higher-order fundamental
solution $\omega_{2}$ such that
\begin{equation}
  \laplacian{\omega_{2}} = \omega_{1}
\end{equation}
and continuing until convergence is reached.

This procedure is based on the recurrence relationships
%\section{}

\begin{alignat}{2}
  b_{j+1} & = \laplacian{b_{j}} & \qquad \text{for }j &= 0,1,2,\ldots
  \label{eq:recurr} \\
  \laplacian{\omega_{j+1}} & = \omega_{j} & \qquad \text{for }j &= 0,1,2,\ldots
  \label{eq:higher}
\end{alignat}
Using these recurrence relationships gives the boundary integral formulation
\begin{equation}
  \fnof{c}{\vect{\xi}} \fnof{u}{\vect{\xi}} + 
  \goneint{\pbrac{u\delby{\omega_{0}}{n} - \omega_{0} \delby{u}{n}}}{\Gamma} +
  \dsuml{j=0}{\infty}   \goneint{\pbrac{b_{j}\delby{\omega_{j+1}}{n} 
      - \omega_{j+1} \delby{b_{j}}{n}}}{\Gamma} = 0
\end{equation}
which is an exact formulation if the infinite series converges.  Errors are
only introduced at the stage of boundary discretisation.  

Introducing interpolattion functions and discretising the boundary gives
the matrix system
\begin{equation}
  \matr{H_{0}} \vect{u} - \matr{G_{0}} \vect{q} = \dsuml{j=0}{\infty}
  \pbrac{\matr{H_{j+1}} \vect{P_{j}} - \matr{G_{j+1}} \vect{R_{j}}}
\end{equation}
where $\matr{H_{j+1}}$ and $\matr{G_{j+1}}$ are influence coefficient
matrices corresponding to the higher-order fundamental solutions and
$\vect{p_{j}}$ and $\vect{r_{j}}$ contain the nodal values of $b_{j}$ and its
normal derivative. 

The MRM can be applied based on operators other than the Laplace operator.
This approach relies on knowledge of the higher-order fundamental solutions
necessary for application of the method.  These solutions have been
determined and successfully used for the Laplace operator in both two and
three dimensions but the extension of the method to other equation types
needs further research.  \citeasnoun{itagaki:1993} have determined the higher
order fundamental solutions for the two-dimensional modified Helmholtz
equation.

The MRM can be extended to other equations by allowing the forcing function
$b_{0}$ to be a general function such that $b_{0} = \fnof{b_{0}}{\vect{x},u,t}$.
The MRM will be restricted to cases where the recurrence relationships -
\eqnrefs{eq:recurr}{eq:higher} - can be employed.  \citeasnoun{brebbia:1989}
have applied the MRM to the Helmholtz equation $\laplacian{u} + \kappa^{2} u =
0$ where $b_{0} = -\kappa^{2} u$ and the recurrence relationship defined by
\eqnref{eq:recurr} becomes simply 
\index{Multiple reciprocity method!Helmholtz equation}
\begin{equation}
  u_{j+1} = \laplacian{u_{j}} = - \kappa^{2j} u
\end{equation}
In this case the boundary integral formulation will be
\begin{equation}
  \fnof{c}{\vect{\xi}} \fnof{u}{\vect{\xi}} + \dsuml{j=0}{\infty}
  \goneint{\kappa^{2j} \pbrac{u \delby{\omega_{j}}{n} -
    \omega_{j} \delby{u}{n}}}{\Gamma} = 0
\end{equation}
\index{Multiple reciprocity method|)}

\subsection{The Dual Reciprocity Boundary Element Method}

\subsubsection{Equation Derivation}

\index{Dual reciprocity boundary element method@Dual reciprocity BEM|)}
The dual reciprocity boundary element method (DR-BEM) was developed to
avoid the need for domain integration in cases where the fundamental
solution of the governing differential equation is unknown or is
impractical to apply.  Instead the DR-BEM is applied using an appropriate
related operator with known fundamental solution.  The most common choice
is the Laplace operator \cite{partridge:1992} and in this chapter the DR-BEM
will be illustrated for this choice.

Consider a second-order PDE which can be expressed in the form
\begin{equation}
  \laplacian{u} = b
\label{eq:drmbase}
\end{equation}
The forcing function $b$ can be completely general. If $b = \fnof{b}{\vect{x}}$
then $b$ is a known function of position and the differential equation
described is simply the Poisson equation. For potential problems $b =
\fnof{b}{\vect{x},u}$ and for transient problems $b = \fnof{b}{\vect{x},u,t}$.
Applying the BEM to \eqnref{eq:drmbase} will give
\begin{equation}
  \matr{H} \vect{u} - \matr{G} \vect{q} = -\goneint{b \omega}{\Omega}
\end{equation}
where $\omega$ is the known fundamental solution to Laplace's equation.
The aim of the DR-BEM is to express the domain integral due to the forcing
function $b$ as equivalent boundary integrals.

The DR-BEM uses the idea of approximating $b$ using
interpolation functions.  A global approximation to $b$ of the form
\label{page:alpha}
\begin{equation}
  b = \dsuml{j=1}{M} \alpha_{j}f_{j}
\label{eq:globalapprox}
\end{equation}
is proposed.  $\alpha_{j}$ are unknown coefficients and $f_{j}$ are
approximating functions used in the interpolation and are generally chosen
to be functions of the source point and the field point of the fundamental
solution. The approximating functions $f_{j}$ are applied at $M$ different
collocation points - called poles - generally most, but not all, of which
are located on the boundary of the problem domain.

As discussed in the previous chapter the solution to a linear PDE $Lu =
\gamma$ can be constructed as the sum of a complimentary function $u_{c}$
(which satisfies the homogeneous equation $Lu_{c} = 0$) and a particular
solution $u_{p}$ to the equation $Lu_{p} = \gamma$.  Instead of using a
single particular solution, which may be difficult to determine, the DR-BEM
employs a series of particular solutions $\hat{u}_{j}$ which are related to
the approximating functions $f_{j}$ as shown in \eqnref{eq:harmonic}.
\begin{equation}
  \laplacian{\hat{u}_{j}} = f_{j} \quad j = 1, \ldots ,M
\label{eq:harmonic}
\end{equation}
By substituting \eqnrefs{eq:globalapprox}{eq:harmonic} into \eqnref{eq:drmbase}
the forcing function $b$ is approximated by a weighted summation of
particular solutions to the Poisson equation.
\begin{equation}
  \laplacian{u} = \dsuml{j=1}{M} \alpha_{j} \laplacian{\hat{u}_{j}}
\label{eq:rhsapprox}
\end{equation}
The DR-BEM essentially constructs an approximate particular solution to the
governing PDE as a summation of localised particular solutions.  

With the governing equation rewritten in the form of \eqnref{eq:rhsapprox} the
standard boundary element approach can be applied. \Eqnref{eq:rhsapprox} is
multiplied by a weighting function $\omega$ and integrated over the domain.
Green's theorem is applied twice and the fundamental solution of the
Laplacian is used to remove the remaining domain integrals. The name dual
reciprocity BEM is derived from the application of reciprocity
relationships to both sides of \eqnref{eq:rhsapprox}. After applying these
steps \eqnref{eq:drmmaths} is obtained, where the fundamental solution pole
is applied at point $\vect{\xi}$.
\begin{multline}
  \fnof{c}{\vect{\xi}} \fnof{u}{\vect{\xi}} + \goneint{\pbrac{u
    \delby{\omega}{n} - \omega \delby{u}{n}}}{\Gamma} \\ =
  \dsuml{j=1}{M} \alpha_{j} \pbrac{\fnof{c}{\vect{\xi}}
    \fnof{u_{j}}{\vect{\xi}} + \goneint{\pbrac{\hat{u}_{j}
      \delby{\omega}{n} - \omega \delby{hat{u}_{j}}{n}}}{\Gamma}}
\label{eq:drmmaths}
\end{multline}
In implementing a 
numerical solution of this equation similar steps are taken as for 
the standard BEM.  The boundary is
discretised into elements and interpolation functions are introduced to
approximate the dependent variable within each element. 

The form of each $\hat{u}_{j}$ is known from \eqnref{eq:harmonic} once the
approximating functions $f_{j}$ have been defined. It is not necessary to
use interpolation functions to approximate each $\hat{u}_{j}$. However by
using the same interpolation functions to approximate $u$ and $\hat{u}_{j}$
the numerical implementation will generate the same matrices $\matr{H}$ and
$\matr{G}$ on both sides of \eqnref{eq:drmmaths}. The error generated by
approximating each $\hat{u}_{j}$ in this manner has been found to be small
and can be justified by the improved computational efficiency of the method
\cite{partridge:1992}.

The application of this method results in the system
\begin{equation}
  \matr{H} \vect{u} - \matr{G} \vect{q} = \dsuml{j=1}{N+I} \alpha_{j}
  \pbrac{\matr{H} \vect{\hat{u}_{j}} - \matr{G} \vect{\hat{q}_{j}}}
\label{eq:sumin}
\end{equation}
where the $M$ poles were chosen to be the $N$ boundary nodes plus $I$ internal
points so that $M = N+I$. Although it is not generally necessary to include
poles at internal points it has been found that in general improved accuracy
is achieved by doing so \cite{nowak:1992}.  It has been shown that for many
problems \cite{partridge:1992} \cite{cruse:1993} using boundary points only in
this procedure is insufficient to define the problem.  In general using
internal points is likely to improve the solution accuracy as it increases the
number of degrees of freedom.  No theory has been developed of how many
internal collocation points should be used for optimal accuracy, or where
these points should be positioned within the problem domain. Using internal
poles in this interpolation does not require domain discretisation - it is
only necessary to specify the coordinates of the internal collocation points.
The internal points can be chosen to be locations where the solution is of
interest.

The $\vect{\hat{u}_{j}}$ and $\vect{\hat{q}_{j}}$ vectors can be treated as
columns of the matrices $\matr{\hat{U}}$ and $\matr{\hat{Q}}$ respectively.
This allows \eqnref{eq:sumin} to be rewritten as
\begin{equation}
  \matr{H} \vect{u} - \matr{G} \vect{q} = \pbrac{\matr{H} \matr{\hat{U}} -
    \matr{G} \matr{\hat{Q}}} \vect{\alpha}
\label{eq:drmalpha}
\end{equation}
where $\vect{\alpha}$ is a vector containing the nodal values of $\alpha$.
To solve this system it is necessary to evaluate $\vect{\alpha}$.
$\vect{\alpha}$ is defined by \eqnref{eq:globalapprox} which, for the nodal
values, can be expressed in matrix form as $\vect{b} = \matr{F}
\matr{\alpha}$. If the $\matr{F}$ matrix is nonsingular this expression can
be rearranged to give \eqnref{eq:expalpha} which provides an explicit
expression for $\matr{\alpha}$.
\begin{equation}
  \vect{\alpha} = \matr{F}^{-1} \vect{b}
\label{eq:expalpha}
\end{equation}
Including this explicit expression for $\vect{\alpha}$ in 
\eqnref{eq:drmalpha} gives
\begin{equation}
  \matr{H} \vect{u} - \matr{G} \vect{q} = \pbrac{\matr{H} \matr{\hat{U}} -
    \matr{G} \matr{\hat{Q}}} \matr{F}^{-1} \vect{b}
\label{eq:drmsystem}
\end{equation}
The approach taken to solve this equation will depend on the form of $b$.

\subsubsection{The Approximating Function $f$}
\label{sec:approxfnchoice}

\index{Dual reciprocity boundary element method@Dual reciprocity
  BEM!approximating function}
The accuracy of the DR-BEM hinges on the accuracy of the global
approximation to the forcing function $b$ (defined by
\eqnref{eq:globalapprox}).  Therefore the choice of the approximating
functions $f_{j}$ is a key consideration when implementing the DR-BEM. The
only requirement so far prescribed on the form of the approximating
functions $f_{j}$ is that the $\matr{F}$ matrix generated should be
nonsingular and that the related particular solutions $\hat{u}_{j}$ can be
determined and can be expressed in a practical closed form.  Some work has
been conducted into investigating what form of $f_{j}$ should be used in a
given situation to provide the highest accuracy and computational
efficiency.

Usually a form of $f_{j}$ is defined and this can be used, applying
\eqnref{eq:harmonic}, to specify $\hat{u}$ and $\hat{q}$.  The fundamental
solution of Laplace's equation is $\fnof{\omega}{\vect{x},\vect{\xi}} =
-\dfrac{1}{2\pi} \ln r$ in two-dimensional space and
$\fnof{\omega}{\vect{x},\vect{\xi}} = \dfrac{1}{4 \pi r}$ in three-dimensional
space - where $r$ is the Euclidean distance between the field point
$\vect{x}$ and the source point $\vect{\xi}$ of the fundamental
solution.  Due to the dependence of this fundamental solution only on $r$
the approximating function is generally chosen to be some radial function
\ie  $f_{j} = \fnof{f_{j}}{r}$. Several other options for $f_{j}$ have been tried
\cite{partridge:1992} but it has been found that in general the most accurate
results were generated using some radial function.  For both two and
three-dimensional problems \citeasnoun{wrobel:1986} recommended choosing
$f_{j}$ from the series
\begin{equation}
  f_{j} = 1 + r_{j} + r_{j}^{2} +\ldots+ r_{j}^{m}
\label{eq:fdefine}
\end{equation}
where $r_{j}$ is the distance between the field point (node $j$) and the
DR-BEM collocation point (node $i$).  They showed that accurate results can
be achieved using some combination of terms from this series.  Generally
the same approximating function $f_{j}$ is used at all the collocation
points so in this thesis, for simplicity, the form of approximating
functions $f_{j}$ will be referred to by a single $f$.

Choosing $f$ to be a function of only one variable simplifies the process
of determining $\hat{u}$ and $\hat{q}$.  For two-dimensional problems, if $f
= \fnof{f}{r}$ then the relationship
\begin{equation}
  \laplacian{\hat{u}} = \fnof{f}{r} 
\end{equation}
can be reduced to the ordinary differential equation
\begin{equation}
  \dtwosqby{\hat{u}}{r} + \dfrac{1}{r} \dby{\hat{u}}{r} = f 
\end{equation}
Using $f$ defined by \eqnref{eq:fdefine} the corresponding forms of $\hat{u}$
and $\hat{q}$, for two-dimensional problems, can be shown to be
\begin{align}
  \hat{u} & = \dfrac{r^{2}}{4} + \dfrac{r^{3}}{9} + \ldots +
  \dfrac{r^{m+2}}{\pbrac{m+2}^{2}} \\ \hat{q} & = \pbrac{r_{x} \delby{x}{n} +
    r_{y} \delby{y}{n}} \pbrac{\dfrac{1}{2} + \dfrac{r}{3} +\ldots+
    \dfrac{r^{m}}{m+2}}
\end{align}
where $r_{x} = x_{j} - x_{i}$ and $r_{y} = y_{j} - y_{i}$.

Any combination of terms from \eqnref{eq:fdefine} can be used for specifying
$f$. It has been found that in general including higher-order terms leads
to little improvement in accuracy \cite{partridge:1992}. The most commonly
used form is $f = 1 + r$ as this approximation will generally give accurate
results with greater computational efficiency than other choices.

\Eqnref{eq:fdefine} was recommended as a basis for the approximating function
$f$ due to the particular form of the fundamental solution of Laplace's
equation and its dependence on $r$ only. If a different operator is used as
the basis of the DR-BEM then it is likely a different form of $f$ will be
more appropriate.  The choice of $f$ in this case will be discussed in
\secref{sec:otheroperators}.

The performance of the DR-BEM hinges on the choice of the approximating
function $f$.  The theory of how to determine the best approximating
function is therefore a vital component of the DR-BEM.  Unfortunately the
approximating function has generally been chosen and used in a rather
ad-hoc manner.  Recently some more formal analysis of the use of
approximating functions has been undertaken. 

\citeasnoun{golberg:1994} argued that a formal analysis of the approximating
function $f$ can be undertaken using the theory of radial basis functions.
Radial basis functions are a generalisation of cubic splines in
multi-dimensions.  Cubic splines are known to be optimal for one-dimensional
interpolation.  Therefore, rather than being an arbitrary choice, it seems
that choosing $f$ to be a radial function is a logical extension for two
or three-dimensional problems.  \citename{golberg:1994} showed that, for the
Poisson equation, choosing $f$ to be a radial basis function ensures
convergence of the DR-BEM.

They also demonstrated that $f = 1 + r$ is a specific member of the group
of radial basis functions.  The theory of using radial basis functions for
multi-dimensional approximation is fairly advanced.  It has been shown that
$f = r$ is optimal for three-dimensional problems which justifies the use
of $f = 1 + r$ when applying the DR-BEM to three-dimensional problems - the
constant is included to ensure a non-zero diagonal for $\matr{F}$.  However
for two-dimensional problems it has been shown that optimal approximation
is attained using the thin plate spline $f = r^{2} log r$.  This
observation lead \citename{golberg:1994} to suggest that choosing $f$ to be a
thin plate spline may improve the accuracy of the DR-BEM in two dimensions.
Recently \citeasnoun{golberg:1995} has published a review of the DR-BEM
concentrating on developments since 1990 concerning the numerical
evaluation of particular solutions.

\subsubsection{Inhomogeneous Equations}

If the forcing function $b$ is a function 
of position only then the differential equation under consideration 
is simply Poisson's equation. In this case it is not necessary to 
invert the $\matr{F}$ matrix as $\vect{\alpha}$ can simply be calculated from 
$\vect{b} = \matr{F} \vect{\alpha}$ using Gaussian
elimination. \Eqnref{eq:drmsystem} can be rewritten as
\begin{equation}
  \matr{H} \vect{u} - \matr{G} \vect{q} = \vect{d}\quad\text{where
    }\vect{d} = \pbrac{\matr{H} \matr{\hat{U}} - \matr{G} \matr{\hat{Q}}}
  \vect{\alpha}
\label{eq:drmpoisson}
\end{equation}
By applying boundary 
conditions \eqnref{eq:drmpoisson} can be 
reduced to a linear system $\matr{A} \vect{x} = \vect{\tau}$ which can be
solved to give the unknown nodal values of $u$ and $q$.

\citeasnoun{zheng:1991} and \citeasnoun{coleman:1991} have proposed
a method which uses a global shape function to construct an approximate
particular solution.  As discussed by \citeasnoun{polyzos:1994} this method
is essentially equivalent to the DR-BEM.  However,
\citename{zheng:1991} and \citename{coleman:1991} suggested several
alternative ways of determining the unknown coefficients $\alpha_{j}$ for
inhomogeneous equations.  \citeasnoun{zheng:1991} used a least-squares
method where they minimised the sum of squares
\begin{equation}
  S = \dsuml{m=1}{M} \pbrac{\fnof{b}{r_{m}} - \dsuml{j=1}{N}
    \alpha_{j} \fnof{f_{j}}{r_{m}}}
\end{equation}
using singular value decomposition.  For large systems they found the
computational efficiency could be improved by employing the conjugate
gradient method.  \citeasnoun{coleman:1991} successfully solved
inhomogeneous potential and elasticity problems which are governed by
operators other than the Laplacian.

\subsubsection{Elliptic Problems}

\index{Dual reciprocity boundary element method@Dual reciprocity
  BEM!elliptic problems}
If $b$ is a function of the dependent variable then $\vect{\alpha}$ will also 
be a function of the dependent variable. Consider, for example, the 
linear second-order differential equation
\begin{equation}
  \laplacian{u} + u = 0
\label{eq:helmtypeeqn}
\end{equation}
In this case $b = -u$ so $\vect{\alpha} = \matr{F}^{-1} \vect{-u}$.
Applying the DR-BEM to \eqnref{eq:helmtypeeqn}, based on the fundamental
solution to Laplace's equation, gives
\begin{equation}
  \matr{H} \vect{u} - \matr{G}\vect{q} = - \pbrac{\matr{H} \matr{\hat{U}} -
    \matr{G} \matr{\hat{Q}}} \matr{F}^{-1} \vect{u}
\end{equation}
which can be rearranged to give 
\begin{equation}
  \pbrac{\matr{H} + \matr{S}} \vect{u} = \matr{G}
  \vect{q}\quad\text{where }\matr{S} = \pbrac{\matr{H} \matr{\hat{U}}
    - \matr{G} \matr{\hat{Q}}} \matr{F}^{-1}
\label{eq:drmpotential}
\end{equation}
Again, by applying boundary conditions \eqnref{eq:drmpotential} can be reduced 
to a linear system $\matr{A} \matr{x} = \matr{\tau}$ which can be solved to
determine the unknown nodal values.

Due to the presence 
of the fully-populated $\matr{F}^{-1}$ matrix in \eqnref{eq:drmpotential}
 it is not possible to solve the boundary problem and internal 
problem separately. Instead the solution can be treated as a coupled 
problem and the solutions at boundary and internal nodes are generated 
simultaneously.

\paragraph{Derivative Terms}
\label{sec:deriv}

\index{Dual reciprocity boundary element method@Dual reciprocity
  BEM!derivative terms}
The DR-BEM can also be applied for elliptic problems where
$b$ involves derivatives of the dependent variable
\cite{partridge:1992}. Consider, for example, the differential equation 
\begin{equation}
  \laplacian{u} + \delby{u}{x} = 0
\end{equation}
In this case applying DR-BEM, using the Laplace fundamental solution, gives
\begin{equation}
  \matr{H} \vect{u} - \matr{G} \vect{q} = - \pbrac{\matr{H} \matr{\hat{U}} -
    \matr{G} \matr{\hat{Q}}} \matr{F}^{-1} \vect{\delby{u}{x}}
\end{equation}
To solve this problem it is necessary to relate the nodal values of $u$ to
the nodal values of $\delby{u}{x}$. This is achieved by
using interpolation functions to approximate $\vect{u}$ in a similar manner
as was used to approximate $b$ in \eqnref{eq:globalapprox}.  A global
approximation function of the form
\begin{equation}
  u = \dsuml{j=1}{M} \fnof{\phi_{j}}{x,y} \beta_{j}
\end{equation}
can be used to approximate $u$ where $\phi_{j}$ are the chosen
interpolation functions and $\beta_{j}$ are the unknown coefficients.  In
system form this can be expressed as
\begin{equation}
  \vect{u} = \matr{\Phi} \vect{\beta}
  \label{eq:uapprox}
\end{equation}
Although it is not necessary, equating $\matr{\Phi}$ to $\matr{F}$ 
improves the computational efficiency of the method as only one matrix 
inversion procedure is required. Differentiating \eqnref{eq:uapprox} gives 
\begin{equation}
  \vect{\delby{u}{x}} = \matr{\delby{\Phi}{x}} \vect{\beta}
  \label{eq:uappder}
\end{equation}
Choosing $\matr{\Phi} = \matr{F}$ and inverting \eqnref{eq:uapprox} to give an
explicit expression for $\vect{\beta}$ allows \eqnref{eq:uappder} to be
rewritten as
\begin{equation}
  \vect{\delby{u}{x}} = \matr{\delby{F}{x}} \matr{F}^{-1} \vect{u}
  \label{eq:dudxexp}
\end{equation}
\Eqnref{eq:drmsystem} can now be rewritten as
\begin{equation}
  \pbrac{\matr{H} + \matr{R}} \vect{u} = \matr{G}
  \vect{q}\quad\text{where }\matr{R} = \pbrac{\matr{H} \matr{\hat{U}}
    - \matr{G} \matr{\hat{Q}}} \matr{F}^{-1} \matr{\delby{F}{x}}
  \matr{F}^{-1}
\label{eq:drmderiv}
\end{equation}
By applying boundary conditions \eqnref{eq:drmderiv} can be reduced to 
a linear system which can be solved to give the unknown nodal values.

As mentioned earlier, the approximating function $f$ is generally chosen to be
$f = 1 + r$.  This can lead to numerical problems if derivative terms are
included in the forcing function $b$.  As shown in \eqnref{eq:dudxexp}
derivative terms require derivatives of $f$ to be evaluated.  For example,
evaluating the $\matr{\delby{F}{x}}$ matrix requires calculation of
$\delby{f}{x}$.  Using the approximating function $f = 1 + r$ gives
\begin{equation}
  \delby{f}{x} = \delby{f}{r} \delby{r}{x} = \delby{f}{r} \dfrac{r_{x}}{r} 
\end{equation}
This derivative function can become singular, so - as shown by
\citeasnoun{zhang:1993} - significant numerical error may result.  This
will especially be the case in problems where collocation points are located
close together.

\citeasnoun{zhang:1993} suggested two possibilities for avoiding this
problem.  The first suggestion involved using a mapping procedure to map
the governing equation to an equation without convective terms. This method
was shown to produce accurate results but is somewhat cumbersome and can
only be applied to linear problems.  A simpler approach is to choose an
approximating function which does not lead to singularities for convective
terms.  \citename{zhang:1993} recommended use of either $f = 1 + r^{3}$
or $f = 1 + r^{2}+ r^{3}$.  These approximating functions produce accurate
results and can be simply applied for both linear and nonlinear problems.
\citename{zhang:1993} recommended the adoption of these approximating
functions for all use of the DR-BEM.

The same idea of using \eqnref{eq:uapprox} to allow nodal values of $u$ to be
associated to its derivatives can be applied to extend the DR-BEM to cases
involving higher-order derivatives or cross derivatives of the dependent
variable.  Appropriate approximating functions need to be chosen to avoid
the problem of singularities.

\paragraph{Variable Coefficients}

\index{Dual reciprocity boundary element method@Dual reciprocity
  BEM!variable coefficients}
The DR-BEM can be readily extended to equations with variable coefficients.
Consider the variable coefficient Helmholtz equation
\begin{equation}
  \laplacian{u} + \fnof{\kappa}{\vect{x}} u = 0
\end{equation}
where $\kappa$ is a function of position - $\kappa = \fnof{\kappa}{x,y}$ in
two dimensions.  If the DR-BEM is applied using the known fundamental solution
to the Laplace operator then the forcing function is $b = -\kappa u$.
Applying the DR-BEM gives
\begin{equation}
  \matr{H} \vect{u} - \matr{G} \vect{q} = \pbrac{\matr{H} \matr{\hat{U}} -
    \matr{G} \matr{\hat{Q}}} \matr{F}^{-1} \vect{b}
  \label{eq:vcb}
\end{equation}
where $\vect{b}$ is a vector of the nodal values of the forcing function
$b$.  The relationship $b = -\kappa u$ can be written in matrix form as
$\vect{b} = - \matr{K} \vect{u}$ where $\matr{K}$ is a diagonal matrix
containing the nodal values of $\fnof{\kappa}{x,y}$ \ie
\begin{equation}
  \matr{K} =  \begin{bmatrix} \fnof{\kappa}{x_{1},y_{1}} & 0 &
      \cdots & 0 \\ 0 & \fnof{\kappa}{x_{2},y_{2}} & \cdots & 0 \\ \vdots &
      \vdots & \ddots & \vdots \\ 0 & 0 & \cdots & \fnof{\kappa}{x_{M},y_{M}} 
\end{bmatrix}
\end{equation}
where $M$ is the number of collocation points used in applying the DR-BEM.

Using this matrix expression for $\vect{b}$ \eqnref{eq:vcb} can be rearranged
to give
\begin{equation}
  \pbrac{\matr{H} + \matr{S} \matr{K}} \vect{u} = \matr{G}
  \vect{q}\quad\text{where}\matr{S} = \pbrac{\matr{H} \matr{\hat{U}} -
    \matr{G} \matr{\hat{Q}}} \matr{F}^{-1}
\end{equation}
which is a boundary-only expression for the variable coefficient Helmholtz
equation.  This method is general and can easily be extended to accommodate
variable coefficient derivative terms and a sum of variable coefficient
terms. 

\paragraph{Formulating the DR-BEM for a General Elliptic Problem}
\label{sec:genform}

In this section it has been shown how the DR-BEM can be applied for
elliptic problems with varying forms of $b$.  The DR-BEM can be applied in
cases where $b$ involves a sum of terms due to the basic property
\begin{equation}
  \goneint{\pbrac{b_{1} + b_{2}}}{\Omega} = \goneint{b_{1}}{\Omega} +
  \goneint{b_{2}}{\Omega}
\end{equation}
Consider a two-dimensional equation of the form
\begin{equation}
  \laplacian{\fnof{u}{x,y}} = \fnof{k}{x,y} u + \fnof{l}{x,y} \delby{u}{x} +
  \fnof{m}{x,y} \delby{u}{y} + \fnof{n}{x,y}
\end{equation}
Applying the DR-BEM to this equation gives a matrix system of the form
\begin{equation}
  \pbrac{\matr{H} - \matr{R}} \vect{u} = \matr{G} \vect{q} + \matr{S} \vect{n}
\end{equation}
where
\begin{align}
  \matr{S} & = \pbrac{\matr{H} \matr{\hat{U}} - \matr{G} \matr{\hat{Q}}}
  \matr{F}^{-1} \label{eq:srelate} \\ \matr{R} & = \matr{S} \sqbrac{\matr{K} +
    \pbrac{\matr{L} \matr{\delby{F}{x}} + \matr{M} \matr{\delby{F}{y}}}
  \matr{F}^{-1}}
\label{eq:rrelate} 
\end{align}
$\matr{K}$, $\matr{L}$ and $\matr{M}$ are diagonal matrices where the
diagonals contain the nodal values of $k$, $l$ and $m$ respectively.
$\vect{n}$ is a vector containing the nodal values of $n$.  

\subsubsection{The DR-BEM Using Other Operators}
\label{sec:otheroperators}

The DR-BEM has been presented in this chapter based on the Laplace
operator.  However the DR-BEM can be be applied using essentially any
operator of appropriate order with known fundamental solution.  If an
appropriate operator can be found the complexity of the forcing function
$b$ can be reduced.  This should improve the accuracy of the method.  The
problem with applying the DR-BEM based on another operator is in
choosing the approximating function $f$.  A choice of $f$ which produces
accurate results is required but it is also necessary to choose an $f$ for
which a particular solution $\hat{u}$ can be determined. 

\citeasnoun{zhu:1993} has determined the particular solutions necessary for
applying the DR-BEM based on the two-dimensional Helmholtz operator.
\begin{equation}
  \laplacian{u} + \kappa^2 u = \fnof{b}{x,y,u,t}
\end{equation}
Radial functions have generally been used when applying the DR-BEM.  Along
the lines of \citeasnoun{wrobel:1986}, \citename{zhu:1993} chose an
approximating function of the form $f = r^{m}$ where $m$ is a positive
integer.  Determining the particular solution $\hat{u}$ requires solving the
ordinary differential equation
\begin{equation}
  \dtwosqby{\hat{u}}{r} + \dfrac{1}{r} \dby{\hat{u}}{r} + \kappa^2 u = r^{m}
\end{equation}
which can be achieved using a variation of coefficients method.

\citeasnoun{partridge:1992} applied the DR-BEM to the transient convection
diffusion equation
\begin{equation}
  D \laplacian{u} - v_{x} \delby{u}{x} - v_{y} \delby{u}{y} - k u = \delby{u}{t}
\end{equation}
where the material parameters $D$, $v_{x}$, $v_{y}$ and $k$ are all assumed to
be homogeneous.  They applied the DR-BEM based on the steady-state
convection-diffusion operator
\begin{equation}
  D \laplacian{u} - v_{x} \delby{u}{x} - v_{y} \delby{u}{y} - k u = 0
\label{eq:steadycd}
\end{equation}
which has a known fundamental solution.

This analysis requires the determination of a particular solution $\hat{u}$
which satisfies
\begin{equation}
  D \laplacian{\hat{u}} - v_{x} \delby{\hat{u}}{x} - v_{y} \delby{\hat{u}}{y}
  - k \hat{u} = f
\label{eq:cdpartic}
\end{equation}
Instead of defining a form of the approximating function $f$ and solving
for $\hat{u}$ \citename{partridge:1992} chose to define $\hat{u}$ and use
\eqnref{eq:cdpartic} to determine the corresponding approximating function.
Although somewhat ad-hoc this approach was found to produce accurate
results.
\index{Dual reciprocity boundary element method@Dual reciprocity BEM|)}

%%% Local Variables: 
%%% mode: latex
%%% TeX-master: t
%%% End: 
