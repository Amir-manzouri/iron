 \chapter{Modal Analysis}

\section{Introduction}

The system of ordinary differential equations which results from the
application of the Galerkin finite element (or other) discretization of the
spatial domain to linear parabolic or hyperbolic equations can either be
integrated directly - as in the last section for parabolic equations - or
analysed by \emph{mode superposition}. That is, the time-dependent solution
is expressed as the superposition of the natural (or resonant) modes of the
system. To find these modes requires the solution of an eigenvalue problem.

\section{Free Vibration Modes}

Consider an extension of \eqnref{eqn:orddiff} which includes second order
time derivatives (\eg nodal point accelerations)
\begin{equation}
  \matr{M}\fnof{\ddot{\vect{u}}}{t} + \matr{C}\fnof{\dot{\vect{u}}}{t}
         + \matr{K}\fnof{\vect{u}}{t} = \fnof{\vect{f}}{t}
  \label{eqn:npa}
\end{equation}
$\matr{M},\matr{C}$ and $\matr{K}$ are the mass, damping and stiffness
matrices, respectively, $\fnof{\vect{f}}{t}$ is the external load vector
and $\fnof{\vect{u}}{t}$ is the vector of $n$ nodal unknowns. In direct
time integration methods $\fnof{\ddot{\vect{u}}}{t}$ and
$\fnof{\dot{\vect{u}}}{t}$ are replaced by finite differences and the
resulting system of algebraic equations is solved at successive time steps.
For a small number of steps this is the most economical method of solution
but, if a solution is required over a long time period, or for a large
number of different load  vectors $\fnof{\vect{f}}{t}$, a suitable transformation
\begin{equation}
  \fnof{\vect{u}}{t}=\matr{P}\fnof{\vect{x}}{t}
  \label{eqn:suittrans}
\end{equation}
applied to \eqnref{eqn:npa} can result in the matrices of the transformed system
\begin{equation}
  \transpose{\matr{P}}\matr{M}\matr{P}\fnof{\ddot{\vect{x}}}{t} +
  \transpose{\matr{P}}\matr{C}\matr{P}\fnof{\dot{\vect{x}}}{t} +
  \transpose{\matr{P}}\matr{K}\matr{P}\fnof{\vect{x}}{t}
        = \transpose{\matr{P}}\vect{f}
  \label{eqn:tsyst}
\end{equation}
having a much smaller bandwidth than in the original system and hence being
more economical to solve. In fact, if damping is neglected, $\matr{P}$ can be
chosen to diagonalize $\matr{M}$ and $\matr{K}$ and thereby uncouple the equations
entirely.  This transformation (which is still applicable when damping is
included but does not then result in an uncoupled system unless further
simplications are made) is found by solving the free vibration problem
\begin{equation}
  \matr{M}\fnof{\ddot{\vect{u}}}{t} + \matr{K}\fnof{\vect{u}}{t} 
        = \vect{0}
  \label{eqn:vibprob}
\end{equation}
\textbf{Proof}: Consider a solution to \eqnref{eqn:vibprob} of the form
\begin{equation}
  \fnof{\vect{u}}{t}=\vect{s} \sin \omega \pbrac{t-t_{0}},
  \label{eqn:sineq}
\end{equation}
where $\omega$ and $t_{0}$ are constants and $\vect{s}$ is a vector of order $n$. 
Substituting \eqnref{eqn:sineq} into \eqnref{eqn:vibprob} gives the
\emph{generalized eigenproblem}
\begin{equation}
  \matr{K}\vect{s}=\omega^{2}\matr{M}\vect{s}
  \label{eqn:gene}
\end{equation}
having $n$ \emph{eigensolutions} $\pbrac{\omega_{1}^{2}\vect{s}_{1}},
\pbrac{\omega_{2}^{2}\vect{s}_{2}},\ldots,\pbrac{\omega_{n}^{2}\vect{s}_{n}}$.
If $\matr{K}$ is a symmetric matrix (as is the case when the original partial
differential operator is self-adjoint) the eigenvectors are orthogonal and can
be ``\emph{normalized}'' such that
\begin{equation}
  \transpose{\vect{s}_{i}}\matr{M}\vect{s}_{j}= \left\{
    \begin{array}{cc}
      1 & i=j\\
      0 & i \neq j
    \end{array} \right.
  \label{eqn:norm}
\end{equation}
(the eigenvectors are said to be $M$-\emph{orthonormalised}). Combining the
$n$ eigenvectors into a matrix $\matr{S} = \sqbrac{\vect{s}_{1}, \vect{s}_{2},\ldots,
\vect{s}_{n}}$ - the \emph{modal matrix} - rewriting \eqnref{eqn:norm} as
\begin{equation}
  \transpose{\matr{S}}\matr{M}\matr{S} = \matr{I}
  \label{eqn:mm}
\end{equation}
where $\matr{I}$ is the identity matrix, \eqref{eqn:gene} becomes
\begin{equation}
  \matr{K}\matr{S} = \matr{M}\matr{S}\matr{\Lambda}
  \label{eqn:id}
\end{equation}
where
\begin{equation}
  \matr{\Lambda} = \begin{bmatrix}
    \omega_{1}^{2} &  &  & 0 \\
    & \omega_{2}^{2} & & \\
    & & \ddots & \\
    0 & & & \omega_{n}^{2}
  \end{bmatrix}
  \label{eqn:lambda}
\end{equation} 
or
\begin{equation}
  \transpose{\matr{S}}\matr{K}\matr{S} =
  \transpose{\matr{S}}\matr{M}\matr{S}\matr{\Lambda} = \matr{I}\matr{\Lambda}
  = \matr{\Lambda} 
  \label{eqn:oreq}
\end{equation}

Thus the modal matrix - whose columns are the $M$-orthonormalised eigenvectors
of $\matr{K}$ (\ie satisfying \eqnref{eqn:gene}) - can be used as the
transformation matrix $\matr{P}$ in \eqnref{eqn:suittrans} required to reduce the
original system of equations \bref{eqn:npa} to the \emph{canonical} form
\begin{equation}
  \fnof{\ddot{\vect{x}}}{t} + 
        \transpose{\matr{S}}\matr{C}\matr{S}\fnof{\dot{\vect{x}}}{t} +
  \matr{\Lambda}\fnof{\vect{x}}{t} 
        = \transpose{\matr{S}}\fnof{\vect{f}}{t}
  \label{eqn:canon}
\end{equation}
With damping neglected equation \eqnref{eqn:canon} becomes a system of
uncoupled equations
\begin{equation}
  \fnof{\ddot{x}_{i}}{t} + \omega_{i}^{2}\fnof{x_{i}}{t} = \fnof{r_{i}}{t}
  \qquad i=1,2,\ldots, n
  \label{eqn:uncoup}
\end{equation}
where $x_{i}$ is the \nth{i} component of $\vect{x}$ and $r_{i}$ is the \nth{i}
component of the vector $\transpose{\matr{S}}\vect{f}$. The solution of this
system is given by the Duhamel integral  
\begin{equation}
  \fnof{x_{i}}{t}=\dfrac{1}{\omega_{i}}\gint{0}{t}
  {\fnof{r_{i}}{\tau}\sin\fnof{\omega_{i}}{t-\tau}}{\tau}
  +\alpha_{i}\sin\omega_{i}t+\beta_{i}\cos\omega_{i} d\tau
 \label{eqn:Duh}
\end{equation}
where the constants $\alpha_{i}$ and $\beta_{i}$ are determined from the
initial conditions
\begin{gather}
  \begin{aligned}
    \evalat{\fnof{x_{i}}{0}}{t=0} &= \transpose{{s}_{i}}\matr{M}
      \evalat{\fnof{\vect{u}}{0}}{t=0} \\
    \evalat{\fnof{\dot{x}_{i}}{0}}{t=0} &= \evalat{\fnof{x_{i}}{0}}{t=0} 
                = \transpose{{s}_{i}}\matr{M}\evalat{\fnof{\vect{u}}{0}}{t=0} 
      \transpose{{s}_{i}}\matr{M}\evalat{\fnof{\vect{u}}{0}}{t=0}
  \label{eqn:incond}
  \end{aligned}
\end{gather}

%
%
% Note :
%
%
%\remark{I think the $\dot{x}_{i}$ expression is wrong}
%
%

\section{An Analytic Example}

As an example, consider the equilibrium equations $\matr{M}\ddot{\vect{u}} +
\matr{K}\vect{u}=\vect{f}$ where
\begin{displaymath}
 \matr{M}=\begin{bmatrix}
   2 & 0 \\
   0 & 1
 \end{bmatrix}, \quad
 \matr{K}=\begin{bmatrix}
   6 & -2 \\
   -2 & 4
 \end{bmatrix}\quad\text{and}\quad
 \vect{f}=\begin{bmatrix}
   0 \\
   10
 \end{bmatrix}
\end{displaymath}
To find the solution by modal analysis we first solve the generalised
eigenproblem $\matr{K}\vect{s} = \omega^{2}\matr{M}\vect{s}$ \ie
\begin{displaymath}
  \begin{bmatrix}
   6-2\omega^{2} & -2 \\
   -2 & 4-\omega^{2}
 \end{bmatrix}\vect{s}=0
\end{displaymath}   
has a solution when $\det \sqbrac{\matr{K}-\omega^{2}\matr{M}} = 0$ or $\omega^{4} -
7\omega^{2} + 10= 0$. This \emph{characteristic polynomial} has solutions
$\omega^{2} = 2,5$ with corresponding eigenvectors $\transpose{\vect{s}}_{1} = a
\begin{bmatrix} 1 & 1 \end{bmatrix},\transpose{\vect{s}}_{1} =
b\begin{bmatrix} -1 & 2 \end{bmatrix}$. To find the magnitude of the
eigenvectors we use \eqnref{eqn:norm}, \ie
\begin{displaymath}
  a^{2}\begin{bmatrix} 1 & 1 \end{bmatrix}
  \begin{bmatrix}
    2 & 0 \\
    0 & 1
  \end{bmatrix} 
  \begin{bmatrix}
     1 \\
     1
  \end{bmatrix}
  = 1 \Rightarrow a=\dfrac{1}{\sqrt{3}} \qquad b^{2}
  \begin{bmatrix} -1 & 2 \end{bmatrix} 
  \begin{bmatrix}
    2 & 0 \\
    0 & 1
  \end{bmatrix} \begin{bmatrix}
    -1 \\
    2
  \end{bmatrix}=1 \Rightarrow b=\dfrac{1}{\sqrt{3}}
\end{displaymath}

(Notice that the orthogonality condition is satisfied: 
$ab\begin{bmatrix} 1 & 1 \end{bmatrix}
\begin{bmatrix}
  2 & 0\\
  0 & 1
\end{bmatrix} 
\begin{bmatrix}
  -1 \\
  2
\end{bmatrix}=0$).

The $M$-orthognormalised eigenvectors are now
$\transpose{\vect{s}}_{1}=\begin{bmatrix} \dfrac{1}{\sqrt{3}} &
  \dfrac{1}{\sqrt{3}}\end{bmatrix}$ and $\transpose{\vect{s}}_{2}=
\begin{bmatrix} -\dfrac{1}{\sqrt{6}} &
  \dfrac{2}{\sqrt{6}}\end{bmatrix}$, giving the modal matrix
$\matr{S}=\begin{bmatrix} 
  \dfrac{1}{\sqrt{3}} & -\dfrac{1}{\sqrt{6}} \\ 
  \dfrac{1}{\sqrt{3}} & \dfrac{2}{\sqrt{6}}
 \end{bmatrix}$
 which, when used as the transformation matrix, reduces the stiffness matrix
 to
\begin{displaymath}
  \transpose{\matr{S}}\matr{K}\matr{S}=\begin{bmatrix}
    \dfrac{1}{\sqrt{3}} & \dfrac{1}{\sqrt{3}}\\
   -\dfrac{1}{\sqrt{6}} & \dfrac{2}{\sqrt{6}}
 \end{bmatrix}\begin{bmatrix}
   6  & -2 \\
   -2 & 4
 \end{bmatrix}\begin{bmatrix}
   \dfrac{1}{\sqrt{3}} & -\dfrac{1}{\sqrt{6}} \\
   \dfrac{1}{\sqrt{3}} & \dfrac{2}{\sqrt{6}}
 \end{bmatrix} = \begin{bmatrix}
   2 & 0 \\
   0 & 5
 \end{bmatrix} = \matr{\Lambda}
\end{displaymath}
and the mass matrix to
\begin{displaymath}
  \transpose{\matr{S}}\matr{M}\matr{S}=\begin{bmatrix}
    \dfrac{1}{\sqrt{3}} & \dfrac{1}{\sqrt{3}} \\
    -\dfrac{1}{\sqrt{6}} & \dfrac{2}{\sqrt{6}}
 \end{bmatrix}\begin{bmatrix}
   2 & 0 \\
   0 & 1
 \end{bmatrix}\begin{bmatrix}
   \dfrac{1}{\sqrt{3}} & -\dfrac{1}{\sqrt{6}}\\
   \dfrac{1}{\sqrt{3}} & \dfrac{2}{\sqrt{6}}
 \end{bmatrix} = \begin{bmatrix}
   1 & 0\\
   0 & 1
 \end{bmatrix} = \matr{I}
\end{displaymath}
Thus the natural modes of the system are given by
\begin{displaymath}
 \fnof{\vect{u}_{1}}{t} = \begin{bmatrix}
   \dfrac{1}{\sqrt{3}}\\
   \dfrac{1}{\sqrt{3}}
 \end{bmatrix}
 \sin \sqrt{2}\pbrac{t-t_{0}}\quad \text{and} \quad
 \fnof{\vect{u}_{2}}{t} = \begin{bmatrix}
   -\dfrac{1}{\sqrt{6}} \\
   \dfrac{2}{\sqrt{6}}
 \end{bmatrix} \sin \sqrt{5}\pbrac{t-t'_{0}}.
\end{displaymath}      
The solution of the non-homogeneous system, subject to given initial
conditions, is found by solving the uncoupled equations
\begin{displaymath}
 \fnof{\ddot{\vect{x}}}{t} + \begin{bmatrix}
   2 & 0 \\
   0 & 5
 \end{bmatrix} \fnof{\vect{x}}{t} = \begin{bmatrix}
   \dfrac{1}{\sqrt{3}} & \dfrac{1}{\sqrt{3}} \\
   -\dfrac{1}{\sqrt{6}} & \dfrac{2}{\sqrt{6}}
 \end{bmatrix}\begin{bmatrix}
   0 \\
   10
 \end{bmatrix} = \begin{bmatrix}
   \dfrac{10}{\sqrt{3}}\\
   \dfrac{20}{\sqrt{6}}
 \end{bmatrix} 
\end{displaymath}
by means of the Duhamel integral \bref{eqn:Duh} (in this case with $\vect{r}$
constant) and then, from \eqnref{eqn:suittrans} with
$\matr{P} \equiv \matr{S} = [(\vect{s}_{1}, \vect{s}_{2}, \ldots ,\vect{s}_{n}]$
\begin{equation}
  \fnof{\vect{u}}{t} = \matr{S}\fnof{\vect{x}}{t} = \dsuml{i=1}{n}
  \vect{s}_{i}\fnof{x_{i}}{t} 
  \label{eqn:Duhint}
\end{equation}

Notice that the solution is expressed in \eqnref{eqn:Duhint} as the superposition
of the natural modes (eigenvectors) of the homogeneous equations. If the
forcing function (load vector) is close to one of these modes the
corresponding coefficient $x_{i}$ will be large and will dominate the response
- if it coincides then resonance will occur. Very often it is unnecessary to
evaluate all $n$ eigenvectors of the system; the higher frequency modes can be
ignored and the solution adequately represented by superposition of the $p$
eigenvectors associated with the $p$ lowest eigenvalues, where $p<n$.

\section{Proportional Damping}

When element damping terms are included in the original dynamic equations
\bref{eqn:npa} the transformation to a lower bandwidth system is still based
on the model matrix $\matr{S}$ but \eqnref{eqn:canon} is then not a system of
uncoupled equations. One simplification often made in order to retain the
diagonal nature of \eqnref{eqn:canon} is to approximate the overall energy
dissipation of the finite element system with \emph{proportional damping}

\begin{equation}
  \transpose{\vect{{s}_{i}}}\matr{C}\vect{s}_{j} = 2\omega_{i}\xi_{i}\delta_{ij},
 \label{eqn:propd}
\end{equation}
where $\xi_{i}$ is a modal damping parameter and $\delta_{ij}$ is the
Kronecker delta. \Eqnref{eqn:canon} now reduces to $n$ equations of the
form
\begin{equation}
  \fnof{\ddot{x}_{i}}{t} + 2 \omega_{i}\xi_{i}\fnof{\dot{x}_{i}}{t} +
  \omega_{1}^{2}\fnof{x_{1}}{t} = \fnof{r_{i}}{t}
 \label{eqn:form}
\end{equation}
with solution (the Duhamel integral)
\begin{equation}
  \fnof{x_{i}}{t}=\dfrac{1}{\overline{\omega}_{i}}\gint{0}{t}
  {\fnof{r_{i}}{\tau}.e^{\xi_{i}\omega_{i}\pbrac{t-\tau}}.
  \sin\fnof{\overline{\omega}_{i}}{t-\tau}}{t}
  +e^{-\xi_{i}\omega_{i}t}\bbrac{\alpha_{i}\sin\overline{\omega}_{i}t
    +\beta_{i}\cos\overline{\omega}_{i}t}
 \label{eqn:calc}
\end{equation}
where $\overline{\omega}_{i}=\omega_{i}\sqrt{1-\xi_{i}^{2}}$. $\alpha_{i}$ and
$\beta_{i}$ are calculated from the initial conditions \eqnref{eqn:incond}.
Once the components $\fnof{x_{i}}{t}$ have been found from \eqnref{eqn:calc} (or
alternative time integration methods applied to \eqref{eqn:form}), the
solution $\fnof{\vect{u}}{t}$ is expressed as a superposition of the mode shapes 
$\vect{s}_{i}$ by \eqnref{eqn:Duhint}.

\section{CMISS Examples}

%\begin{tabular}{ll}
%(a) & Vibration modes of a rectangular plate\\
%(b) & Vibration modes of a circular plate\\
%(c) & Vibration modes of a guitar top plate
%\end{tabular}
% LC no longer exist ? 3/3/97

\begin{enumerate}
  \item To analyse a plane stress modal analysis run CMISS example 451
  \item To analyse a clamped beam modal analysis run CMISS example 452
  \item To analyse a steel-framed building modal analysis run CMISS example 453
\end{enumerate}


%%% Local Variables: 
%%% mode: latex
%%% TeX-master: "~/documents/notes/fembemnotes/fembemnotes"
%%% End: 
