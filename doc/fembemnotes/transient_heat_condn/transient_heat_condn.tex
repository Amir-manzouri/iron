\chapter{Transient Heat Conduction}

\section{Introduction}

In the previous discussion of steady state boundary value problems the
principal advantage of the finite element method over the finite difference
method has been the greater ease with which complex boundary shapes can be
modelled. In time-dependent problems the solution proceeds from an initial
solution at $t = 0$ and it is almost always convenient to calculate each new
solution at a constant time ($t>0$) throughout the entire spatial domain
$\Omega$.  There is, therefore, no need to use the greater flexibility (and
cost) of finite elements to subdivide the time domain: finite difference
approximations of the time derivatives are usually preferred.  Finite
difference techniques are introduced in \Secref{sec:finite} to solve the
transient one dimensional heat equation. A combination of finite elements for
the spatial domain and finite differences for the time domain is used in
\Secref{sec:transient} to solve the transient advection-diffusion
equation - a slight generalization of the heat equation.

%%
%% Finite Differences
%%
\section{Finite Differences}
\label{sec:finite} 

%%
%% Explicit Transient Finite Differences
%%
\subsection{Explicit Transient Finite Differences}

Consider the transient one-dimensional heat equation 
\begin{equation}
  \delby{u}{t} = D\deltwosqby{u}{x}, \qquad \pbrac{0<x<L, t>0}
  \label{eqn:1DHT}
\end{equation}
where $D$ is the conductivity and $u=\fnof{u}{x,t}$ is the temperature,
subject to the boundary conditions $\fnof{u}{0,t} = u_{0}$ and $\fnof{u}{L,t}
= u_{1}$ and the initial conditions $\fnof{u}{x,0} = 0$. A finite difference
approximation of this equation is obtained by defining a grid with spacing
$\Delta x$ in the x-domain and $\Delta t$ in the time domain, as shown in
\Figref{fig:findiff}.

Grid points are labelled by the indices $i=0,1,\ldots,I$ (for the
$x$-direction) and $n=0,1,\ldots,N$ (for the $t$-direction). The temperature at the
grid point $\pbrac{i,n}$ is therefore labelled as
\begin{equation}
  \fnof{u}{x,t}=\fnof{u}{i\Delta x, n\Delta t}=u_{i}^{n}.
  \label{eqn:temp}
\end{equation}
Finite difference equations are derived by writing Taylor Series expansions
for $u_{i+1}^{n},u_{i-1}^{n}u_{i}^{n+1}$ about the grid point $\pbrac{i,n}$
\begin{align}
  u_{i+1}^{n}=u_{i}^{n} &+ \Delta x.\pbrac{\delby{u}{x}}_{i}^{n}
  +\dfrac{1}{2} \Delta x^{2}.\pbrac{\deltwosqby{u}{x}}_{i}^{n}
  +\frac{1}{6}\Delta x^{3}.\pbrac{\frac{\del^{3}u}{\del x^{3}}
  }_{i}^{n}+\orderof{\Delta x^{4}} \label{eqn:Taylor1} \\ 
  u_{i-1}^{n}=u_{i}^{n} &- \Delta x.\pbrac{\delby{u}{x}}_{i}^{n}
  +\dfrac{1}{2} \Delta x^{2}.\pbrac{\deltwosqby{u}{x}}_{i}^{n}
  -\frac{1}{6}\Delta x^{3}.\pbrac{\frac{\del^{3}u}{\del x^{3}}
  }_{i}^{n}+\orderof{\Delta x^{4}} \label{eqn:Taylor2} \\
  u_{i}^{n+1}=u_{i}^{n} &+ \Delta t.\pbrac{\delby{u}{t}}_{i}^{n}
  +\orderof{\Delta t^{2}} \label{eqn:Taylor3}
\end{align}
where $\orderof{\Delta x^{4}}$ and $\orderof{\Delta t^{2}}$ represent all the
remaining terms in the Taylor Series expansions.

Adding \eqnrefs{eqn:Taylor1}{eqn:Taylor2} gives
\begin{equation*}
  u_{i+1}^{n}+u_{i-1}^{n} = 2u_{i}^{n} + \Delta x^{2}.\pbrac{
    \deltwosqby{u}{x}}_{i}^{n}+\orderof{\Delta x^{4}}
\end{equation*}  
or
\begin{equation}
  \pbrac{\deltwosqby{u}{x}}_{i}^{n} = \dfrac{u_{i+1}^{n}
    -2u_{i}^{n}+u_{i-1}^{n}}{\Delta x^{2}} + \orderof{\Delta x^{2}},
  \label{eqn:expns}
\end{equation} 
which is a ``central difference'' approximation of the second order spatial
derivative.

Rearranging \eqnref{eqn:Taylor3} gives a ``difference'' approximation of the
first order time derivative
\begin{equation}
  \pbrac{\delby{u}{t}}_{i}^{n} = \dfrac{u_{i}^{n+1} -u_{i}^{n}}{\Delta t}
  + \orderof{\Delta t}.
  \label{eqn:fordiff}
\end{equation} 
 
Substituting \eqnref{eqn:expns} and \eqnref{eqn:fordiff} into the transient heat
equation \eqnref{eqn:1DHT} gives the finite difference approximation
\begin{equation*}
  \dfrac{u_{i}^{n+1}-u_{i}^{n}}{\Delta t} + \orderof{\Delta t}
   = D\dfrac{u_{i+1}^{n}-2u_{i}^{n}+u_{i-1}^{n}}{\Delta x^{2}} 
  + \orderof{\Delta x^{2}}
\end{equation*}
which is rearranged to give an expression for $u_{i}^{n+1}$ in terms of the
values of $u$ at the $\nth{n}$ time step
\begin{equation}
  u_{i}^{n+1}  =  u_{i}^{n} + D \dfrac{\Delta t}{\Delta x^{2}}\pbrac{
    u_{i+1}^{n} -2u_{i}^{n} +u_{i-1}^{n}} 
  + \orderof{\Delta t^{2}, \Delta x^{2}}.
  \label{eqn:rearr}
\end{equation}
 
\begin{figure} \centering
 \input{figs/transient_heat_condn/findiff.pstex}
 \caption{A finite difference grid for the solution of the transient 1D 
   heat equation. The equation is centred at grid point $\pbrac{i,n}$ shown by the
   $\mathbf{O}$. The lightly shaded region shows where the solution is known at
   time step $n$. With central differences in $x$ and a forward difference in
   $t$ an explicit finite difference formula gives the solution at time step
   $n+1$ explicitly in terms of the solution at the three points below it at
   step $n$, as indicated by the dark shading.}
 \label{fig:findiff}
\end{figure}

Given the initial values of $u_{i}^{n}$ at $n=0$ (\ie $t=0$), the values of
$u_{i}^{n+1}$ for the next time step are found from \eqnref{eqn:rearr} with
$i=1,2,\ldots,I$. Applying \eqnref{eqn:rearr} iteratively for time steps
$n=1,2,\ldots$ \etc yields the time dependent temperatures at the grid points
(see \Figref{fig:findiff}). This is an \emph{ explicit} finite difference
formula because the value of $u_{i}^{n}$ depends only on the values of
$u_{i}^{n} \pbrac{i=1,2,\ldots,I}$ at the previous time step and not on the
neighbouring terms $u_{i+1}^{n+1}$ and $u_{i-1}^{n+1}$ at the latest time
step. The accuracy of the solution depends on the chosen values of $\Delta x$
and $\Delta t$ and in fact the stability of the scheme depends on these
satisfying the \emph{Courant} condition:
\begin{equation}
  D\dfrac{\Delta t}{\Delta x^{2}} \leq \frac{1}{2}.
  \label{eqn:Courant}
\end{equation}

%%
%% Von Neumann Stability Analysis
%%
\subsection{Von Neumann Stability Analysis}

The concept behind the Von Neumann analysis is that all Fourier
components decay as time advances or as they are processed by an iterative
solver. Considering \eqnref{eqn:rearr}, we can rearrange this to be of the
form,
\begin{equation}
  u_i^{n+1} = \Upsilon u_{i+1}^n  + (1-2\Upsilon)u_i^n + \Upsilon u_{i-1}^n
\end{equation}
where $\Upsilon=D\dfrac{\Delta t}{\Delta x^2}$. By subsituting the general Fourier
component $u_j^n = A_k^n e^{i \pbrac{\frac{\pi kj\Delta x}{L}}}$, we obtain,
\begin{equation}
  A_k^{n+1}e^{i \pbrac{\frac{\pi kj \Delta x}{L}}} = 
  A_k^n \sqbrac{\Upsilon e^{i \pbrac{\frac{\pi k(j+1)\Delta x}{L}}}
   + \pbrac{1-2\Upsilon} e^{i \pbrac{\frac{\pi kj\Delta x}{L}}}
    + \Upsilon e^{i \pbrac{\frac{\pi k(j-1) \Delta x}{L}}} }
  \label{eqn:genfourier}
\end{equation}
%%
If divide \eqnref{eqn:genfourier} by, 
$A_k^n e^{i \pbrac{\frac{\pi kj \Delta x}{L}}}$ we obtain (no sum on $k$),
%%
\begin{equation}
\begin{split}
 \dfrac{A_k^{n+1}}{A_k^n} 
   &= \pbrac{1-2\Upsilon} + \Upsilon e^{i \pbrac{\frac{\pi k \Delta x}{L}}} 
      + \Upsilon e^{-i \pbrac{\frac{\pi k \Delta x}{L}}} \\
   &= 1-2\Upsilon + 2\Upsilon cos\pbrac{\frac{\pi k \Delta x}{L}} \qquad
   \text{using} \qquad cos(x)=\frac{e^{ix}+e^{-ix}}{2} \\
   &= 1- 4\Upsilon sin^2\pbrac{\frac{\pi k \Delta x}{2L}} \qquad
   \text{using} \qquad cos (2x) = 1 - 2 sin^{2} (x)
\end{split}
\label{eqn:stability}
\end{equation}

\Eqnref{eqn:stability} predicts the growth of any component (specified by $k$)
admitted by the system. If all components are to decay,
\begin{equation}
  \abs{\dfrac{A_k^{n+1}}{A_k^n}} \leq 1 \qquad \text{for stability (no sum on $k$)}
\end{equation}
Since the $sin^2$ term in \eqnref{eqn:stability} is always between $0$ and $1$,
we effectively have the stablity criteria that
\begin{equation}
  1 - 4 \Upsilon \leq 1 \qquad \text{and} \qquad 1 - 4 \Upsilon \geq -1
\end{equation}
The first inequality is trivially satisfied, since $\Upsilon \geq 0$ for
positive values of $\Delta t$ and $D$, and the second condition will always
hold if
\begin{equation}
   \Upsilon = D\dfrac{\Delta t}{\Delta x^2} \leq \dfrac{1}{2}
\end{equation}
Thus, to ensure stability, the time step should be chosen such that
\begin{equation}
   \Delta t \leq \dfrac{\Delta x^2}{2D} \qquad \text{The \emph{Courant}
   condition}
\end{equation}


%%
%% Higher Order Approximations
%%
\subsection{Higher Order Approximations}

An improvement in accuracy and stability can be obtained by using a higher
order approximation for the time derivative. For example, if a central
difference approximation is used for $\delby{u}{t}$ by centering the equation
at $(i \Delta x,\pbrac{n+\frac{1}{2}} \Delta t)$ rather than $\pbrac{i\Delta
  x,n\Delta t}$ we get
\begin{equation}
  \pbrac{\delby{u}{t}}_{i}^{n+\frac{1}{2}} = \dfrac{u_{i}^{n+1}
    -u_{i}^{n}}{\Delta t} + \orderof{\Delta t^{2}}
  \label{eqn:centering}
\end{equation}
in place of \eqnref{eqn:fordiff} and \eqnref{eqn:1DHT} is approximated with the
``Crank-Nicolson''formula
\begin{equation}
  \dfrac{u_{i}^{n+1}-u_{i}^{n}}{\Delta t}=D\bbrac{\dfrac{1}{2}\pbrac{
      \deltwosqby{u}{x}}_{i}^{n+1} + \dfrac{1}{2}\pbrac{\deltwosqby{u}{x}}_{i}^{n}}
  \label{eqn:C-N}
\end{equation} 
in which the spatial second derivative term is weighted by $\frac{1}{2}$ at the old
time step $n$ and by $\frac{1}{2}$ at the new time step $n+1$. Notice that the
finite difference time derivative has not changed - only the time position at
which it is centred. The price paid for the better accuracy (for a given
$\Delta t$) and unconditional stability (\ie stable for \textbf{any}
$\Delta t$) is that \eqnref{eqn:C-N} is an \emph{implicit} scheme - the
equations for the new time step are now coupled in that $u_{i}^{n+1}$ depends
on the neighbouring terms $u_{i+1}^{n+1}$ and $u_{i-1}^{n+1}$. Thus each new
time step requires the solution of a system of coupled equations.
\begin{figure} \centering
 \input{figs/transient_heat_condn/impfd.pstex}
 \caption{An implicit finite difference scheme based
   on central differences in $t$, as well as $x$, which tie together the 6
   points shown by $\mathbf{x}$. The equation is centred at the point
   ($i,n+\dfrac{1}{2}$) shown by the $\mathbf{O}$. The lightly shaded region shows
   where the solution is known at time step $n$. The dark shading shows the
   region of the coupled equations.}
 \label{fig:impfd}
\end{figure}

%
%
% NOTE:
%
%\remark{In CMISS this is implemented as $\pbrac{1-\theta}
%    \pbrac{\deltwosqby{u}{x}_{i}}^{n+1} + 
%    \theta\pbrac{\deltwosqby{u}{x}_{i}}^{n}$}\\
%
%

A generalization of \bref{eqn:C-N} is
\begin{equation}
  \frac{u_{i}^{n+1}-u_{i}^{n}}{\Delta t} =D\bbrac{\theta
    \pbrac{\deltwosqby{u}{x}_{i}}^{n+1} + \pbrac{1-\theta}
    \pbrac{\deltwosqby{u}{x}_{i}}^{n}}
  \label{eqn:C-N2}
\end{equation}
in which the spatial second derivative of \eqnref{eqn:1DHT} has been weighted
by $\theta$ at the new time step and by $\pbrac{1-\theta}$ at the old time step. The
original explicit forward difference scheme \eqnref{eqn:rearr} is recovered
when $\theta =0$ and the implicit central difference (Crank-Nicolson) scheme
\bref{eqn:C-N2} when $\theta=\frac{1}{2}$. An implicit backward difference
scheme is obtained when $\theta =1$.

In the following section the transient heat equation is approximated for
numerical analysis by using finite differences in time and finite elements in
space. We also generalize the partial differential equation to include an
advection term and a source term.


\section{The Transient Advection-Diffusion Equation}
\label{sec:transient}

Consider a linear parabolic equation
\begin{equation}
  \delby{u}{t} +\dotprod{\vect{v}}{\grad u} = D\laplacian{u} + f
  \label{eqn:lpe}
\end{equation}
where $u$ is a scalar variable (\eg the advection-diffusion
equation\index{Advection-diffusion equation}, where $u$ is concentration or
temperature; $\dotprod{\vect{v}}{\grad u}$ then represents advective transport
by a velocity field $\vect{v}, D$ is the diffusivity and $f$ is source term.
The ratio of advective to diffusive transport is characterised by the
\emph{Peclet number} $VL/D$ where $V = \norm{\vect{v}}$ and $L$ is a
characteristic length).

Applying the Galerkin weighted residual method to \eqnref{eqn:lpe} with
weight \vect{\omega} gives
\begin{displaymath}
  \goneint{\pbrac{\delby{u}{t} + \dotprod{\vect{v}}{\grad u} 
    - D\laplacian{u}-f}\omega}{\Omega} =0
  \label{eqn:Gal}
\end{displaymath}
or
\begin{equation}
  \goneint{\sqbrac{\pbrac{\delby{u}{t} + \dotprod{\vect{v}}{\grad u}} 
    \omega + D\dotprod{\grad u}{\grad \omega}}}{\Omega}= \goneint{f\omega}{\Omega}
  + D \gint{\del \Omega}{}{\delby{u}{n}\omega}{\Gamma}
  \label{eqn:Gal2}
\end{equation}
where $\delby{ }{n}$is the normal derivative to the boundary $\del \Omega$.

Putting $u=\lbfnsymb{n}u_{n}$ and  $\omega=\lbfnsymb{m}$ and summing
the element contributions to the global equations, \eqnref{eqn:Gal2} can be 
represented by a system of first order ordinary differential equations,
\begin{equation}
  \matr{M}\dby{\vect{u}}{t} + \matr{K}\vect{u} = \matr{K}\vect{u}_{\infty}
  \label{eqn:orddiff}
\end{equation}
where $\matr{M}$ is the global mass matrix, $\matr{K}$ the global stiffness matrix
and $\vect{u}$ a vector of global nodal unknowns with steady state values ($t
\rightarrow\infty$) $\vect{u}_{\infty}$ . The element contributions to $\matr{M}$
and $\matr{K}$ are given by
\begin{equation}
  M_{{mn}_{e}} = \gint{0}{1}{\lbfnsymb{m} \lbfnsymb{n} J}{\xi}
 \label{eqn:elemcont}
\end{equation}
and
\begin{equation}
  K_{{mn}_{e}} = \gint{0}{1}{D\delby{\lbfnsymb{m}}{\xi_{i}}
  \delby{\lbfnsymb{n}}{\xi_{j}}\cdot \delby{\xi_{i}}{x_{k}} 
  \delby{\xi_{j}}{x_{k}}J}{\xi} + \gint{0}{1}{v_{j}\lbfnsymb{m}
  \delby{\lbfnsymb{n}}{\xi_{i}}\delby{\xi_{i}}{x_{j}}J}{\xi}
  \label{eqn:elemcont2}
\end{equation}

If the time domain is now discretized $\pbrac{t = n \Delta t, n = 0,1,2,\ldots}$
\eqnref{eqn:orddiff} can be replaced by
\begin{equation}
  \matr{M}\dfrac{\vect{u}^{n+1}-\vect{u}^{n}}{\Delta t} +
        \matr{K}\sqbrac{\theta \vect{u}^{n+1} + \pbrac{1-\theta}\vect{u}^{n}} =
        \matr{K}\vect{u}_{\infty} \qquad 0 \leq \theta \leq 1
  \label{eqn:elemcont3}
\end{equation}
where $\theta$ is a weighting factor discussed in \Secref{sec:finite}.
Note that for $\theta = \dfrac{1}{2}$ the method is known as the
\emph{Crank-Nicolson-Galerkin} method and errors arising from the time domain
discretization are $\orderof{\Delta t^{2}}$. Rearranging \eqnref{eqn:elemcont3}
as
\begin{eqnarray}
  \sqbrac{\matr{M} + \theta \Delta t \matr{K}}\vect{u}^{n+1}
  =\sqbrac{\matr{M}-\pbrac{1-\theta}\Delta t\matr{K}}\vect{u}^{n} + \Delta t 
  \matr{K}\vect{u}_{\infty}
  \label{eqn:elemcont4}
\end{eqnarray}
gives a set of linear algebraic equations to solve at the new time step
$\pbrac{n+1}\Delta t$ from the known solution $\vect{u}^{n}$ at the previous time
step $n\Delta t$.

The stability of the above scheme can be examined by expanding $\vect{u}$
(assumed to be smoothly continuous in time) in terms of the eigenvectors
$\vect{s}_{i}$ (with associated eigenvalues $\lambda_{i}$) of the matrix $\matr{A}
= \matr{M}^{-1}\matr{K}$. Writing the initial conditions $\vect{u}(0) = \dsuml{i}{}
a_{i}\vect{s}_{i}$ and steady state solution $\vect{u}_{\infty} = \dsuml{i}{}
b_{i}\vect{s}_{i}$ , the set of ordinary differential equations
\eqnref{eqn:orddiff} has solution
\begin{equation}
  \vect{u} = \dsuml{i}{} \sqbrac{b_{i} + \pbrac{a_{i}- b_{i}}e^{-\lambda_{i}t}}
  \vect{s}_{i}
\label{eqn:ordde}
\end{equation}

The time-difference scheme \eqnref{eqn:elemcont4} on the other hand, with
$\vect{u}$ now replaced by a set of discrete values $\vect{u}^{n}$ at each time
step $n\Delta t$, can be written as the recursion formula
\begin{equation}
  \sqbrac{\matr{I}+\theta \Delta t\matr{A}}\vect{u}^{n+1} = \sqbrac{\matr{I}-\pbrac{1-\theta} 
    \Delta t\matr{A}}\vect{u}^{n} +\Delta t\matr{A} \vect{u}_{\infty}
  \label{eqn:recform}
\end{equation}
with solution
\begin{equation}
  \vect{u}=\dsuml{i}{} \bbrac{b_{i} + \pbrac{a_{i}- b_{i}}\sqbrac{
      \dfrac{1 - \Delta t\pbrac{1-\theta}\lambda_{i}}{1 + \Delta t \theta 
        \lambda_{i}}}^{n}}\vect{s}_{i}
  \label{eqn:wsol}
\end{equation}
(You can verify that \eqnref{eqn:ordde} and \eqnref{eqn:wsol} are indeed the
solutions of \eqnref{eqn:orddiff} and \eqnref{eqn:elemcont3}, respectively, by
substituting and using $\matr{A}\vect{s}_{i} = \lambda_{i}\vect{s}_{i}$.)

Comparing \eqnref{eqn:ordde} and \eqnref{eqn:wsol} shows that replacing the
ordinary differential equations \bref{eqn:orddiff} by the finite difference
approximation \eqnref{eqn:elemcont3} is equivalent to replacing the exponential
$e^{-\lambda_{i}t}$ in \eqnref{eqn:ordde} by the approximation
\begin{equation}
  e^{-\lambda_{i}t} \sim \sqbrac{\dfrac{1- \Delta t \pbrac{1-\theta}\lambda_{i}}
    {1 + \Delta t \theta \lambda_{i}}}^{n}
  \label{eqn:approx}
\end{equation}
or, with $t = n\Delta t$,
\begin{equation}
  e^{-\lambda_{i}t} \sim \dfrac{1 - \Delta t\pbrac{1-\theta}\lambda_{i}}{
    1 + \Delta t\theta \lambda_{i}}=1-\dfrac{\Delta t \lambda_{i}}
  {1+\Delta t \theta \lambda_{i}}
  \label{eqn:approx2}
\end{equation}

The stability of the numerical time integration scheme can now be investigated
by examining the behaviour of this approximation to the exponential. For
stability we require
\begin{equation}
  -1 \leq 1 - \dfrac{\Delta t \lambda_{i}}{1 + \Delta t \theta 
    \lambda_{i}} \leq 1
  \label{eqn:stab}
\end{equation}
since this term appears in \eqnref{eqn:wsol} raised to the power $n$. The right
hand inequality in \eqnref{eqn:stab} is trivially satisfied, since $\Delta
t,\lambda_{i}$ and $\theta $ are all positive, and the left hand inequality
gives
\begin{equation}
  \dfrac{\Delta t \lambda_{i}}{1 + \Delta t \theta \lambda_{i}} \leq 2 \qquad 
  \text{or} \quad \Delta t \lambda_{i} \pbrac{1-2\theta} \leq 2
\label{eqn:LHineq}
\end{equation}

A consequence of \eqnref{eqn:LHineq} is that the scheme is \emph{unconditionally
  stable} if $\frac{1}{2} \leq \theta \leq 1$. For $\theta < \frac{1}{2}$ the
\emph{stability criterion} is
\begin{equation}
  \Delta t \lambda_{i}< \dfrac{2}{1-2\theta}
  \label{eqn:scrit}
\end{equation}
   
If the exponential approximation given by \eqnref{eqn:approx2} is negative for
any $\lambda_{i}$ the solution will contain components which change sign with
each time step $n$. This \emph{oscillatory noise} can be avoided by choosing
\begin{equation}
  \Delta t < \dfrac{1}{\pbrac{1-\theta}\lambda_{\text{max}}},
  \label{eqn:oscnoise}
\end{equation}
where $\lambda_{\text{max}}$ is the largest eigenvalue in the matrix $\matr{A}$,
but in practice this imposes a limit which is too severe for $\Delta t$ and a small
amount of oscillatory noise, associated with the high frequency vibration modes
of the system, is tolerated.  Alternatively the oscillatory noise can be
filtered out by averaging.

These theoretical results are explored numerically with a
Crank-Nicolson-Galerkin scheme ($\theta= \frac{1}{2}$) in \Figref{fig:annumsol},
where the one-dimensional diffusion equation
\begin{equation}
  \begin{array}{rcl}
    \delby{u}{t}= D  \deltwosqby{u}{x}   & \text{on} &0 \leq x \leq 1 \\\\
    \mbox{subject to initial conditions} && \fnof{u}{x,0} = 0 \\
    \mbox{and boundary conditions} && \fnof{u}{0,t} = 0 , \fnof{u}{1,t} = 1 
  \end{array}
  \label{1DDE}
\end{equation}
is solved for various time increments ($\Delta t$) and element lengths
($\Delta x$) for both linear and cubic Hermite elements.
\begin{figure} \centering
 \input{figs/transient_heat_condn/annumsol.pstex}
 \caption{Analytical and numerical solutions of the
   transient 1D heat equation showing the effects of element size $\Delta x$
   and time step size $\Delta t$. The top graph shows the exact and
   approximate solutions as functions of $x$ at various times. The lower
   graphs show the solution through time at the specified $x$ positions and
   with various choices of $\Delta x$ and $\Delta t$ as indicated.}
 \label{fig:annumsol}
\end{figure}

Decreasing $\Delta x$ from $0.25$ to $0.1$ with linear elements produces more
oscillation because the system has more degrees of freedom and leads to greater
oscillation. At a sufficiently small $\Delta t$ the oscillations are
negligible (bottom right, \Figref{fig:annumsol}). With this value of $\Delta
t$ (\nunit{0.01}{\s}) the numerical results agree well with the exact solution
(top, \Figref{fig:annumsol}) given by
\begin{equation}
  \fnof{u}{x,t} = x + \dfrac{2}{\pi}\dsuml{n=1}{\infty}\dfrac{(-1)^{n}}{n}
  e^{-n^{2}\pi^{2}t} \sin\pbrac{n\pi x}
  \label{eqn:exact}
\end{equation}

\section{Mass lumping}
\label{sec:mass}

A technique known as \emph{mass lumping}\index{Mass lumping} is sometimes used
in which the mass matrix $\matr{M}$ is replaced by a diagonal matrix having
diagonal terms equal to the row sums. For example, consider the mass matrix
(\eqnref{eqn:elemcont}) for a bilinear element (see \Figref{fig:2Dbbf} and
\eqnref{eqn:2,3DE}).
\begin{alignat*}{2}
    M_{11} &= \iint{}{}{\pbrac{1-\xi_{1}}^{2}
        \pbrac{1-\xi_{2}}^{2}}{\xi_{1}}{\xi_{2}} &&
        = \evalat{-\dfrac{\pbrac{1-\xi_{1}}^{3}}{3}}{0}{1}
        \evalat{-\dfrac{\pbrac{1-\xi_{2}}^{3}}{3}}{0}{1}=\dfrac{1}{3}\cdot
        \dfrac{1}{3} = \frac{1}{9} \\ 
    M_{22} &=
      \iint{}{}{\xi_{1}^{2}\pbrac{1-\xi_{2}}^{2}}
      {\xi_{1}}{\xi_{2}} &&=
      \dfrac{1}{3}\cdot\dfrac{1}{3} = \dfrac{1}{9} 
      \text{ and similarly $M_{33}$
      and $M_{44}$.} \\ 
    M_{12} &= \iint{}{}{\xi_{1}\pbrac{1-\xi_{1}}
      \pbrac{1-\xi_{1}}^{2}}{\xi_{1}}{\xi_{2}} &&= \pbrac{-\dfrac{1}{2}
      -\dfrac{1}{3}}\cdot\dfrac{1}{3} = \dfrac{1}{18} \\ 
    M_{13} &= \iint{}{}{\pbrac{1-\xi_{1}}^{2}
        \xi_{2}\pbrac{1-\xi_{2}}}{\xi_{1}}{\xi_{2}} &&= \dfrac{1}{18} \text{
        and similarly $M_{34}$ and $M_{24}$.} \\ 
    M_{14} &= \iint{}{}{\xi_{1}\pbrac{1-\xi_{1}}
        \xi_{2}\pbrac{1-\xi_{2}}}{\xi_{1}}{\xi_{2}} 
        &&= \dfrac{1}{36} \text{ and similarly $M_{23}$.}
\end{alignat*}

\begin{displaymath}
  \text{therefore } \matr{M}=\begin{bmatrix}
    \frac{1}{9} & \frac{1}{18} & \frac{1}{18} & \frac{1}{36} \\
    \frac{1}{18} & \frac{1}{9} & \frac{1}{36} & \frac{1}{18} \\
    \frac{1}{18} & \frac{1}{136} & \frac{1}{9} & \frac{1}{18} \\
    \frac{1}{36} & \frac{1}{18} & \frac{1}{18} & \frac{1}{9} \\
  \end{bmatrix}
  \stackrel{\text{mass lumping}}{\longrightarrow} \begin{bmatrix}
    \frac{1}{4} & 0 & 0 & 0 \\
    0 & \frac{1}{4} & 0 & 0 \\
    0 & 0 & \frac{1}{4} & 0 \\
    0 & 0 & 0 & \frac{1}{4}
  \end{bmatrix}
\end{displaymath}

The element mass is effectively lumped at the element vertices.  Such a scheme
has computational advantages when $\theta = 0$ in \eqnref{eqn:elemcont4}
because each component of the vector $\vect{u}^{n+1}$ is obtained directly
without the need to solve a set of coupled equations. This \emph{explicit}
time integration scheme, however, is only conditionally stable (see
\bref{eqn:scrit}) and suffers from \emph{phase lag errors} - see below. For
evenly spaced elements the finite element scheme with mass lumping is
equivalent to finite differences with central spatial differences.

In \Figref{fig:advection}, the finite element and finite differences (lumped
f.e. mass matrix) solutions of the one-dimensional advection-diffusion
equation \bref{eqn:lpe} with $V =$ \nunit{5}{\mps}, $D = 0.1$
\units{m^{2}s^{-1}}, $f = 0$ are compared for the propogation and dispersion
of an initial unit mass pulse at $x = 0$. The length of the solution domain is
sufficient to avoid reflected end effects.
\begin{figure} \centering
 \input{figs/transient_heat_condn/advection.pstex}
 \caption{Advection-diffusion of a unit mass pulse. The finite element
   solutions (at $t$=\nunit{0.01}{\s}, \nunit{0.05}{\s}, \nunit{0.2}{\s} and
   \nunit{0.4}{\s}) and finite difference solutions (at $t$=\nunit{0.4}{\s}
   only) are compared with the exact solution.  $\Delta x$= 0.1, $\Delta t$ =
   \nunit{0.001}{\s} for 0$<t<$\nunit{0.01}{\s} and $\Delta t$ = 0.01 for $t
   \geq$ \nunit{0.01}{\s}.}
 \label{fig:advection}
\end{figure}

The exact solution is a Gaussian distribution whose variance increases with
time:
\begin{equation}
  \fnof{u}{x,t} =  \dfrac{M}{\sqrt{4\pi\Delta t}} 
    e^{-\dfrac{\pbrac{x-Vt}^{2}}{4Dt}}
 \label{eqn:Gdist}
\end{equation}

The finite element solution, using the Crank-Nicolson-Galerkin technique,
shows excellent amplitude and phase characteristics when compared with the
exact solution. The finite difference, or lumped mass, solution also using
centered time differences, reproduces the amplitude of the pulse very well but
shows a slight phase lag.

\section{CMISS Examples}

\begin{enumerate}
\item   To solve for the transient heat flow in a plate run CMISS example $331$

\item  To investigate the stability of time integration schemes run CMISS
  examples $3321$ and $3322$.
\end{enumerate}

%%% Local Variables: 
%%% mode: latex
%%% TeX-master: "~/documents/notes/fembemnotes/fembemnotes"
%%% End: 


