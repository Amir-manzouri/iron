\chapter{Derivative BIE}

\section{Boundary Element Formulation}

The BEM will be used in any region of the torso in which the conductivity can
be reasonably taken to be constant (e.g. the lungs).  Thus the equation to be
solved in such a region is simply the Laplace equation.  The conventional
boundary integral equation for Laplace's equation, $\nabla\phi = 0$ in a
(closed) domain $\Omega \subset \Re^{m}\; (m=2$ or 3) can be written as
\begin{equation}
 C(P_{0})\phi(P_{0}) =
 \left(G(P,P_{0})\delby{\phi(P)}{n}-\delby{G(P,P0)}{n}\phi(P)\right)
 d\Gamma(P)
 \label{eq:closed}
\end{equation}
where $C(P_{0}) = 1$ if $P_{0}\in \Omega^{0}$, and $G(P,P_{0})$ is the
fundamental solution for the Laplace equation in either two or three
dimensions (i.e. $G(P,P_{0}) = -\frac{1}{2\pi}\log \parallel P
-P_{0}\parallel$ or $\frac{1}{4\pi}\parallel P-P_{0}\parallel$ ).

The standard procedure in a boundary element formulation is to take the point
$P_{0}$ to be one of the nodes used to approximate the dependent variable
$\phi$, and this generates one integral equation for each node.  When Lagrange
interpolation is used for $\phi$ there is only one unknown per node (allowing
for boundary conditions and assuming no free surfaces). This means that a
square system of equations is produced and a unique solution can be found.
However, when using cubic Hermite interpolation there is always more than one
unknown per node - in two-dimensions there are 2 (with Neuman boundary
conditions these are $\phi$ and $\delby{\phi}{s}$) and in three-dimensions
there are 4 (with Neuman boundary conditions these are $\phi,
\delby{\phi}{s_{1}}, \delby{\phi}{s_{2}}$ and $\frac{\del^{2}\phi}{\del
  s_{1}\del s_{2}}$).  Thus one needs to generate extra linearly independent
equations.  This is achieved by differentiating (\ref{eq:closed}) in various
directions \cite{tomlinson:1996}, yielding a hypersingular integral
formulation.

Taking the directional derivative of (\ref{eq:closed}) at $P_{0}$ in an
(initially) arbitrary direction $n_{0} (\parallel n_{0}\parallel =1)$ gives
\begin{equation}       
  \delby{\phi(P_{0})}{n_{0}} = \int_{\del\Gamma}\left(\delby{G(P,P_{0})}{n}
    \delby{\phi(P)}{n}-\frac{\del^{2}G(P,P_{0})}{\del n\del n_{0}}
    \phi(P)\right)d\Gamma(P)
 \label{eq:arbdir}
\end{equation} 
valid $\forall P_{0}\in \Omega^{0}$.  The hypersingular fundamental solution
$\frac{\del^{2}G(P,P_{0})}{\del n\del n_{0}}$ is given by either
\begin{eqnarray*}
  &\frac{1}{2\pi}\left[\frac{{\bf n.n}_{0}}{r^{2}}-\frac{2({\bf r.n}_{0})
      ({\bf r.n})}{r^{4}}\right] & \mbox{(two-dimensions)}\\ \mbox{or
    }&\frac{1}{4\pi}\left[\frac{{\bf n.n}_{0}}{r^{3}} -\frac{3({\bf r.n}_{0})
      ({\bf r.n})}{r^{5}}\right] & \mbox{(three-dimensions)}
\end{eqnarray*}
Here ${\bf r}$ is the vector of modulus $r = \parallel P-P_{0}\parallel$ from
$P_{0}$ to $P, {\bf n}$ is the unit outward normal vector and $n_{0}$ is the
unit vector in the direction of differentiation.

At this stage the integral expression (\ref{eq:arbdir}) is well defined.  With
$P_{0}$ placed on the boundary of $\Omega$ the second integrand is termed
hypersingular and it is the interpretation of this term and the procedure used
to take $P_{0}$ to the boundary that gives rise to the variety of integral
expressions currently in use in hypersingular boundary integral formulations.
To obtain a weakly-singular form of the derivative equation , one requires the
following identities
\begin{eqnarray}
 \int_{\del\Omega}\delby{G(P,P_{0})}{n}d\Gamma(P)=-1 & & 
  \forall P_{0}\in \Omega^{0}\\
 \nonumber\\
 \int_{\del\Omega}\frac{\del^{2}G(P,P_{0})}{\del n\del n_{0}}d\Gamma(P)=0  & & \forall P_{0}\in \Omega^{0}\\
 \nonumber\\
 \int_{\del\Omega}\left(n_{k}\delby{G(P,P_{0})}{n_{0}}-(x_{k}-x_{0k})
  \frac{\del^{2}G(P,P_{0})}{\del n\del n_{0}}\right)d\Gamma(P)=
  n_{0k}(P_{0}) & & 
  \forall P_{0}\in \Omega^{0}
\end{eqnarray}
The first two of these identities are obtained from (\ref{eq:closed}) and
(\ref{eq:arbdir}) respectively by putting $\phi \equiv 1$.  The third identity
is also obtained from (\ref{eq:arbdir}) by substitution of $x_{k}-x_{0k}$
(where $x_{k}, x_{0k}, n_{k}, n_{0k}$ are the $k$th cartesian components of
$P, P_{0}, n,$ and $n_{0}$ respectively).

The derivation of the weakly singular form of the derivative equation follows
the traditional derivation of the conventional boundary integral equation.
The point $P_{0}$ is placed on the boundary of the domain $\Omega$ and the
domain is enlarged about this point to include a semicircular or hemispherical
region of some small radius $\varepsilon > 0$.  The limits of the integral
expression (\ref{eq:arbdir}) over each part of the enlarged domain are then
considered in detail as$\varepsilon \rightarrow 0$, with judicious application
of the above identities \cite{tomlinson:1996,liu:1991}.  The result is
\begin{eqnarray}
 \int_{\del\Omega}\frac{\del^{2}G(P,P_{0})}{\del n\del n_{0}}
  &\left(\phi(P)-\phi(P_{0})-\delby{\phi(P_{0})}{x_{k}}\right)
  d\Gamma(P)\nonumber\\
  &=\int_{\del\Omega} n_{k}\delby{G(P,P_{0})}{n_{0}}\left(
  \delby{\phi(P)}{n}-\delby{\phi(P_{0})}{x_{k}}n_{k}(P)\right)
  d\Gamma(P)
 \label{eq:result}
\end{eqnarray}
where now $P_{0}\in \del\Omega$ (the singular point), $n$ is a unit outward
normal on $\del\Omega, n_{0}$ is an arbitrary unit vector at $P_{0}$ (the
direction in which (\ref{eq:Laplace}) was differentiated) and a sum over $k$
is implied.

\subsection{Discretisation}
The surface is discretised into ``boundary elements'' i.e.$\Omega =
\displaystyle{\bigcup_{m=1}^{M}}\Omega_{m}$.  Introducing cubic Hermite
interpolation for $\phi $, and, as above, approximating $\delby{\phi(P)}{n}$
by $N_{\alpha}\left(\delby{\phi(P)}{n}\right)_{\alpha}$ and also interpolating
the geometry as $x_{k} = M_{\alpha} x_{k}^{\alpha}$ for some set of basis
functions $\{M_{\alpha}\}$ (these will be taken to be cubic Hermite, but for
now we leave these general), Equation~(\ref{eq:result}) becomes
\begin{eqnarray}
  \sum_{m=1}^{M}\int_{\Gamma_{m}}\frac{\del^{2}G(P,P_{0})} {\del n\del n_{0}}
  &&\left(\Psi_{\alpha}^{i}\phi_{i}^{\alpha}-\phi(P_{0})
    -\delby{\phi(P_{0})}{x_{k}}(M_{\alpha}x_{k}^{\alpha} - x_{0k})
  \right)d\Gamma(P)\nonumber\\ 
  &=&\sum_{m=1}^{M}\int_{\Gamma_{m}}\delby{G(P,P_{0})}{n_{0}}\left(
    N_{\alpha}\delby{\phi^{\alpha}}{n}-\delby{\phi(P_{0})}{x_{k}}
    n_{k}(P)\right)d\Gamma(P)
 \label{eq:result2}
\end{eqnarray}
or, in terms of the local $\xi$ coordinate
\begin{eqnarray}
 \sum_{m=1}^{M}\int_{0}^{1}\frac{\del^{2}G(\xi,P_{0})}
  {\del n\del n_{0}}
  &&\left(\Psi_{\alpha}^{i}(\xi)\phi,_{i}^{\alpha}-\phi(P_{0})
  -\delby{\phi(P_{0})}{x_{k}}(M_{\alpha}(\xi)x_{k}^{\alpha} - x_{0k})
  \right)|J(\xi)d\xi\nonumber\\
 &=&\sum_{m=1}^{M}\int_{0}^{1}\delby{G(\xi,P_{0})}{n_{0}}\left(
  N_{\alpha}(\xi)\delby{\phi^{\alpha}}{n}
  -\delby{\phi(P_{0})}{x_{k}}n_{k}(\xi)
  \right)|J(\xi)d\xi
 \label{eq:result3}
\end{eqnarray}
As with the conventional boundary integral equation, $P_{0}$ is located at
each of the solution variable nodes in turn.

The global unknowns we are dealing with are nodal values of
$\phi,\delby{\phi}{s}$ and $\phi,\delby{\phi}{n}$ (and
also$\frac{\del^{2}\phi}{\del s\del n}$ if cubic Hermite interpolation is used
for the normal derivative).  By adopting a local coordinate system ($s,n$) [or
$(\xi,n)$] one has, by the chain rule,
\begin{eqnarray}
 \delby{\phi}{x_{k}}=\left(\delby{\phi}{\sigma_{l}}\delby{\sigma_{l}}
  {x_{k}}\right) & & (l=1,2)
 \label{eq:chain}
\end{eqnarray}
where ($\sigma_{1},\sigma_{2}) = (s,n)$ [or ($s_{1}, s_{2}, n$) in three
dimensions]. $\delby{\sigma_{l}}{x_{k}}$ can either be obtained exactly (as in
some of the test problems) or, more generally, by inverting the matrix
$\left[\delby{x_{k}}{\sigma_{l}}\right]$.  For a cubic Hermite geometric mesh,
$\delby{x_{k}}{s}$ are known nodal values (either entered or calculated as
part of the mesh fitting) so the matrix entries are all known.  The use of
(\ref{eq:chain}) in (\ref{eq:result3}) yields an integral expression, which,
with the application of element scale factors (if required), involves only the
required nodal unknowns as coefficients.  We concentrate now on obtaining
expressions suitable for numerical computation for these coefficients.  As in
\cite{liu:1992} two separate cases must be considered - that for which $P_{0}$
is contained in the element $\Gamma_{m}$ being considered, and that for which
$P_{0}$ is removed from the current element $\Gamma_{m}$.  For each case, we
investigate both integrals of (\ref{eq:result3}).

\subsection*{Case 1:  $P_{0}??? \Gamma_{m}$ SHOULD BE NOT AN ELEMENT} 
In this case the integrands of both integrals in (\ref{eq:result3}) are
nonsingular and both integrals exist.  The first integral can be written as
\begin{eqnarray}
 \phi,_{i}^{\alpha}\int_{0}^{1}\frac{\del^{2}G(\xi,P_{0})}
  {\del n\del n_{0}}\Psi_{\alpha}^{i}(\xi)|J(\xi)|d\xi &-& \phi(P_{0})
  \int_{0}^{1}\frac{\del^{2}G(\xi,P_{0})}
  {\del n\del n_{0}}|J(\xi)|d\xi\nonumber\\
 &-&\delby{\phi}{\sigma_{l}}(P_{0})\delby{\sigma_{l}}{x_{k}}(P_{0})
   \int_{0}^{1}\frac{\del^{2}G(\xi,P_{0})}{\del n\del n_{0}}
   (M_{\alpha}(\xi)x_{k}^{\alpha}-x_{0k})|J(\xi)|d\xi
 \label{eq:1stint}
\end{eqnarray}
Since $\phi,_{i}^{\alpha}$ is either a nodal value of $\phi$ or
$\delby{\phi}{\xi}$, multiplication of the appropriate first integrals by
element scale factors generates coefficients of nodal values of
$\delby{\phi}{s}$ .  Each integral expression is (\ref{eq:1stint}) is regular
and standard Gaussian quadrature can be used to evaluate these.

The second integral can be written as
\begin{equation}
 \delby{\phi^{\alpha}}{n}\int_{0}^{1}\frac{\del^{2}G(\xi,P_{0})}
  {\del n\del n_{0}}N_{\alpha}(\xi)|J(\xi)|d\xi - 
  \delby{\phi}{\sigma_{l}}(P_{0})\delby{\sigma_{l}}{x_{k}}(P_{0})
   \int_{0}^{1}\frac{\del G(\xi,P_{0})}{\del n\del n_{0}}
   n_{k}(\xi)|J(\xi)|d\xi
 \label{eq:2ndint}
\end{equation}
and again each integral can be evaluated by standard Gaussian quadrature.  A
more efficient grouping of the integrals in (\ref{eq:1stint}) and
(\ref{eq:2ndint}) can be obtained by grouping like coefficients, resulting in
some computational saving.

\subsection*{Case 2:  $P_{0}\in \Gamma_{m}$} 
In this case special care must be exercised to retain the weakly singular
nature of the integral expressions in (\ref{eq:result3}).  A straightforward
application of either (\ref{eq:1stint}) or (\ref{eq:2ndint}) will result in
divergent integrals.  We identify with $\beta$ the local node on element
$\Gamma_{m}$ corresponding to $P_{0}$.  Keeping the appropriate coefficients
grouped, we have from the first integral of (\ref{eq:result3})
\begin{eqnarray}
 \phi(P_{0})\int_{0}^{1}\frac{\del^{2}G(\xi,P_{0})}
  {\del n\del n_{0}}(\Psi_{\beta}^{0}(\xi)-1)|J(\xi)|d\xi & + 
  \phi ,_{i}^{\alpha\{\alpha\neq\beta\}}\int_{0}^{1}
  \frac{\del^{2}G(\xi,P_{0})}
  {\del n\del n_{0}}(\Psi_{\alpha}^{i}(\xi)-1)|J(\xi)|d\xi\nonumber\\
 +\delby{\phi}{s}(P_{0})\int_{0}^{1}\frac{\del^{2}G(\xi,P_{0})}
  {\del n\del n_{0}}\left[\delby{s}{\xi}(P_{0})\Psi_{\beta}^{1}(\xi)\right.
 & \left.- \delby{s}{x_{k}}(P_{0})(M_{\alpha}(\xi)x_{k}^{\alpha}-x_{0k})\right]
 |J(\xi)|d\xi\nonumber\\
 -\delby{\phi}{n}(P_{0})\int_{0}^{1}\frac{\del^{2}G(\xi,P_{0})}
  {\del n\del n_{0}}\delby{n}{x_{k}}(P_{0})
  \left(M_{\alpha}(\xi)x_{k}^{\alpha}-x_{0k}\right)|J(\xi)|d\xi
 \label{eq:from1stint}
\end{eqnarray}
Here we have explicitly included the element scale factors, and the second
integral involves a sum over $\alpha$ with $\alpha\neq\beta$.  With this
grouping, all integrals are at most weakly singular and amenable to numerical
integration.  The second integral of (\ref{eq:result3}) is more
straightforward, although certain groupings must still be retained.  The
appropriate expression for numerical integration is
\begin{eqnarray}
 \delby{\phi}{n}(P_{0}) &\int_{0}^{1}\frac{\del G(\xi,P_{0})}
  {\del n_{0}}\left[ N_{\beta}(\xi)-\delby{n}{x_{k}}(P_{0})n_{k}(\xi)\right]
  |J(\xi)|d\xi \nonumber\\
 &-&\delby{\phi^\alpha{\alpha\neq\beta}}{n}\int_{0}^{1}
  \frac{\del G(\xi,P_{0})}{\del n_{0}}N_{\alpha}(\xi)|J(\xi)|d\xi\nonumber\\
 &-&\delby{\phi}{s}(P_{0})\int_{0}^{1}\frac{\del G(\xi,P_{0})}
  {\del n_{0}}\delby{s}{x_{k}}(P_{0})n_{k}(\xi)|J(\xi)|d\xi
 \label{eq:approp}
\end{eqnarray}
The above equations provide numerically well-behaved expressions for the
coefficients of the discretised derivative boundary integral equation.  It
only remains to specify the direction(s) of $n_{0}$.  After extensive testing
\cite{tomlinson:1996} we concluded that the direction of differentiation
should be that of $s$ (i.e.  tangential) for two-dimensional problems (where
one extra equation is required) and for three-dimensional problems, one should
differentiate in both the $s_{1}$ and $s_{2}$ directions (and set the cross
derivative coefficients to zero).

\section{Three-dimensional element splitting}

If the singular point $P_{0}$ is contained in the current element then the
element is sub- divided in local ($\xi_{1}, \xi_{2}$) space according to the
location of $P_{0}$.  Each of these triangular elements is then transformed to
a regular four-noded element in another space [say ($s, t$) space] in such a
way that the Jacobian of this last transformation cancels the dominant
singularity in the integrand (which is $O( \frac{1}{r} )$ in the original
space, where $r$ is the distance from the singular point $P_{0}$).  The
numerical integrations are then performed over the elements in the ($s,t$)
space.  We use the following subdivisions and transformations.  In each case
the absolute value of the Jacobian of the transformation for element $A$ is
$s$ and for element $B$ is $t$.

\begin{figure}
\centering
 %\input{figs/A1.pstex}
 \caption[The element splitting used when $P_{0}$ is at local node 1]{The element splitting used when $P_{0}$ is at local node 1.  The individual transformations 
   yield Jacobians that reduce the order of the singularity in the otherwise
   singular boundary element integrals.}
\label{fig:A1}
\end{figure}

\subsection{$P_{o}$ at local node 1}

Sub-divide from local node 1 to 4, giving two elements $A$ and $B$.

Transformation for element $A$:
\begin{displaymath}
\begin{array}{llllr}        
  s = \xi_{1} & t = \frac{\xi_{2}}{\xi_{1}} & \left(\mbox{inverse mapping
      is}\right. & \xi_{1} = s & \left.\xi_{2} = s t \right)
\end{array}
\end{displaymath}

Transformation for element $B$:
\begin{displaymath}
\begin{array}{llllr}        
 s = \frac{\xi_{2}}{\xi_{1}} & t = \xi_{2} & 
  \left(\mbox{inverse mapping is}\right. & \xi_{1} = st & 
  \left.\xi_{2} = t \right)
\end{array}
\end{displaymath} 
 
\subsection{$P_{o}$ at local node 2}
Sub-divide from local node 2 to 3.  Element $A$ contains local nodes 1, 2 and
3.

Transformation for element $A$:
\begin{displaymath}
\begin{array}{llllr}        
 s = 1-\xi_{1} & t = \frac{1-\xi_{1}-\xi_{2}}{1-\xi_{1}} & 
  \left(\mbox{inverse mapping is}\right. & \xi_{1} = 1-s & 
  \left.\xi_{2} = s (1-t) \right)
\end{array}
\end{displaymath}

Transformation for element $B$:
\begin{displaymath}
\begin{array}{llllr}        
 s = \frac{1-\xi_{1}}{\xi_{2}} & t = \xi_{2} & 
  \left(\mbox{inverse mapping is}\right. & \xi_{1} = 1-st & 
  \left.\xi_{2} = t \right)
\end{array}
\end{displaymath}  

\subsection{$P_{o}$ at local node 3}
Sub-divide from local node 2 to 3.  Element $A$ contains local nodes 2,3 and 4.

Transformation for element $A$:
\begin{displaymath}
\begin{array}{llllr}        
 s = \xi_{1} & t = \frac{1-\xi_{2}}{\xi_{1}} & 
  \left(\mbox{inverse mapping is}\right. & \xi_{1} = s & 
  \left.\xi_{2} = 1-st \right)
\end{array}
\end{displaymath}

Transformation for element $B$:
\begin{displaymath}
\begin{array}{llllr}        
 s = \frac{1-\xi_{1}-\xi_{2}}{1-\xi_{2}} & t = 1-\xi_{2} & 
  \left(\mbox{inverse mapping is}\right. & \xi_{1} = t(1-s) & 
  \left.\xi_{2} = 1-t \right)
\end{array}
\end{displaymath}

\subsection{$P_{o}$ at local node 4}
Sub-divide from local node 1 to 4.  Element $A$ contains local nodes 1,3 and 4.
        
Transformation for element $A$:
\begin{displaymath}
\begin{array}{llllr}        
 s = 1-\xi_{1} & t = \frac{1-\xi_{2}}{1-\xi_{1}} & 
  \left(\mbox{inverse mapping is}\right. & \xi_{1} = 1-s & 
  \left.\xi_{2} = 1-st \right)
\end{array}
\end{displaymath}

Transformation for element $B$:
\begin{displaymath}
\begin{array}{llllr}        
 s = \frac{1-\xi_{1}}{1-\xi_{2}} & t = 1-\xi_{2} & 
  \left(\mbox{inverse mapping is}\right. & \xi_{1} = 1-st & 
  \left.\xi_{2} = 1-t \right)
\end{array}
\end{displaymath}

\section{Hermite ``Simplexes''}

We describe here the special three-noded Hermite elements that have been used
to close a cubic Hermite surface in three-dimensional space.  The need for
such elements was illustrated in Figure~\ref{fig:2}.  We describe the two
special elements that were used - one in which the derivatives come to a point
at local node 3 (the top), and one in which the derivatives come together at
local node 1 (the bottom).

\begin{figure}
\centering
 %\input{figs/B1.pstex}
 \caption[The two Hermite ``simplexes'']{The two Hermite ``simplexes'' and their connection to standard cubic Hermite 
elements.}
\label{fig:B1}
\end{figure}

\subsection{Apex at local node 3}
With reference to Figure~\ref{fig:B1}~a, we construct the appropriate
two-dimensional interpolation functions via a tensor product (instead of using
area functions, which are commonly used in three-noded elements).  The tensor
product formulation requires a description of the basis functions to be used
in each direction.  We use standard Hermite interlation in the $\xi_{1}$
direction, so that any standard cubic Hermite element sharing this edge
maintains consistent interpolation.  In the $\xi_{2}$ direction we require the
interpolation function to interpolate nodal values of function and derivative
at the first node, and only the value of the function at the second node (i.e.
we have dropped the arclength derivative at the second node since it is no
longer uniquely defined).  Thus we require
\begin{displaymath}
 {\bf x}(\xi_{2}) = \Psi_{1}(\xi_{2}){\bf x}_{1} + \Psi_{2}(\xi_{2}) 
  \delby{{\bf x}_{1}}{\xi_{2}}  + \Psi_{3}(\xi_{2}) {\bf x}_{3}
\end{displaymath}  
where   
\begin{eqnarray*}
 \Psi_{1} (0)  = 1,&\Psi_{1} (1)=0,&\delby{\Psi_{1}}{\xi_{2}} (0) = 0\\  \\
 \Psi_{2} (0)  = 0,&\Psi_{2} (1)=0,&\delby{\Psi_{2}}{\xi_{2}}(0) = 1\\
 \\  
 \Psi_{3} (0)  = 0,&\Psi_{3} (1)=1,&\delby{\Psi_{3}}{\xi_{2}} (0) = 0
\end{eqnarray*}  
These conditions yield
\begin{eqnarray*}
 \Psi_{1} (\xi) & = & 1 - \xi^{2}\\ 
 \Psi_{2} (\xi) & = & \xi - \xi^{2}\\ 
 \Psi_{3} (\xi) & = & \xi^{2}
\end{eqnarray*} 
and the appropriate two-dimensional basis function is obtained from a tensor
product of the standard Hermite basis function and the family given above.

\subsection{Apex at local node 1}
Again, with reference to Figure~\ref{fig:B1}~b, we construct the appropriate
two-dimensional interpolation functions via a tensor product with standard
Hermite interpolation in the $\xi_{1}$ direction.  In the $\xi_{2}$ direction
we require the interpolation function to interpolate nodal values of function
at the first node, and the value of the function and derivative at the second
node.  Thus we require
\begin{displaymath}
 {\bf x}(\xi_{2}) = \zeta_{1}(\xi_{2}){\bf x}_{1} + \zeta_{2}(\xi_{2}) 
  \delby{{\bf x}_{3}}{\xi_{2}}  + \zeta_{3}(\xi_{2}) {\bf x}_{3}
\end{displaymath}  
where   
\begin{eqnarray*}
 \zeta_{1} (0)= 1,&\zeta_{1} (1)=0,&\delby{\zeta_{1}}{\xi_{2}} (1) = 0\\  \\
 \zeta_{2} (0)= 0,&\zeta_{2} (1)=0,&\delby{\zeta_{2}}{\xi_{2}}(1) = 1\\
 \\  
 \zeta_{3} (0)= 0,&\zeta_{3} (1)=1,&\delby{\zeta_{3}}{\xi_{2}} (1) = 0
\end{eqnarray*}  
These conditions yield
\begin{eqnarray*}
 \zeta_{1} (\xi) & = & (\xi-1)^{2}\\ 
 \zeta_{2} (\xi) & = & \xi^{2} - \xi\\ 
 \zeta_{3} (\xi) & = & 2\xi-\xi^{2}
\end{eqnarray*} 
and the appropriate two-dimensional basis function are obtained as above.

%%% Local Variables: 
%%% mode: latex
%%% TeX-master: "/product/cmiss/documents/notes/fembemnotes/fembemnotes"
%%% End: 
