 
\subsection{Finite Elasticity}
\label{subsec:FiniteElasticity}

\subsubsection{Kinematics}

As shown in \figref{fig:configurationsettting}, consider a \textit{material
  body} which is a three-dimensional smooth manifold with a boundary, $\manifold{B}$,
which consists of a set of points which are refered to as \textit{material
  points}. Consider also an ambient space manifold,
$\manifold{S}\element\rntopology{n}$. The material body is only accessible to
the observer when it moves through the ambient space. This motion is a
time-dependent embedding on the material body into the ambient space. The
embedding is known as a \textit{placement of the body}. It is given by the
mapping
\begin{equation}
  \mapping{\fnof{\kappa}{\mathcal{X},t}}{\manifold{B}}{\manifold{S}}
\end{equation}

The embedded submanifold occupying a location in the ambient space is is
called a \textit{configuration} of $\manifold{B}$ and is given by
\begin{equation}
  \embedmanifold{B}_{t}=\fnof{\kappa_{t}}{\manifold{B}}=\fnof{\kappa}{\manifold{B},t}
\end{equation}

The customary (but not necessary) \textit{reference placement} is given by
\begin{equation}
  \mapping{\kappa_{0}}{\manifold{B}}{\manifold{S}}
\end{equation}
and the region of space occupied by the reference placement \ie the
\textit{reference configuration} is given by
\begin{equation}
  \embedmanifold{B}_{0}=\fnof{\kappa_{0}}{\manifold{B}}
\end{equation}
Points in $\embedmanifold{B}$ are denoted by capital letters \ie $X, Y,
\ldots$.

\epstexfigure{svgs/EquationSets/Elasticity/FiniteElasticity/setup.eps_tex}{}{}{fig:configurationsetting}{0.75}

A new configuration of $\manifold{B}$ is given by the deformation mapping
\begin{equation}
  \mapping{\chi}{\embedmanifold{B}}{\rntopology{3}}
\end{equation}
where a configuration represents a deformed state of the body. As the body
moves we obtain a family of configurations. If we hold $X\in\embedmanifold{B}$
fixed can write $\fnof{V_{t}}{X}=\fnof{V}{X,t}$. We then have
\begin{equation}
  \fnof{V_{t}}{X}=\fnof{V}{X,t}=\delby{\fnof{\chi}{X,t}}{t}=\dby{\fnof{\chi_{X}}{t}}{t}
\end{equation}

Here $V_{t}$ is called the \textit{material velocity} of the motion. The
\textit{material acceleration} of the body is defined as
\begin{equation}
  \fnof{A_{t}}{X}=\fnof{A}{X,t}=\delby{\fnof{V}{X,t}}{t}=\dby{\fnof{V_{X}}{t}}{t}
\end{equation}

The \textit{spatial velocity} of the motion is defined by $v_{t}$ and the
\textit{spatial acceleration} of the motion is defined by $a_{t}$.

\subsubsection{Deformation Gradient}

Let
$\mapping{\chi}{\embedmanifold{B}_{0}}{\fnof{\chi}{\embedmanifold{B}_{0}}\subset\manifold{S}}$
be a deformation configuration of $\embedmanifold{B}$ in $\manifold{S}$. The
tangent of the mapping \ie $\tangentbundle{\chi}$ is denoted as $\tensor{F}$
and is called the \textit{deformation gradient} of $\chi$ \ie
$\tensor{F}=\tangentbundle{\chi}$. For $X\in\embedmanifold{B}$ we have
\begin{equation}
  \tensor{F}_{X}=\mapping{\fnof{\tensor{F}}{X}}{\tangentspace{\embedmanifold{B}}{X}}{\tangentspace{\mainfold{S}}{\fnof{\chi}{X}}}
\end{equation}
 
If $X^{A}$ and $x^{a}$ are the coordinates on $\embedmanifold{B}$ and
$\manifold{S}$ then the deformation gradient tensor with respect to the
coordinate bases are
\begin{equation}
  \fnof{F^{a}_{A}}{X}=\delby{\fnof{\chi^{a}}{X}}{X^{A}}
\end{equation}

Note that F is a two-point tensor. 

The \textit{right Cauchy-Green (or Green) deformation tensor}, $\tensor{C}$, is defined by
\begin{equation}
  \mapping{\fnof{\tensor{C}}{X}}{\tangentspace{\embedmanifold{B}}{X}}{\tangentspace{\embedmanifold{B}}{X}}}
\end{equation}
\ie $\fnof{\tensor{C}}{X}=\transpose{\fnof{\tensor{F}}{X}}\fnof{\tensor{F}}{X}$
or $\tensor{C}=\transpose{\tensor{F}}\tensor{F}$. In terms of coordinates we
have
\begin{equation}
  C^{A}_{B}=\pbrac{\transpose{F}}^{A}_{a}F^{a}_{B}=g_{ab}G^{AC}\delby{\chi^{b}}{X^{C}}\delby{\chi^{a}}{X^{B}}
\end{equation}

The \textit{left Cauchy-Green (or Finger) deformation tensor}, $\tensor{C}$, is defined by
\begin{equation}
  \mapping{\fnof{\tensor{C}}{x}}{\tangentspace{\fnof{\chi}{\embedmanifold{B}}}{x}}{\tangentspace{\fnof{\chi}{\embedmanifold{B}}}{x}}}
\end{equation}
\ie $\fnof{\tensor{b}}{x}=\fnof{\tensor{F}}{X}\transpose{\fnof{\tensor{F}}{X}}$
or $\tensor{C}=\tensor{F}\transpose{\tensor{F}}$ where $X=\fnof{\inverse{\chi}}{x}$. In terms of coordinates we
have
\begin{equation}
  b^{a}_{b}=\pbrac{\transpose{F}}^{A}_{a}F^{a}_{B}=g_{ab}G^{AC}\delby{\chi^{b}}{X^{C}}\delby{\chi^{a}}{X^{B}}
\end{equation}

The polar decomposition

\begin{diagram}
 & & \tangentspace{B}{X} & & \\
 & \ruTo^{\tensor{U}} & & \rdTo^{\tensor{R}} \\
\tangentspace{B}{X} & & \rTo^{\tensor{F}} & & \tangentspace{S}{x}\\
 & \rdTo_{\tensor{R}} & & \ruTo_{\tensor{V}} \\
 & &  \tangentspace{S}{x} & &
\end{diagram}

\subsection{Old Stuff}

Formulation of finite element equations for finite elasticity (large
deformation mechanics) implemented in OpenCMISS is based on the
\textit{\textbf{principle of virtual work}}. The finite element model consists
of a set of non-linear algebraic equations. Non-linearity of equations stems
from non-linear stress-strain relationship and quadratic terms present in the
strain tensor. A typical problem in large deformation mechanics involves
determination of the deformed geometry or mesh nodal parameters, from the
finite element point of view, of the continuum from a known undeformed
geometry, subject to boundary conditions and satisfying stress-strain
(constitutive) relationship.
  
The boundary conditions can be either \textit{\textbf{Dirichlet}}
(displacement), \textit{\textbf{Neumann}} (force) or a combination of them,
known as the mixed boundary conditions. Displacement boundary conditions are
generally nodal based. However, force boundary conditions can take any of the
following forms or a combination of them - nodal-based, distributed load
(e.g. pressure) or force acting at a discrete point on the boundary. In the
latter two forms, the equivalent nodal forces are determined using the
\textit{\textbf{method of work equivalence}} \cite{hutton:2004} and the forces
so obtained will then be added to the right hand side or the residual vector
of the linear equation system.

There are a numerous ways of describing the mechanical characteristics of
deformable materials in large deformation mechanics or finite elasticity
analyses. A predominantly used form for representing constitutive properties
is a strain energy density function. This model gives the energy required to
deform a unit volume (hence energy density) of the deformable continuum as a
function of Green-Lagrange strain tensor components or its derived variables
such as invariants or principal stretches. A material that has a strain energy
density function is known as a \textit{\textbf{hyperelastic}} or
\textit{\textbf{Green-elastic material}}.

The deformed equilibrium state should also give the minimum total elastic
potential energy. One can therefore formulate finite element equations using
the \textit{\textbf{Variational method}} approach where an extremum of a
functional (in this case total strain energy) is determined to obtain mesh
nodal parameters of the deformed continuum. It is also possible to derive the
finite element equations starting from the governing equilibrium equations
known as Cauchy equation of motion. The weak form of the governing equations
is obtained by multiplying them with suitable weighting functions and
integrating over the domain (method of weighted residuals). If interpolation
or shape functions are used as weighting functions, then the method is called
the Galerkin finite element method. All three approaches (virtual work,
variational method and Galerkin formulation) result in the same finite element
equations.

In the following sections the derivation of kinematic relationships of
deformation, energy conjugacy, constitutive relationships and final form the
finite element equations using the virtual work approach will be discussed in
detail.

\subsubsection{Kinematics of Deformation}
In order to track the deformation of an infinitesimal length at a particle of
the continuum, two coordinates systems are defined. An arbitrary orthogonal
spatial coordinate system, which is fixed in space and a material coordinate
system which is attached to the continuum and deforms with the continuum. The
material coordinate system, in general, is a curvi-linear coordinate system
but must have mutually orthogonal axes at the undeformed state. However, in
the deformed state, these axes are no longer orthogonal as they deform with
the continuum (fig 1). In addition to these coordinate systems, there exist
finite element coordinate systems (one for each element) as well. These
coordinates are normalised and vary from 0.0 to 1.0. The following notations are used to represent various coordinate systems and coordinates of a particle of the continuum.\\

\noindent $Y_{1}$-$Y_{2}$-$Y_{3}$ - fixed spatial coordinate system axes - orthogonal\\
$N_{1}$-$N_{2}$-$N_{3}$ - deforming material coordinate system axes  - orthogonal in the undeformed state\\
$\Xi_{1}$-$\Xi_{2}$-$\Xi_{3}$ - element coordinate system - non-orthogonal in general and deforms with continuum\\

\noindent $x_{1}$-$x_{2}$-$x_{3}$ [$\vect{x}$] - spatial coordinates of a particle in the undeformed state wrt $Y_{1}$-$Y_{2}$-$Y_{3}$ CS \\
$z_{1}$-$z_{2}$-$z_{3}$ [$\vect{z}$] - spatial coordinates of the same particle in the deformed state wrt $Y_{1}$-$Y_{2}$-$Y_{3}$ CS \\
$\nu_{1}$-$\nu_{2}$-$\nu_{3}$ [$\vect{\nu}$] - material coordinates of the particle wrt $N_{1}$-$N_{2}$-$N_{3}$ CS (these do not change) \\
$\xi_{1}$-$\xi_{2}$-$\xi_{3}$ [$\vect{\xi}$] - element coordinates of the particle wrt $\Xi_{1}$-$\Xi_{2}$-$\Xi_{3}$ CS (these too do not change)\\

Since the directional vectors of the material coordinate system at any given
point in the undeformed state is mutually orthogonal, the relationship between
spatial $\vect{x}$ and material $\vect{\nu}$ coordinates is simply a
rotation. The user must define the undeformed material coordinate
system. Typically a nodal based interpolatable field known as fibre
information (fibre, imbrication and sheet angles) is input to OpenCMISS. These
angles define how much the \textit{\textbf{reference or default material
    coordinate system}} must be rotated about the reference material axes. The
reference material coordinate system at a given point is defined as
follows. The first direction $\nu_{1}$ is in the $\xi_{1}$ direction. The
second direction, $\nu_{2}$ is in the $\xi_{1}-\xi_{2}$ plane but orthogonal
to $\nu_{1}$. Finally the third direction $\nu_{3}$ is determined to be normal
to both $\nu_{1}$ and $\nu_{2}$. Once the reference coordinate system is
defined, it is then rotated about $\nu_{3}$ by an angle equal to the
interpolated fibre value at the point in counter-clock wise direction. This
will be followed by a rotation about new $\nu_{2}$ axis again in the
counter-clock wise direction by an angle equal to the sheet value. The final
rotation is performed about the current $\nu_{1}$ by an angle defined by
interpolated sheet value. Note that before a rotation is carried out about an
arbitrary axis one must first align(transform) the axis of rotation with one
of the spatial coordinate system axes. Once the rotation is done, the rotated
coordinate system (material) must be inverse-transformed.

Having defined the undeformed orthogonal material coordinate system, the
metric tensor $\delby{\vect{x}}{\vect{\nu}}$ can be determined. As mentioned,
the tensor $\delby{\vect{x}}{\vect{\nu}}$ contains rotation required to align
material coordinate system with spatial coordinate system. This tensor is
therefore orthogonal. A similar metric tensor can be defined to relate the
deformed coordinates $\vect{z}$ of the point to its material coordinates
$\vect{\nu}$. Note that the latter coordinates do not change as the continuum
deforms and more importantly this tensor is not orthogonal as well. The metric
tensor, $\delby{\vect{z}}{\vect{\nu}}$ is called the
\textit{\textbf{deformation gradient tensor}} and denoted as $\matr{F}$.

\begin{equation}
  \matr{F}=\delby{\vect{z}}{\vect{\nu}}
  \label{eqn:deformationgradienttensor}
\end{equation}
 
It can be shown that the deformation gradient tensor contains rotation when an
infinitesimal length $\vect{dr_{0}}$ in the undeformed state undergoes
deformation. Since rotation does not contribute to any strain, it must be
removed from the deformation gradient tensor. Any tensor can be decomposed
into an orthogonal tensor and a symmetric tensor (known as polar
decomposition). In other words, the same deformation can be achieved by first
rotating $\vect{dr}$ and then stretching (shearing and scaling) or
vice-verse. Thus, the deformation gradient tensor can be given by,

\begin{equation}
  \matr{F}=\delby{\vect{z}}{\vect{\nu}}=\matr{R}\matr{U}=\matr{V}\matr{R_{1}}
  \label{eqn:polardecomposition}
\end{equation}
 
The rotation present in the deformation gradient tensor can be removed either
by right or left multiplication of $\matr{F}$. The resulting tensors lead to
different strain measures. The right Cauchy deformation tensor $\matr{C}$ is
obtained from,

\begin{equation}
  \matr{C}=\transpose{[\matr{R}\matr{U}]}[\matr{R}\matr{U}]=\transpose{\matr{U}}\transpose{\matr{R}}\matr{R}\matr{U}=\transpose{\matr{U}}\matr{U}
  \label{eqn:rightcauchy}
\end{equation}

Similarly the left Cauchy deformation tensor or the Finger tensor \matr{B} is
obtained from the left multiplication of \matr{F},

\begin{equation}
  \matr{B}=[\matr{V}\matr{R_{1}}]\transpose{[\matr{V}\matr{R_{1}}]}=\matr{V}\matr{R_{1}}\transpose{\matr{R_{1}}}\transpose{\matr{V}}=\matr{V}\transpose{\matr{V}}
  \label{eqn:leftcauchy}
\end{equation}

\noindent Note that both $\matr{R}$ and $\matr{R_{1}}$ are orthogonal tensors
and therefore satisfy the following condition,

\begin{equation}
  \transpose{\matr{R}}\matr{R}=\matr{R_{1}}\transpose{\matr{R_{1}}}=\matr{I}
  \label{eqn:orthoganality}
\end{equation}

Since there is no rotation present in both $\matr{C}$ and $\matr{B}$, they can
be used to define suitable strain measures as follows,

\begin{equation}
  \matr{E}=\frac{1}{2}\pbrac{\transpose{\delby{\vect{z}}{\vect{\nu}}}\delby{\vect{z}}{\vect{\nu}}-
                       \transpose{\delby{\vect{x}}{\vect{\nu}}}\delby{\vect{x}}{\vect{\nu}}}=
	    \frac{1}{2}(\matr{C}-\matr{I})	       
  \label{eqn:greenstrain}
\end{equation}

\noindent and

\begin{equation}
  \vect{e}=\frac{1}{2}\bbrac{\pbrac{\delby{\vect{x}}{\vect{\nu}}\transpose{\delby{\vect{x}}{\vect{\nu}}}}^{-1}-
                             \pbrac{\delby{\vect{z}}{\vect{\nu}}\transpose{\delby{\vect{z}}{\vect{\nu}}}}^{-1}}=
			     \frac{1}{2}\pbrac{\matr{I}-\matr{B}^{-1}}  
  \label{eqn:almansistrain}
\end{equation}

\noindent where $\matr{E}$ and $\vect{e}$ are called Green and Almansi strain tensors respectively. 
Also note that $\delby{\vect{x}}{\vect{\nu}}$ is an orthogonal tensor. \\

It is now necessary to establish a relationship between strain and displacement. Referring to figure 1, 

\begin{equation}
  \vect{z}=\vect{x}+\vect{u}
  \label{eqn:displacement}
\end{equation}

\noindent where \vect{u} is the displacement vector. \\

\noindent Differentiating \eqnref{eqn:displacement} using the chain rule,

\begin{equation}
  \delby{\vect{z}}{\vect{\nu}}=\delby{\vect{x}}{\vect{\nu}}+\delby{\vect{u}}{\vect{x}}\delby{\vect{x}}{\vect{\nu}}=
                               \pbrac{\matr{I}+\delby{\vect{u}}{\vect{x}}}\delby{\vect{x}}{\vect{\nu}}  
  \label{eqn:displacementgradient}
\end{equation}

\noindent Substituting \eqnref{eqn:displacementgradient} into \eqnref{eqn:greenstrain},

\begin{equation}
  \matr{E}=\frac{1}{2}\bbrac{\transpose{\delby{\vect{x}}{\vect{\nu}}}\transpose{\pbrac{\matr{I}+\delby{\vect{u}}{\vect{x}}}}
                  \pbrac{\matr{I}+\delby{\vect{u}}{\vect{x}}}\delby{\vect{x}}{\vect{\nu}}-\matr{I}}
  \label{eqn:greendisplacement1}
\end{equation}

\noindent Simplifying,

\begin{equation}
  \matr{E}=\frac{1}{2}\transpose{\delby{\vect{x}}{\vect{\nu}}}
           \pbrac{\delby{\vect{u}}{\vect{x}}+\transpose{\delby{\vect{u}}{\vect{x}}}+
	   \transpose{\delby{\vect{u}}{\vect{x}}}\delby{\vect{u}}{\vect{x}}}
	   \delby{\vect{x}}{\vect{\nu}}
  \label{eqn:greendisplacement2}
\end{equation}
 
As can be seen from \eqnref{eqn:greendisplacement2} the displacement gradient
tensor $\delby{\vect{u}}{\vect{x}}$ is defined with respect to undeformed
coordinates $\vect{x}$. This means that the strain tensor $\matr{E}$ has
Lagrangian description and hence it is also also called the Green-Lagrange
strain tensor.
 
A similar derivation can be employed to establish a relationship between the
Almansi and displacement gradient tensors and the final form is given by,

\begin{equation}
  \vect{e}=\frac{1}{2}\delby{\vect{u}}{\vect{z}}+\transpose{\delby{\vect{u}}{\vect{z}}}-
	   \transpose{\delby{\vect{u}}{\vect{z}}}\delby{\vect{u}}{\vect{z}}
  \label{eqn:almansidisplacement}
\end{equation}
 
The displacement gradient tensor terms in \eqnref{eqn:almansidisplacement} are defined with respect to deformed coordinates $\vect{z}$ and
therefore the strain tensor has Eulerian description. Thus it is also known as the Almansi-Euler strain tensor.

\subsubsection{Energy Conjugacy}



\subsubsection{Constitutive models}



\subsubsection{Principle of Virtual Work}
Elastic potential energy or simply elastic energy associated with the
deformation can be given by strain and its energetically conjugate stress.
Note that the Cauchy stress and Almansi-Euler strain tensors and Second
Piola-Kirchhoff (2PK) and Green-Lagrange tensors are energetically
conjugate. Thus, the \textit{\textbf{total internal energy}} due to strain in
the body at the deformed state (fig. 3.1) can be given by,
 
\begin{equation}
  W_{int}=\gint{0}{v}{(\vect{e}:\vect{\sigma})}v
  \label{eqn:totalenergy}
\end{equation}

where \vect{e} and \vect{\sigma} are Almansi strain tensor and Cauchy stress
tensor respectively.

If the deformed body is further deformed by introducing virtual displacements,
then the new internal elastic energy can be given by,

\begin{equation}
  {W_{int}+\delta W_{int}}=\gint{0}{v}{[\vect{(e+\delta{e})}:\vect{\sigma}]}v
  \label{eqn:virtualtotalenergy}
\end{equation}

Deducting \eqnref{eqn:totalenergy} from \eqnref{eqn:virtualtotalenergy},

\begin{equation}
  \delta W_{int}=\gint{0}{v}{\pbrac{\vect{\delta \epsilon} : \vect{\sigma}}}v
  \label{eqn:virtualenergy}
\end{equation}

Using \eqnref{eqn:almansidisplacement} for virtual strain,

\begin{equation}
  \vect{\delta e}=\delby{\vect{\delta u}}{\vect{z}} + \transpose{\delby{\vect{\delta u}}{\vect{z}}} + 
                  \transpose{\delby{\vect{\delta u}}{\vect{z}}}\delby{\vect{\delta u}}{\vect{z}}
  \label{eqn:virtualalmansidisplacement}
\end{equation}

Since virtual displacements are infinitesimally small, quadratic terms in
\eqnref{eqn:virtualalmansidisplacement} can be neglected.  The resulting
strain tensor, known as small strain tensor \vect{\epsilon}, can be given as,

\begin{equation}
  \vect{\delta \epsilon}=\delby{\vect{\delta u}}{\vect{z}} + \transpose{\delby{\vect{\delta u}}{\vect{z}}} 
  \label{eqn:virtualsmalldisplacement}
\end{equation}
 
Since both $\vect{\sigma}$ and $\vect{\delta \epsilon}$ are symmetric, new
vectors are defined by inserting tensor components as follows,

\begin{equation}
  \vect{\delta \epsilon}=\transpose{\sqbrac{\delta \epsilon_{11} \hspace{4 pt} \delta \epsilon_{22} \hspace{4 pt} \delta \epsilon_{33} 
      \hspace{4 pt} 2\delta \epsilon_{12} \hspace{4 pt} 2\delta \epsilon_{23} \hspace{4 pt} 2\delta \epsilon_{13}}} :
  \vect{\sigma}=\transpose{\sqbrac{\delta \sigma_{11} \hspace{4 pt} \delta \sigma_{22} \hspace{4 pt} \delta \sigma_{33} 
      \hspace{4 pt} 2\delta \sigma_{12} \hspace{4 pt} 2\delta \sigma_{23} \hspace{4 pt} 2\delta \sigma_{13} }}	  		  
  \label{eqn:newvectors}
\end{equation} 

Substituting \eqnref{eqn:newvectors} into \eqnref{eqn:virtualenergy},

\begin{equation}
  \delta W_{int}=\gint{0}{v}{\pbrac{\transpose{\vect{\delta \epsilon}} \vect{\sigma}}}v
  \label{eqn:virtualenergy1}
\end{equation}

The strain vector $\vect{\delta \epsilon}$ can be related to displacement
vector using the following equation,

\begin{equation}
  \vect{\delta \epsilon}=\matr{D} \vect{\delta u} 
  \label{eqn:virtualsmalldisplacement1}
\end{equation}

\noindent where $\matr{D}$ and $\vect{u}$ are linear differential operator and
displacement vector respectively and given by,

\begin{equation}
  \begin{array}{c} \matr{D} \end{array} =
  \pbrac{ \begin{array}{ccc} \delby{}{z_{1}} & 0 & 0 \\ 
      0 & \delby{}{z_{2}} & 0 \\
      0 & 0 & \delby{}{z_{3}} \\
      \delby{}{z_{2}} & \delby{}{z_{1}} & 0 \\ 
      0 & \delby{}{z_{3}} & \delby{}{z_{2}} \\ 
      \delby{}{z_{3}} & 0 & \delby{}{z_{1}} \\ \end{array} }
  \label{eqn:differentialoperator}
\end{equation}

\begin{equation}
  \vect{\delta u}=\transpose{\pbrac{\delta u_{1} \hspace{4 pt} \delta u_{2} \hspace{4 pt} \delta u_{3}}}
  \label{eqn:displacementvector}
\end{equation}

The virtual displacement is a finite element field and hence the value at any
point can be obtained by interpolating nodal virtual displacements.

\begin{equation}
  \vect{\delta u}=\matr{\Phi}\matr{\Delta}
  \label{eqn:interpolation}
\end{equation}

