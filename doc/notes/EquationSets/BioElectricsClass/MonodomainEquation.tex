
\subsection{Monodomain Equation}
 

\renewcommand{\vec}[1]{\boldsymbol{#1}}
\newcommand{\Iion}{I_{\text{ion}}}
\newcommand{\Iext}{I_{\text{ext}}}
\newcommand{\pd}[2]{\frac{\partial #1}{\partial #2}}
\newcommand{\lrbr}[1]{\left(#1\right)}
\newcommand{\dt}{\Delta t}



This section describes the implementation of the monodomain equations and hardcoded cell models in OpenCMISS.

\subsection{Equations}
The monodomain equations are:
\begin{eqnarray}
 \pd{V_m}{t}  &=&  \nabla \cdot \lrbr{\vec{D} \nabla V_m}  -  \frac{1}{C_m}\lrbr{\Iion + \Iext}
\end{eqnarray}

where $V_m$ is the membrane potential, $\vec{D}=\frac{\vec{\sigma}}{\chi C_m}$ is the diffusion coefficient, $\vec{\sigma}$ is the conductivity, $C_m$ the membrane capacitance, $\chi$ the cell's surface area to volume ratio, $\Iion$ the currents across the membrane, and $\Iext$ the external stimulus current.

The numerical method developed to solve the monodomain model uses the idea of operator splitting based on Strang splitting outlined in (Qu \& Garfinkel 1999). The splitting algorithm is used to decouple the cell model from the diffusion part of the monodomain equation. 

The splitting method enables us to solve:
\begin{equation}
\frac{d V_m}{d t} = \nabla \cdot (\mathbf{D}\nabla V)
\label{diff}
\end{equation}

and use the solution of this step in the set of ODEs describing the ion concentrations,
\begin{eqnarray}
\frac{d V_m}{d t} &=& f(t,V_m,\vec{v})\\
\frac{d V_m}{d t} &=&  - \frac{1}{C_m}(I_{ion} + I_{ext}) 
\end{eqnarray}
where $\vec{v}$ is the vector of other variables defining the cell model.
The solution obtained this way is not second-order accurate as normal Strang splitting uses half steps, but the error does not accumulate as
fast since

\subsubsection{Finite element discretization}
Equation \ref{diff} is discretized using a finite element approach, with a Crank-Nicholson discretization in time:\\
\begin{equation}
 (\vec{M}/\dt+\vec{A}/2) \vec{V_m}(t+\dt) = (\vec{M}/\dt-\vec{A}/2) \vec{V_m}(t)
\end{equation}

With the stiffness matrix:
\begin{eqnarray}
A_{kl} &=& \int_v \sum_{ij} D_{ij} \pd{\phi^{(l)}}{x_j} \pd{\phi^{(k)}}{x_i} ~dv
\end{eqnarray}

And the damping matrix (lumped, shown to give better convergence):
\begin{eqnarray}
M_{kl} &=& \delta_{kl}  \int_v \phi^{(k)} ~ dv \\
\end{eqnarray}
Where $\phi^{(k)}$ is the basis function such that $\phi^{(k)}(x_j)=\delta_{kj}$.

\subsection{Example usage}

\subsubsection{Types}
The problem class and type are always \verb!CMISSProblemBioelectricsClass! and \verb!CMISSProblemMonodomainStrangSplittingEquationType! respectively.
The problem subtype defines the cell model, currently either
\verb!CMISSProblemMonodomainBuenoOrovioSubtype! or \verb!CMISSProblemMonodomainTenTusscher06Subtype!.

The equations set class and type are always \verb!CMISSEquationsSetBioelectricsClass! and \verb!CMISSEquationsSetMonodomainSSEquationType! respectively.
The problem subtype defines the cell model, currently either
\verb!CMISSEquationsSetMonodomainBuenoOrovioSubtype! or \verb!CMISSEquationsSetMonodomainTenTusscher06Subtype!.

\subsubsection{Cell models}
There are currently two options for the cell model:
\begin{itemize}
 \item   Bueno-Orovio minimal cell model  with 1 Hz paced initial condition, 4 variables. 
 Integration scheme is forward Euler with an adaptive time step 
Functions \verb!bueno_orovio_initialize!, \verb!bueno_orovio_integrate! are called from \verb!monodomain_equations_routines.f90! in respectively the fields setup and post solve stages. 
 \item   Ten Tusscher and Panfilov (2006) cell model, 19 variables and initial condition as in the euHeart benchmark. Integration scheme is forward Euler with an adaptive time step and a specialized scheme for the $m,d$ gates to prevent them from crossing the steady state solution.
Functions \verb!tentusscher06_initialize!, \verb!tentusscher06_integrate! are used from \verb!monodomain_equations_routines.f90!.
\end{itemize}



\subsubsection{Fields}
The main fields relevant for the example is the material field, which has the following components:
\begin{enumerate}
 \item Activation flag. Set to $1.0$ to have a cell be stimulated (standard $\Iext$) and to $0.0$ for no stimulation. Any other value $c$ will apply $c \Iext$ as stimulus current to that cell, allowing the strength to be tuned manually from the example file.
 \item Fiber ``diffusivity'', i.e. $D_f=\frac{\sigma_f}{\chi C_m}$ where $\sigma_f$ is the conductivity along the fiber direction.
 \item Transverse ``diffusivity'', i.e. $D_t=\frac{\sigma_t}{\chi C_m}$ where $\sigma_t$ is the conductivity orthogonal to the fiber direction.
 \item Fiber direction unit vector $x$.
 \item Fiber direction unit vector $y$.
 \item Fiber direction unit vector $z$ (3D only).
\end{enumerate}
The activation flag and fiber direction are nodal by default, the others have constant interpolation but this can be changed.
Fiber direction is normalized to lenght $1$ in the code, so input does not need to be.

Internally, the dependent field contains a single variable $V_m$ (and its normal which is required by OpenCMISS but not used). The independent
field contains the cell model, with the first component always being the copy of $V_m$ used in the cell model.

\subsection{Internal implementation details}
The diffusion equation is solved using OpenCMISS standard setup for a linear dynamic equation solved using a finite element approach.
\subsubsection{Finite element calculation}
The stiffness and damping matrices are calculated only once, and the update flags are set to false afterwards.
Still, the basis function derivatives are tabulated instead of recalculated in the inner loop, and code for this table this could be used to speed up other equation types.


\subsubsection{Examples}
Current examples using this code are:
\begin{itemize}
 \item \verb!examples/Bioelectrics/MonodomainTP06!
 \item \verb!examples/Bioelectrics/MonodomainBuenoOrovio!
\end{itemize}
The first of these is set up to match the euHeart benchmark problem.

\subsection{Known issues and TODO}
\begin{itemize}
 \item We are planning to implement an adaptive time step.
 \item Lookup tables would speed up the code as well, but there is currently no support for them.
\end{itemize}

\subsection{Testing}
Solutions for spacing $0.5, 0.2, 0.1$mm match a reference implementation, and further tests
with non-axis aligned fibers also match.


%\begin{thebibliography}{}
%\bibitem{cp} Clayton R. H. and Panfilov A.V., Prog. Bio-phys. Mol. Biol. 96(1-3):19-43, 2008
%\bibitem{qg} Qu Z. and Garfinkel A. IEEE Trans Biomed. Eng. 25(4):389-392, 1998
%\bibitem{bo} Bueno-Orovio A., Cherry E.M. and Fenton F. H., J. Theo. Bio. 253(3): 405-628 2008
%\bibitem{tp} ten Tusscher, K. H. W. J and Panfilov A. V., Am. J. Physiol. Heart Circ. Physiol. 291(3):H1088--H1100, 2006
%\end{thebibliography}

